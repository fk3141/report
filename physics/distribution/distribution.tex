\documentclass{ltjsarticle}
%%%package読み込み
\usepackage{amsmath}
\usepackage{amssymb}
\usepackage{amsfonts}
\usepackage{mathtools}
\usepackage{bm}
\usepackage{booktabs}
% \usepackage{tikz} % ★消去: 代わりに graphicx 追加
% \usetikzlibrary{cd}
\usepackage{url}
\usepackage{graphicx} % ★追加: 図を挿入するため
\usepackage{float} % ★追加: 図の位置を制御するため
\usepackage{caption} % ★追加: 図のキャプションを柔軟に扱うため
%\usepackage{xcolor}
\usepackage{ascmac}
\usepackage{tcolorbox}
%\usepackage[dvipdfmx, setpagesize=false, bookmarks=true, bookmarksdepth=tocdepth, bookmarksnumbered=true, colorlinks=true, linkcolor=red]
\usepackage{hyperref}
\usepackage[version=4]{mhchem}
\usepackage{braket} % 追加した
\usepackage{booktabs}
\usepackage{bookmark}
\usepackage{multirow}
%\usepackage[textwidth=45zw,lines=44]{geometry}
%\usepackage{pxjahyper}
%%%黒板太字
\newcommand{\N}{\mathbb{N}}
\newcommand{\Z}{\mathbb{Z}}
\newcommand{\Q}{\mathbb{Q}}
\newcommand{\R}{\mathbb{R}}
\newcommand{\C}{\mathbb{C}}
\newcommand{\F}{\mathbb{F}}
%%%約物
\newcommand{\abs}[1]{\left|#1\right|}
\newcommand{\lr}[1]{\left(#1\right)}
\newcommand{\st}{\; \mathrm{s.t.}\; }
\newcommand{\Ae}{\textrm{-a.e.}} 
%%%繰り返し
\newcommand{\pluss}[3]{#1_{#2}+\cdots+#1_{#3}}
\newcommand{\minuss}[3]{#1_{#2}-\cdots-#1_{#3}}
\newcommand{\timess}[3]{#1_{#2}\times\cdots\times #1_{#3}}
\newcommand{\leqs}[3]{#1_{#2}\leq\cdots\leq #1_{#3}}
\newcommand{\geqs}[3]{#1_{#2}\geq\cdots\geq #1_{#3}}
\newcommand{\opluss}[3]{#1_{#2}\oplus\cdots\oplus #1_{#3}}
\newcommand{\otimess}[3]{#1_{#2}\otimes\cdots\otimes #1_{#3}}
\newcommand{\commas}[3]{#1_{#2},\ldots,#1_{#3}}
%%%微分
\newcommand{\dx}[1]{\mathrm{d}#1}
\newcommand{\ddx}[1]{\frac{\mathrm{d}}{\mathrm{d}#1}}
\newcommand{\dydx}[2]{\frac{\mathrm{d}#1}{\mathrm{d}#2}}
\newcommand{\dydxn}[3]{\frac{\mathrm{d}^{#3}#1}{\mathrm{d}#2^{#3}}}
\newcommand{\del}[2]{\frac{\partial#1}{\partial#2}}
\newcommand{\dell}[2]{\frac{\partial^2#1}{{\partial#2}^2}}
\newcommand{\deln}[3]{\frac{\partial^{#3}#1}{{\partial#2}^{#3}}}
%%%
%%%演算子
%log type
\let\Re\relax
\DeclareMathOperator{\Re}{Re}
\let\Im\relax
\DeclareMathOperator{\Im}{Im}
\DeclareMathOperator{\sgn}{sgn}
\DeclareMathOperator{\sign}{sign}
\DeclareMathOperator{\Supp}{Supp}
\DeclareMathOperator{\tr}{tr}
\DeclareMathOperator{\Tr}{Tr}
\DeclareMathOperator{\Det}{Det}
\DeclareMathOperator{\Log}{Log}
\DeclareMathOperator{\rank}{rank}
\DeclareMathOperator{\diag}{diag}
\DeclareMathOperator{\corank}{corank}
\DeclareMathOperator{\Res}{Res}
\DeclareMathOperator{\Ker}{Ker}
\DeclareMathOperator{\coker}{coker}
\DeclareMathOperator{\Coker}{Coker}
\DeclareMathOperator{\Var}{Var}
\DeclareMathOperator{\Cov}{Cov}
\DeclareMathOperator{\sech}{sech}
\DeclareMathOperator{\csch}{csch}
\DeclareMathOperator{\arcsec}{arcsec}
\DeclareMathOperator{\arccot}{arccot}
\DeclareMathOperator{\arccsc}{arccsc}
\DeclareMathOperator{\arccosh}{arccosh}
\DeclareMathOperator{\arcsinh}{arcsinh}
\DeclareMathOperator{\arctanh}{arctanh}
\DeclareMathOperator{\arcsech}{arcsech}
\DeclareMathOperator{\arccsch}{arccsch}
\DeclareMathOperator{\arccoth}{arccoth}
\DeclareMathOperator{\grad}{grad}
\let\div\relax
\DeclareMathOperator{\div}{div}
\DeclareMathOperator{\rot}{rot}
%\DeclareMathOperator{\GL}{GL} % ★消去 : ここから↓
%\DeclareMathOperator{\SL}{SL}
%\DeclareMathOperator{\Sym}{Sym}
%\DeclareMathOperator{\Aut}{Aut}
%\DeclareMathOperator{\Inn}{Inn}
%\DeclareMathOperator{\Out}{Out}
%\DeclareMathOperator{\id}{id}
%\DeclareMathOperator{\pr}{pr}
%\DeclareMathOperator{\supp}{supp}
%\DeclareMathOperator{\diam}{diam}
%\DeclareMathOperator{\End}{End}
%\DeclareMathOperator{\Cl}{Cl}
%\DeclareMathOperator{\Hom}{Hom} % ★消去 : ここまで↑
%limit type
\DeclareMathOperator*{\argmin}{arg~min}
\DeclareMathOperator*{\argmax}{arg~max}
%%%
%%%定理
\usepackage{amsthm}
\theoremstyle{definition}
\newtheorem{lem}{補題}
\newtheorem*{lem*}{補題}
\newtheorem{prf}{証明}
\newtheorem*{prf*}{証明}
\newtheorem*{ex*}{Example}
\newtheorem*{rem*}{Remark}
\newenvironment{prb}[1]%
{\begin{itembox}[l]{\textbf{問題 #1}}}%
{\end{itembox}}
\newenvironment{sol}[2]%
{\setcounter{lem}{0}
\setcounter{prf}{0}
\par\noindent\textbf{解答 #1} (#2)\par}%
{\par\normalfont}

\renewcommand{\refname}{Reference}


%%%%%%%%%%%%%%%%%%%%%
\numberwithin{equation}{section}
%%%%%%%%%%%%%%%%%%%%%%

\newcounter{boxeddefcounter}
\newenvironment{problem}
{\refstepcounter{boxeddefcounter}\begin{itembox}[l]{問\theboxeddefcounter}}
{\end{itembox}}

%\usepackage[hang,small,bf]{caption}
%\usepackage[subrefformat=parens]{subcaption}
\captionsetup{compatibility=false}


\newcommand{\D}{^\circ\text{C}}
\newcommand{\ka}{\textasciitilde}


\pagestyle{myheadings}
\title{分配係数}
\date{\today}
\author{報告者: No.7 05253011 Fumiya Kashiwai / 柏井史哉\\
共同実験者: No.17 林、No.25 缶、No.27 眞岩}
\begin{document}
\maketitle
\markboth{Physics experiment No.7 05253011 Fumiya Kashiwai / 柏井史哉} {Physics experiment No.7 05253011 Fumiya Kashiwai / 柏井史哉}
%%ここまでタイトル

\newpage
\section{Introduction and Background}

\section{Experimental}
\begin{enumerate}
    \item イオン交換水を煮沸した。沸騰開始したのち、10 min程度煮沸を継続した。その後、放置して室温程度まで冷却した。
    \item S$\phi$rensen油状液 13 mLを2 Lの煮沸した水に溶かし、およそ0.1 Mの\ce{NaOH}\textit{aq.}とした。
    \item S$\phi$rensen油状液 1.3 mLを2 Lの水に溶かし、およそ0.01 Mの\ce{NaOH}\textit{aq.}とした。
    \item テキストの写真と同様に、\ce{NaOH}保存滴定装置を組み立てた。
    \item フタル酸カリウムをそれぞれ300 mg, 30 mg程度精秤し、少量の煮沸した水に溶解したのち、フェノールフタレイン溶液(PP)を2滴滴下し、0.1 M、0.01 M\ce{NaOH}\textit{aq.}により滴定した。溶液のピンク色の呈色が、10 s程度振り混ぜても消失しなくなった点を当量点とした。
    \item フタル酸カリウム1001.5 mgを秤量し、煮沸した水で溶解して500 mLとした。
    \item この溶液を10 mLずつ分取し、PPを2滴加えて0.01 M\ce{NaOH}\textit{aq.}により滴定した。溶液のピンク色の呈色が、10 s程度振り混ぜても消失しなくなった点を当量点とした。
    \item 安息香酸 15.0030 gをtolueneに溶かし、全量を500 mLとした。
    \item すりつき300 mL三角フラスコに、表\ref{table_prop}に示した量の安息香酸溶液、煮沸した水、tolueneを加えた。
    \item ガラス栓で蓋をし、激しく撹拌した。
    \item 3 minごとに撹拌しながら$25\D$の恒温槽に30 min浸した。
    \item 分液漏斗を用いて水槽とtoluene層を分離した。
    \item 各層を10 mLずつ三角フラスコに分取し、PPを2滴加えた。
    \item それぞれの層の溶液を、\ce{NaOH}\textit{aq.}を用いて滴定した。水槽およびa,bのtoluene層は0.01 M\ce{NaOH}\textit{aq.}、それ以外は0.1 M\ce{NaOH}\textit{aq.}を用いて滴定を行った。toluene層に対しては50 mL程度の\ce{EtOH}を加えて滴定を行った。溶液のピンク色の呈色が、10 s程度振り混ぜても消失しなくなった点を当量点とした。
\end{enumerate}

\begin{table}[htp]
\caption{各溶液の組成}
\begin{center}
\begin{tabular}{cccc}
\toprule
ID & Benzoic acid / mL & H2O & toluene\\
\midrule
a & 1 & 100 & 99  \\
b & 2 & 100 & 98  \\
c & 5 & 100 & 95  \\
d & 15 & 100 & 85  \\
e & 25 & 100 & 75  \\
f & 40 & 100 & 60  \\
g & 50 & 100 & 50  \\
h & 70 & 100 & 30  \\
\bottomrule
\end{tabular}
\end{center}
\label{table_prop}
\end{table}%


\section{Result and Discussion}

\begin{table}[htp]
\caption{0.1 M\ce{NaOH}\textit{aq.}の滴定}
\begin{center}
\begin{tabular}{ccccc}
\toprule
フタル酸K / mg & 始点/mL & 終点/mL & 滴下量/ mL & 濃度計算値 / mM\\
\midrule
    309.0 & 3.42 & 19.45 & 16.03 & 94.39 \\
    304.0 & 4.20 & 19.81 & 15.61 & 95.36 \\
    241.8 & 3.22 & 15.92 & 12.70 & 93.23 \\
    196.4 & 3.75 & 14.01 & 10.26 & 93.73 \\
\bottomrule
\end{tabular}
\end{center}
\label{0.1M}
\end{table}%

滴定より決定された実際の濃度は、$94.18 \pm 0.92$ mMであった。

\begin{table}[htp]
\caption{0.01 M\ce{NaOH}\textit{aq.}の滴定}
\begin{center}
\begin{tabular}{ccccc}
\toprule
フタル酸K / mg & 始点/mL & 終点/mL & 滴下量/ mL & 濃度計算値 / mM\\
\midrule
    31.7 & 3.51 & 20.38 & 16.87 & 9.201 \\
    29.4 & 3.42 & 20.68 & 17.26 & 8.341 \\
    19.3 & 2.91 & 17.11 & 14.20 & 6.655 \\
\bottomrule
\end{tabular}
\end{center}
\label{0.01M}
\end{table}%

0.01 Mの溶液に関して、計算された実際の濃度は$8.0 \pm 1.2$ mMと、誤差が非常に大きかった。少量のフタル酸Kを測定する際の誤差が大きくなっていると考えられた。そのため、大きい容量(500 mL)の溶液を調整して滴定を行った。

\begin{table}[htp]
\caption{0.01 M\ce{NaOH}\textit{aq.}の再滴定}
\begin{center}
\begin{tabular}{cccc}
\toprule
始点/mL & 終点/mL & 滴下量/ mL & 濃度計算値 / mM\\
\midrule
    1.32 & 11.57 & 10.25 & 9.569 \\
    11.61 & 21.81 & 10.20 & 9.616 \\
    5.11 & 15.33 & 10.22 & 9.597 \\
\bottomrule
\end{tabular}
\end{center}
\label{0.01M_re}
\end{table}%

これにより決定された実際の濃度は$9.594 \pm 0.024$ mMであり、それぞれ異なる質量のフタル酸Kを用いた場合と矛盾しない。よって、この濃度を標準溶液の濃度として用いる。

これらの\ce{NaOH}溶液を用いた、水層、toluene層それぞれの滴定結果を表\ref{water}、\ref{toluene}に示した。

また、これらの滴定により決定された、各層の濃度を表\ref{concentration}にまとめた。

\ce{NaOH}濃度に含まれる誤差としては、フタル酸Kの質量の誤差、および滴定量に関して、ビュレットの読みや、終点の決定による系統誤差、偶然誤差、双方の影響が考えられる。また、安息香酸の滴定量では、滴定に伴う誤差が同様に考えられる。
このうち、系統誤差、すなわちビュレットの容量の誤差については、偶然誤差と比べて小さいと期待される。そのため、\ce{NaOH}濃度、安息香酸の滴定に要した滴下量に含まれる誤差が独立であると仮定し、誤差範囲を見積もった。

\begin{table}[htp]
\caption{水層の滴定}
\begin{center}
\begin{tabular}{ccccc}
\toprule
ID & 始点/mL & 終点/mL & 滴下量/ mL & 濃度計算値 / mM\\
\midrule
\midrule
        \multirow{3}{*}{a}& 10.98 & 12.10 & 1.12 & 0.9034 \\
        &12.10 & 13.20 & 1.10 & 0.8872 \\
        &13.20 & 14.30 & 1.10 & 0.8872 \\
        \midrule
        \multirow{3}{*}{b}& 8.41 & 10.11 & 1.70 & 1.371 \\
        &10.11 & 11.86 & 1.75 & 1.411 \\
        &11.86 & 13.61 & 1.75 & 1.411 \\
        \midrule
        \multirow{3}{*}{c} & 1.20 & 4.42 & 3.22 & 2.597 \\
        &4.42 & 7.70 & 3.28 & 2.646 \\
        &7.70 & 10.70 & 3.00 & 2.420 \\ 
        \midrule
        \multirow{3}{*}{d} & 5.31 & 11.30 & 5.99 & 4.831 \\
        &11.30 & 17.21 & 5.91 & 4.767 \\
        &17.21 & 23.12 & 5.91 & 4.767 \\
        \midrule
        \multirow{3}{*}{e} & 2.58 & 10.19 & 7.61 & 6.141 \\
        & 10.19 & 17.80 & 7.61 & 6.141 \\
        & 10.10 & 17.70 & 7.60 & 6.133 \\
        \midrule
        \multirow{3}{*}{f} & 4.22 & 14.40 & 10.18 & 8.216 \\
        & 9.11 & 19.23 & 10.12 & 8.168 \\
        & 8.01 & 18.12 & 10.11 & 8.160 \\ 
        \midrule
        \multirow{3}{*}{g} &1.78 & 12.46 & 10.68 & 8.621 \\
        & 12.50 & 23.29 & 10.79 & 8.709 \\
        & 2.55 & 13.39 & 10.84 & 8.750 \\ 
        \midrule
        \multirow{3}{*}{h} & 2.71 & 15.76 & 13.05 & 10.53 \\
        & 7.71 & 20.61 & 12.90 & 10.41 \\
        & 2.02 & 15.17 & 13.15 & 10.61 \\
        \bottomrule
\bottomrule
\end{tabular}
\end{center}
\label{water}
\end{table}%


\begin{table}[htp]
\caption{toluene層の滴定}
\begin{center}
\begin{tabular}{ccccc}
\toprule
ID & 始点/mL & 終点/mL & 滴下量/ mL & 濃度計算値 / mM\\
\midrule
\midrule
        \multirow{3}{*}{a}& 7.92 & 9.98 & 2.06 & 1.976 \\
        &11.05 & 13.21 & 2.16 & 2.072 \\
        &13.21 & 15.20 & 1.99 & 1.909 \\
\midrule % \cmidrule を \midrule に変更
        \multirow{3}{*}{b}& 6.82 & 11.89 & 5.07 & 4.864 \\
        &7.81 & 11.86 & 5.00 & 4.800 \\
        &12.81 & 17.84 & 5.03 & 4.826 \\
\midrule % \cmidrule を \midrule に変更
        \multirow{3}{*}{c} & 3.90 & 5.20 & 1.30 & 12.24 \\
        &5.20 & 6.44 & 1.24 & 11.68 \\
        &14.01 & 15.25 & 1.24 & 11.68 \\ 
\midrule % \cmidrule を \midrule に変更
        \multirow{5}{*}{d} & 15.25 & 19.00 & 3.75 & 35.32 \\
        &3.88 & 17.21 & 3.54 & 33.34 \\
        &7.42 & 10.80 & 3.38 & 31.83 \\
        &15.80 & 19.20 & 3.40 & 32.02 \\
        &12.81 & 16.29 & 3.48 & 32.77 \\
\midrule % \cmidrule を \midrule に変更
        \multirow{3}{*}{e} & 10.84 & 16.52 & 5.68 & 53.49 \\
        & 16.52 & 22.20 & 5.68 & 53.49 \\
        & 1.60 & 7.43 & 5.83 & 54.91 \\
\midrule % \cmidrule を \midrule に変更
        \multirow{3}{*}{f} & 7.43 & 16.90 & 9.47 & 89.19 \\
        & 3.92 & 13.50 & 9.58 & 90.22 \\
        & 13.50 & 23.00 & 9.50 & 80.47 \\ 
\midrule % \cmidrule を \midrule に変更
        \multirow{3}{*}{g} & 2.21 & 14.00 & 11.79 & 111.0 \\
        & 4.51 & 16.29 & 11.78 & 110.9 \\
        & 4.00 & 15.80 & 11.80 & 111.1 \\ 
\midrule % \cmidrule を \midrule に変更
        \multirow{3}{*}{h} & 2.42 & 19.25 & 16.83 & 158.5 \\
        & 1.61 & 18.47 & 16.86 & 158.8 \\
        & 2.22 & 18.99 & 16.77 & 157.9 \\
\bottomrule
\bottomrule
\end{tabular}
\end{center}
\label{toluene}
\end{table}%

\begin{table}[htbp]
  \centering
  \caption{Measurement Data}
  \begin{tabular}{ccccc}
    \toprule
      & $C_W$ / mM & s.d. / mM & $C_T$ / mM & s.d. / mM \\
    \midrule
    a & 0.8926 & 0.0096 & 1.986 & 0.082 \\
    b & 1.398  & 0.024  & 4.829 & 0.036 \\
    c & 2.554  & 0.119  & 11.87 & 0.35  \\
    d & 4.788  & 0.039  & 33.06 & 1.44  \\
    e & 6.135  & 0.016  & 53.96 & 0.97  \\
    f & 8.176  & 0.037  & 89.63 & 1.03  \\
    g & 8.687  & 0.069  & 111.0 & 1.1   \\
    h & 10.51  & 0.10   & 158.4 & 1.6   \\
    \bottomrule
  \end{tabular}
\label{extraction_data}
\end{table}

計算された$C_W$および$C_T$は下の図\ref{CtCw}に示した。$C_w$に関しては、単なる線形関係ではなく上に凸な結果を示した。単なる単量体ではなく、解離や多量体の形成などの機構が関連していると考えられる。
\begin{figure}[htbp]
\begin{center}
\includegraphics[width = 15 cm]{CtCw.png}
\caption{$C_W$および$C_T$の一覧}
\label{CtCw}
\end{center}
\end{figure}

\subsection{課題1}
\begin{align}
    K_a = \frac{[\ce{H^+}][\ce{Bz^-}]}{[\ce{HBz}]} &= C_W \frac{\alpha^2}{1-\alpha}\\
    C_W\alpha^2 + K_a \alpha - K_a &= 0\\
    \alpha &= \frac{-K_a + \sqrt{K_a^2 + 4K_aC_W}}{2C_W}
\end{align}
と計算できる。これを元にそれぞれの溶液について計算したものが、下の表\ref{alpha}である。さらに、これを$C_W$に対してプロットすると図\ref{task1}のようになった。

なお、解離度$\alpha$の誤差は、
\begin{equation*}
    \del{\alpha}{C_W} = -\frac{\alpha}{C_W} + \frac{K_a}{C_W}\frac{1}{\sqrt{K_a^2+4C_WK_A}}
\end{equation*}
を用いて、
\begin{equation*}
    \delta \alpha = \left|-\frac{\alpha}{C_W} + \frac{K_a}{C_W}\frac{1}{\sqrt{K_a^2+4C_WK_A}} \right| \delta C_W
\end{equation*}
として見積もった。

\begin{table}[h]
    \centering
    \caption{Processed Data with Error Analysis}
    \begin{tabular}{lccc}
        \toprule
        Label & Cw ($\times 10^{-3}$) & a & Error ($\sigma_a$) \\
        \midrule
        a & 0.8926 & 0.233  & 0.001 \\
        b & 1.398  & 0.191  & 0.001 \\
        c & 2.554  & 0.146  & 0.003 \\
        d & 4.788  & 0.1086 & 0.0004 \\
        e & 6.135  & 0.0966 & 0.0001 \\
        f & 8.176  & 0.0843 & 0.0002 \\
        g & 8.687  & 0.0819 & 0.0003 \\
        h & 10.51  & 0.0747 & 0.0004 \\
        \bottomrule
    \end{tabular}
    \label{alpha}
\end{table}



\begin{figure}[htbp]
\begin{center}
\includegraphics[width = 15 cm]{task1.png}
\caption{解離度$\alpha$の、$C_W$に対するプロット。エラーバーはs.d.を示す。}
\label{task1}
\end{center}
\end{figure}

\subsubsection{課題2}
\begin{equation}
    \ln{C_T} \simeq \ln{C_W(1-\alpha)} + \ln{K_D}
\end{equation}
となるが、$\ln{C_T}$を$\ln{C_W(1-\alpha)}$に対してプロットすると、図\ref{task2}のようになり、$R^2 = 0.997$で直線関係が認められた。傾きは$a = 1.61$であった。

\begin{figure}[htbp]
\begin{center}
\includegraphics[width = 15 cm]{task2.png}
\caption{$\ln{C_T}$の$\ln{C_W(1-\alpha)}$に対するプロット}
\label{task2}
\end{center}
\end{figure}

$y = x + b$としてプロットを行う。その際、誤差の総和を最小化する$b$を求めると、$y = x + 2.055$となった。

\begin{equation*}
    K_D = \exp{(2.055)} = 7.807
\end{equation*}

\subsection{課題3}
$y = 2x + b$としてプロットを行う。その際、誤差の総和を最小化する$b$を求めると、$y = 2x + 7.706$となった。

\begin{equation*}
    K_{\text{puseudo}} = \exp{(7.706)}/2 = 1111
\end{equation*}
と計算された。傾きは1.61であり、整数値$n = 2$と推定される。

全てが単量体であると仮定するCaseAでは誤差が大きく、
Case Bの方がより合致していると考えられる。

\subsection{課題4}
フィッティング結果を\ref{task4}に示した。
ここで、最小二乗法による傾きと切片の誤差を見積もると、傾き$a =  (1.41 \pm 0.15 ) \times 10^3$、切片$b = 2.15 \pm 0.33$となった。また、これらから$K_D = b = 2.15 \pm 0.33$、$K_M = \frac{a}{2b^2} = 153 \pm 38$となった。

\begin{figure}[htbp]
\begin{center}
\includegraphics[width = 15 cm]{task4.png}
\caption{会合平衡のプロット}
\label{task4}
\end{center}
\end{figure}

\subsection{課題5}

\begin{table}[htbp]
    \centering
    \caption{Concentration Data and Calculated Parameters} % 表が横に長いためフォントを少し小さく設定
    \begin{tabular}{lccccccccc}
        \toprule
        Label & $C_\mathrm{w}$ ($\times 10^{-3}$) & $a$ & $[\mathrm{HBz}]_\mathrm{w}$ & $[\mathrm{Bz}^-]$ & $C_\mathrm{t}$ & $b$ & $[\mathrm{HBz}]_\mathrm{T}$ & $[(\mathrm{HBz})_2]$ 
        hoge\\
        \midrule
        a & 0.8926 & 0.233 & $6.84 \times 10^{-4}$ & $2.08 \times 10^{-4}$ & 0.00199 & 0.299 & $1.39 \times 10^{-3}$ & $2.97 \times 10^{-4}$ \\
        b & 1.398  & 0.191 & $1.13 \times 10^{-3}$ & $2.68 \times 10^{-4}$ & 0.00483 & 0.449 & $2.66 \times 10^{-3}$ & $1.08 \times 10^{-3}$ \\
        c & 2.554  & 0.146 & $2.18 \times 10^{-3}$ & $3.72 \times 10^{-4}$ & 0.0119  & 0.595 & $4.80 \times 10^{-3}$ & $3.53 \times 10^{-3}$ \\
        d & 4.788  & 0.109 & $4.27 \times 10^{-3}$ & $5.20 \times 10^{-4}$ & 0.0331  & 0.731 & $8.89 \times 10^{-3}$ & $1.21 \times 10^{-2}$ \\
        e & 6.135  & 0.0966& $5.54 \times 10^{-3}$ & $5.93 \times 10^{-4}$ & 0.0540  & 0.782 & $1.17 \times 10^{-2}$ & $2.11 \times 10^{-2}$ \\
        f & 8.176  & 0.0843& $7.49 \times 10^{-3}$ & $6.89 \times 10^{-4}$ & 0.0896  & 0.826 & $1.56 \times 10^{-2}$ & $3.70 \times 10^{-2}$ \\
        g & 8.687  & 0.0819& $7.98 \times 10^{-3}$ & $7.11 \times 10^{-4}$ & 0.111   & 0.843 & $1.75 \times 10^{-2}$ & $4.68 \times 10^{-2}$ \\
        h & 10.51  & 0.0747& $9.73 \times 10^{-3}$ & $7.85 \times 10^{-4}$ & 0.158   & 0.866 & $2.11 \times 10^{-2}$ & $6.86 \times 10^{-2}$ \\
        \bottomrule
    \end{tabular}
\end{table}

\subsection{課題6}
課題4で議論した通り、$K_D = b = 2.15 \pm 0.33$、$K_M = \frac{a}{2b^2} = 153 \pm 38$
の程度である。誤差はかなり大きく、これは一つ一つの滴定データのばらつきよりも、フィッティングの際の誤差範囲が大きいことに起因する。


\subsection{課題7}
IRスペクトルを取得する。カルボン酸が水素結合を形成することにより、二量体を形成していると考えられる。
IRスペクトルで、カルボニルに対応する波数が、二量体と単量体で異なると予測されるため、これらを区別することができる。

また、UV-Visでの吸収波長が異なる可能性があり、これにより存在を確認できる可能性がある。




\if0
\begin{thebibliography}{99}
\bibitem{nmr_solvent}
Hugo E. Gottlieb, Vadim Kotlyar, and
Abraham Nudelman, 
NMR Chemical Shifts of Common
Laboratory Solvents as Trace Impurities
J. Org. Chem. 1997, 62, 7512-7515
\bibitem{Wurtz}
\url{https://www.chem-station.com/odos/2009/07/wurtz-wurtz-reaction.html}
\bibitem{biphenyl}
\url{https://www.rsc.org/suppdata/cc/c3/c3cc45132a/c3cc45132a.pdf}
\bibitem{o}
\url{https://www.chemspider.com/Chemical-Structure.15646.html}
\end{thebibliography}
\fi
\end{document}

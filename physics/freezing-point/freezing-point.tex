\documentclass{ltjsarticle}
%%%package読み込み
\usepackage{amsmath}
\usepackage{amssymb}
\usepackage{amsfonts}
\usepackage{mathtools}
\usepackage{bm}
% \usepackage{tikz} % ★消去: 代わりに graphicx 追加
% \usetikzlibrary{cd}
\usepackage{url}
\usepackage{graphicx} % ★追加: 図を挿入するため
\usepackage{float} % ★追加: 図の位置を制御するため
\usepackage{caption} % ★追加: 図のキャプションを柔軟に扱うため
%\usepackage{xcolor}
\usepackage{ascmac}
\usepackage{tcolorbox}
%\usepackage[dvipdfmx, setpagesize=false, bookmarks=true, bookmarksdepth=tocdepth, bookmarksnumbered=true, colorlinks=true, linkcolor=red]
\usepackage{hyperref}
\usepackage[version=4]{mhchem}
\usepackage{braket} % 追加した
\usepackage{booktabs}
\usepackage{bookmark}
%\usepackage[textwidth=45zw,lines=44]{geometry}
%\usepackage{pxjahyper}
%%%黒板太字
\newcommand{\N}{\mathbb{N}}
\newcommand{\Z}{\mathbb{Z}}
\newcommand{\Q}{\mathbb{Q}}
\newcommand{\R}{\mathbb{R}}
\newcommand{\C}{\mathbb{C}}
\newcommand{\F}{\mathbb{F}}
%%%約物
\newcommand{\abs}[1]{\left|#1\right|}
\newcommand{\lr}[1]{\left(#1\right)}
\newcommand{\st}{\; \mathrm{s.t.}\; }
\newcommand{\Ae}{\textrm{-a.e.}} 
%%%繰り返し
\newcommand{\pluss}[3]{#1_{#2}+\cdots+#1_{#3}}
\newcommand{\minuss}[3]{#1_{#2}-\cdots-#1_{#3}}
\newcommand{\timess}[3]{#1_{#2}\times\cdots\times #1_{#3}}
\newcommand{\leqs}[3]{#1_{#2}\leq\cdots\leq #1_{#3}}
\newcommand{\geqs}[3]{#1_{#2}\geq\cdots\geq #1_{#3}}
\newcommand{\opluss}[3]{#1_{#2}\oplus\cdots\oplus #1_{#3}}
\newcommand{\otimess}[3]{#1_{#2}\otimes\cdots\otimes #1_{#3}}
\newcommand{\commas}[3]{#1_{#2},\ldots,#1_{#3}}
%%%微分
\newcommand{\dx}[1]{\mathrm{d}#1}
\newcommand{\ddx}[1]{\frac{\mathrm{d}}{\mathrm{d}#1}}
\newcommand{\dydx}[2]{\frac{\mathrm{d}#1}{\mathrm{d}#2}}
\newcommand{\dydxn}[3]{\frac{\mathrm{d}^{#3}#1}{\mathrm{d}#2^{#3}}}
\newcommand{\del}[2]{\frac{\partial#1}{\partial#2}}
\newcommand{\dell}[2]{\frac{\partial^2#1}{{\partial#2}^2}}
\newcommand{\deln}[3]{\frac{\partial^{#3}#1}{{\partial#2}^{#3}}}
%%%
%%%演算子
%log type
\let\Re\relax
\DeclareMathOperator{\Re}{Re}
\let\Im\relax
\DeclareMathOperator{\Im}{Im}
\DeclareMathOperator{\sgn}{sgn}
\DeclareMathOperator{\sign}{sign}
\DeclareMathOperator{\Supp}{Supp}
\DeclareMathOperator{\tr}{tr}
\DeclareMathOperator{\Tr}{Tr}
\DeclareMathOperator{\Det}{Det}
\DeclareMathOperator{\Log}{Log}
\DeclareMathOperator{\rank}{rank}
\DeclareMathOperator{\diag}{diag}
\DeclareMathOperator{\corank}{corank}
\DeclareMathOperator{\Res}{Res}
\DeclareMathOperator{\Ker}{Ker}
\DeclareMathOperator{\coker}{coker}
\DeclareMathOperator{\Coker}{Coker}
\DeclareMathOperator{\Var}{Var}
\DeclareMathOperator{\Cov}{Cov}
\DeclareMathOperator{\sech}{sech}
\DeclareMathOperator{\csch}{csch}
\DeclareMathOperator{\arcsec}{arcsec}
\DeclareMathOperator{\arccot}{arccot}
\DeclareMathOperator{\arccsc}{arccsc}
\DeclareMathOperator{\arccosh}{arccosh}
\DeclareMathOperator{\arcsinh}{arcsinh}
\DeclareMathOperator{\arctanh}{arctanh}
\DeclareMathOperator{\arcsech}{arcsech}
\DeclareMathOperator{\arccsch}{arccsch}
\DeclareMathOperator{\arccoth}{arccoth}
\DeclareMathOperator{\grad}{grad}
\let\div\relax
\DeclareMathOperator{\div}{div}
\DeclareMathOperator{\rot}{rot}
%\DeclareMathOperator{\GL}{GL} % ★消去 : ここから↓
%\DeclareMathOperator{\SL}{SL}
%\DeclareMathOperator{\Sym}{Sym}
%\DeclareMathOperator{\Aut}{Aut}
%\DeclareMathOperator{\Inn}{Inn}
%\DeclareMathOperator{\Out}{Out}
%\DeclareMathOperator{\id}{id}
%\DeclareMathOperator{\pr}{pr}
%\DeclareMathOperator{\supp}{supp}
%\DeclareMathOperator{\diam}{diam}
%\DeclareMathOperator{\End}{End}
%\DeclareMathOperator{\Cl}{Cl}
%\DeclareMathOperator{\Hom}{Hom} % ★消去 : ここまで↑
%limit type
\DeclareMathOperator*{\argmin}{arg~min}
\DeclareMathOperator*{\argmax}{arg~max}
%%%
%%%定理
\usepackage{amsthm}
\theoremstyle{definition}
\newtheorem{lem}{補題}
\newtheorem*{lem*}{補題}
\newtheorem{prf}{証明}
\newtheorem*{prf*}{証明}
\newtheorem*{ex*}{Example}
\newtheorem*{rem*}{Remark}
\newenvironment{prb}[1]%
{\begin{itembox}[l]{\textbf{問題 #1}}}%
{\end{itembox}}
\newenvironment{sol}[2]%
{\setcounter{lem}{0}
\setcounter{prf}{0}
\par\noindent\textbf{解答 #1} (#2)\par}%
{\par\normalfont}

\renewcommand{\refname}{Reference}


%%%%%%%%%%%%%%%%%%%%%
\numberwithin{equation}{section}
%%%%%%%%%%%%%%%%%%%%%%

\newcounter{boxeddefcounter}
\newenvironment{problem}
{\refstepcounter{boxeddefcounter}\begin{itembox}[l]{問\theboxeddefcounter}}
{\end{itembox}}

%\usepackage[hang,small,bf]{caption}
%\usepackage[subrefformat=parens]{subcaption}
\captionsetup{compatibility=false}


\newcommand{\D}{^\circ\text{C}}
\newcommand{\ka}{\textasciitilde}


\pagestyle{myheadings}
\title{凝固点降下}
\date{\today}
\author{Author: No.7 05253011 Fumiya Kashiwai / 柏井史哉}
\begin{document}
\maketitle
\markboth{Physics experiment No.7 05253011 Fumiya Kashiwai / 柏井史哉} {Physics experiment No.7 05253011 Fumiya Kashiwai / 柏井史哉}
%%ここまでタイトル

\newpage
\section{Purpose and Background}
希薄溶液の性質の一つである、凝固点降下を用いて溶質の分子量を求める。また、誤差の評価を通して、凝固点降下度の測定による、分子量の推定の精度を議論する。



\section{Experimental}
\paragraph{溶媒精製}
\begin{enumerate}
    \item シクロヘキサン200 mL程度を三角フラスコにとり、氷-食塩浴(\ce{NaCl}10w\%)により冷却した。
    \item 溶媒の厚い層が形成されたら、残る液体の溶媒を捨て、溶媒を融解した。
    \item 上記の操作を3回繰り返した。
\end{enumerate}

\paragraph{Naphthaleneの分子量測定}
\begin{enumerate}
    \item 精製したシクロヘキサン30.94 gを測定容器に測りとった。
    \item 撹拌棒を用いて撹拌しながら、氷-食塩浴(\ce{NaCl}10w\%)により冷却した。完全に凝固するまで10 sごとに温度を測定した。
    \item 秤量したナフタレンを測定容器に加え、撹拌棒を用いて撹拌しながら、氷-食塩浴(\ce{NaCl}10w\%)により冷却した。完全に凝固するまで10 sごとに温度を測定した。加えたナフタレンの質量は表\ref{table_naphthalene}に示す。
    \item 3種類の質量のナフタレンについて測定を行った。
\end{enumerate}

\paragraph{Azobenzeneの分子量測定}
\begin{enumerate}
    \item 精製したシクロヘキサン31.64 gを測定容器に測りとった。
    \item 撹拌棒を用いて撹拌しながら、氷-食塩浴(\ce{NaCl}10w\%)により冷却した。完全に凝固するまで1 sごとに温度を測定した。
    \item 秤量したアゾベンゼンを測定容器に加え、撹拌棒を用いて撹拌しながら、氷-食塩浴(\ce{NaCl}10w\%)により冷却した。完全に凝固するまで1 sごとに温度を測定した。加えたナフタレンの質量は表\ref{table_azobenzene}に示す。
    \item 5種類の質量のアゾベンゼンについて測定を行った。
\end{enumerate}


\section{Result and Discussion}
測定した温度変化をもとに、凝固点を決定した。ナフタレンを加えた時の温度変化は図\ref{Naphthelene_all.png}に、アゾベンゼンを加えた時の温度変化は図\ref{Azobenzene_all}に示した。なお、温度が概ね$9\D$になった時刻を$t = 0$とした。

\begin{figure}[htbp]
\begin{center}
\includegraphics[width = 10 cm]{naphthalene_all.png}
\caption{Naphthaleneを加えたときの温度変化}
\label{Naphthalene_all}
\end{center}
\end{figure}


\begin{figure}[htbp]
\begin{center}
\includegraphics[width = 10 cm]{Azobenzene_all.png}
\caption{Azobenzeneを加えたときの温度変化}
\label{Azobenzene_all}
\end{center}
\end{figure}

凝固前と凝固中の温度変化を、最小二乗法により線形近似し、それらの交点を求めることにより凝固点を決定した。その後、理論式\ref{eq_1}により溶質の分子量を計算した。決定した凝固点は下の表\ref{freezing_point_naphthalene}および表\ref{freezing_point_azobenzene}に示す。なお、アゾベンゼンの105.5 mgおよび201.0 mgについては凝固後の線形関係が見られなかったため除外した。

\begin{table}[htp]
\caption{Naphthaleneの分子量の測定}
\begin{center}
\begin{tabular}{cccc}
\toprule
Solute / mg & freezing point / $\D$ & $\Delta T$ / K & Moleculer weight \\
\midrule
0.0 & 6.475 &  0 & \\
99.0 & 5.928 & -0.547 & 118.17\\
209.8 & 5.355 & -1.120 & 122.25\\
370.6 & 4.506 & -1.969 & 122.89\\
\bottomrule
\end{tabular}
\end{center}
\label{freezing_point_naphthalene}
\end{table}%

\begin{table}[htp]
\caption{Azobenezeの分子量の測定}
\begin{center}
\begin{tabular}{cccc}
\toprule
Solute / mg & freezing point / $\D$ & $\Delta T$ / K & Moleculer weight \\
\midrule
0.0 & 6.475 &  0 & \\
105.5 & - & - & -\\
201.0 & - & - & -\\
306.1 & 5.448 & -1.002 & 194.97\\
402.2 & 5.130 & -1.320 & 194.45\\
507.8 & 4.799 & -1.652 & 196.26\\
\bottomrule
\end{tabular}
\end{center}
\label{freezing_point_azobenzene}
\end{table}%


\subsection{課題1 : 理論式の導出}
溶媒をA、溶質をBとする。さらに、固体は純粋なAであると仮定する。このとき、液体状態のAの化学ポテンシャル$\mu_A^{(l)}$と固体でのポテンシャル$\mu_A^{(s)}$は一致する。以下、$*$で純粋状態のポテンシャルを示す。また、考えている溶液における溶質Bのmol分率を$x_B \ll 1$とする。

熱平衡状態において、次の式が従う。
\begin{align}
    \mu_A^{(s)} &= \mu_A^{(l)} \\
    \mu_A^{(s)} &= \mu_A^{*(s)} \\
    \mu_A^{(l)} &= \mu_A^{*(l)} + RT \ln{\lr{1-x_B}}
\end{align}
さらに、溶媒Aの固体から液体への相転移に伴うエンタルピー、エントロピー変化を$\Delta_t H, \Delta_t S$とする。このとき
\begin{align}
    \mu_A^{*(l)} - \mu_A^{*(s)} = \Delta_t H - T \Delta_t S
\end{align}
であるから
\begin{align}
    0 = \mu_A^{(l)} - \mu_A^{(s)} &= 
    \mu_A^{*(l)} - \mu_A^{*(s)} + RT \ln{\lr{1-x_B}}  \\
    &= \lr{\Delta_t H - T \Delta_t S} + RT \ln{\lr{1-x_B}}\\
    \ln{\lr{1-x_B}} = - \frac{\Delta_t H - T \Delta_t S}{RT} &= -\frac{\Delta_t H}{R} \frac{1}{T} + \frac{\Delta_t S}{R}
\end{align}
ここで、$x_B \ll 1$により$\ln{\lr{1-x_B}} \simeq -x_B$とする。また、$x_B = 0$の時は、純溶媒Aの議論になるので
\begin{align}
    x_B = \frac{\Delta_t H}{R} \frac{1}{T} - \frac{\Delta_t S}{R}\\
    0 = \frac{\Delta_t H}{R} \frac{1}{T_f} - \frac{\Delta_t S}{R}
\end{align}
ただし$T_f$は凝固点[K]。これより
\begin{align}
    x_B = \frac{\Delta_t H}{R} \lr{\frac{1}{T} - \frac{1}{T_f}} &\simeq \frac{\Delta_t H}{R}\frac{T_f-T}{T_f^2}\\
    \Delta T &= \frac{RT_f^2}{\Delta_t H} x_B 
\end{align}
が成立する。ここで、mol分率$x_B$について
\begin{align}
    x_B = \frac{n_B}{n_A + n_B } \sim \frac{n_B}{n_A} = \frac{w_B}{M_B}\frac{M_A}{w_A} = \frac{w_B}{M_B}\frac{M_A}{w_A}
\end{align}
となる。ただし、$M$は分子量を表し、$w_B$は溶媒$w_A$[g]中の溶質Bの質量[g]である。
これより
\begin{align}
    \frac{w_B}{M_B}\frac{M_A}{w_A} = \frac{\Delta_t H}{RT_f^2}\Delta T\\
    M_B = \frac{w_B M_A}{w_A}\frac{RT_f^2}{\Delta_t H \Delta T}\\
    M_B = \frac{w_B}{w_A} \frac{RT_f^2}{\frac{\Delta_t H}{M_A}\Delta T} 
\end{align}
$\displaystyle K_f = \frac{RT_f^2}{1000\frac{\Delta_t H}{M_A}} = \frac{RT_f^2}{1000Q}$とすると、
\begin{equation}
    M_B = K_f \frac{1000 w_B}{w_A\Delta T_f} = K_f \frac{1000 b}{G \Delta T_f} \label{eq_1}
\end{equation}
を得る。ここで、シクロヘキサンについて$K_f = 20.2 $が知られている。ただし、記号を$w_A = G, w_B = g$と置き直した。  

\subsection{課題2 : 誤差の考察}
今回の実験において生じる統計誤差としては、大きく4つが挙げられる。溶質および溶媒の質量の計測、温度変化の測定、凝固点の測定のための線形近似、およびである。

このうち、一つ目の質量の計測における誤差については、溶媒は実験室の最小0.01 g、溶媒は最小0.1 mgの精密天秤を用いて測定を行ったため、この誤差は質量の$0.1\%$程度に収まっていると考えられ、この誤差は支配的ではない。

温度変化の測定誤差については、温度計の校正による誤差と、溶媒の局所的な冷却による誤差の影響が挙げられる。前者についてはデジタル温度計であり、誤差については不明である。後者については、教科書に倣って空気層を隔てて氷冷したとともに、激しく撹拌を行ったが、温度計が計測している部分が溶液の最下部であったことから、局所的な冷却の影響がみられた。この影響について、定量的に評価することは難しいが、特に105.5 mgおよび201.0 mgのAzobezeneを溶解した場合については、凝固開始後に複数回の溶液温度の上昇が見られたことから、局所的な過冷却が生じていたと考えられる。理想的には、凝固前の直線領域および、凝固中の線形領域からなるはずである。これから逸脱している場合には、誤差が大きいと考えられる。

凝固点降下の理論式\ref{eq_1}における近似において、$\ln{\lr{1-x_B}} = -x_B$などの、溶質のmol分率が非常に小さいことを仮定した近似式を複数用いている。しかしながら、推定される$x_B$は0.01以下であり、この近似による誤差は$1\%$以下と考えられる。

\subsection{課題3 : シクロヘキサンに不純物が含まれた場合の影響}
シクロヘキサンに不純物が溶けており、この不純物が溶質と相互作用しない場合には、不純物のmol分率を$x_{\text{impurity}}$とした時、溶質のmol分率が$x_{\text{impurity}}+x_{\text{solute}}$として振る舞う。今回の測定においては、溶質を加えていない場合についても、一定の凝固点が観測されなかったため、一定量の不純物が含まれていた可能性がある。この場合、課題4で考察する通り、見かけの分子量が、溶質濃度が小さい場合に小さくもとまることになる。

\subsection{課題4 : 見かけの分子量の濃度に対する依存性}
濃度が大きい場合、溶媒の化学ポテンシャルを、$\mu_A^{(l)} = \mu_A^{*(l)} + RT \ln{\lr{1-x_B}}$として近似することが正当ではなくなる。この近似においては、溶質同士の相互作用を加味していないため、溶媒分子間の相互作用が引力的であれば、見かけの分子量が大きく、斥力的であれば見かけの分子量が小さく見られると考えられる。

\section{Appendix}

\begin{figure}[htbp]
\begin{center}
\includegraphics[width = 10 cm]{Naphthalene_0.png}
\includegraphics[width = 10 cm]{Naphthalene_1.png}
\includegraphics[width = 10 cm]{Naphthalene_2.png}
\includegraphics[width = 10 cm]{Naphthalene_3.png}
\caption{Naphthaleneを加えたときの温度変化}
\label{Naphthalene}
\end{center}
\end{figure}

\begin{figure}[htbp]
\begin{center}
\includegraphics[width = 10 cm]{Azobenzene_0.png}
\includegraphics[width = 10 cm]{Azobenzene_1.png}
\includegraphics[width = 10 cm]{Azobenzene_2.png}
\label{Azobenzene}
\end{center}
\end{figure}


\begin{figure}[htbp]
\begin{center}
\includegraphics[width = 10 cm]{Azobenzene_3.png}
\includegraphics[width = 10 cm]{Azobenzene_4.png}
\includegraphics[width = 10 cm]{Azobenzene_5.png}
\caption{Azobenzeneを加えたときの温度変化}
\label{Azobenzene}
\end{center}
\end{figure}

%%参考文献
\if0
\begin{thebibliography}{99}
\bibitem{nmr_solvent}
Hugo E. Gottlieb, Vadim Kotlyar, and
Abraham Nudelman, 
NMR Chemical Shifts of Common
Laboratory Solvents as Trace Impurities
J. Org. Chem. 1997, 62, 7512-7515
\bibitem{Wurtz}
\url{https://www.chem-station.com/odos/2009/07/wurtz-wurtz-reaction.html}
\bibitem{biphenyl}
\url{https://www.rsc.org/suppdata/cc/c3/c3cc45132a/c3cc45132a.pdf}
\bibitem{o}
\url{https://www.chemspider.com/Chemical-Structure.15646.html}
\end{thebibliography}
\fi
\end{document}

A_solvent
B_solute
\mu_{A}^l = \mu_{A}^s
\mu_{A}^l-\mu_{A}^{l,*} = -RT\ln{1-x_B}
\mu_{A}^s-\mu_{A}^{s,*} = 0
\mu_{A}^{s,*}-\mu_{A}^{l,*} = -RT\ln{1-x_B}

\documentclass{ltjsarticle}
%%%package読み込み
\usepackage{amsmath}
\usepackage{amssymb}
\usepackage{amsfonts}
\usepackage{mathtools}
\usepackage{bm}
\usepackage{booktabs}
% \usepackage{tikz} % ★消去: 代わりに graphicx 追加
% \usetikzlibrary{cd}
\usepackage{url}
\usepackage{graphicx} % ★追加: 図を挿入するため
\usepackage{float} % ★追加: 図の位置を制御するため
\usepackage{caption} % ★追加: 図のキャプションを柔軟に扱うため
%\usepackage{xcolor}
\usepackage{ascmac}
\usepackage{tcolorbox}
%\usepackage[dvipdfmx, setpagesize=false, bookmarks=true, bookmarksdepth=tocdepth, bookmarksnumbered=true, colorlinks=true, linkcolor=red]
\usepackage{hyperref}
\usepackage[version=4]{mhchem}
\usepackage{braket} % 追加した
\usepackage{booktabs}
\usepackage{bookmark}
\usepackage{multirow}
%\usepackage[textwidth=45zw,lines=44]{geometry}
%\usepackage{pxjahyper}
%%%黒板太字
\newcommand{\N}{\mathbb{N}}
\newcommand{\Z}{\mathbb{Z}}
\newcommand{\Q}{\mathbb{Q}}
\newcommand{\R}{\mathbb{R}}
\newcommand{\C}{\mathbb{C}}
\newcommand{\F}{\mathbb{F}}
%%%約物
\newcommand{\abs}[1]{\left|#1\right|}
\newcommand{\lr}[1]{\left(#1\right)}
\newcommand{\st}{\; \mathrm{s.t.}\; }
\newcommand{\Ae}{\textrm{-a.e.}} 
%%%繰り返し
\newcommand{\pluss}[3]{#1_{#2}+\cdots+#1_{#3}}
\newcommand{\minuss}[3]{#1_{#2}-\cdots-#1_{#3}}
\newcommand{\timess}[3]{#1_{#2}\times\cdots\times #1_{#3}}
\newcommand{\leqs}[3]{#1_{#2}\leq\cdots\leq #1_{#3}}
\newcommand{\geqs}[3]{#1_{#2}\geq\cdots\geq #1_{#3}}
\newcommand{\opluss}[3]{#1_{#2}\oplus\cdots\oplus #1_{#3}}
\newcommand{\otimess}[3]{#1_{#2}\otimes\cdots\otimes #1_{#3}}
\newcommand{\commas}[3]{#1_{#2},\ldots,#1_{#3}}
%%%微分
\newcommand{\dx}[1]{\mathrm{d}#1}
\newcommand{\ddx}[1]{\frac{\mathrm{d}}{\mathrm{d}#1}}
\newcommand{\dydx}[2]{\frac{\mathrm{d}#1}{\mathrm{d}#2}}
\newcommand{\dydxn}[3]{\frac{\mathrm{d}^{#3}#1}{\mathrm{d}#2^{#3}}}
\newcommand{\del}[2]{\frac{\partial#1}{\partial#2}}
\newcommand{\dell}[2]{\frac{\partial^2#1}{{\partial#2}^2}}
\newcommand{\deln}[3]{\frac{\partial^{#3}#1}{{\partial#2}^{#3}}}
%%%
%%%演算子
%log type
\let\Re\relax
\DeclareMathOperator{\Re}{Re}
\let\Im\relax
\DeclareMathOperator{\Im}{Im}
\DeclareMathOperator{\sgn}{sgn}
\DeclareMathOperator{\sign}{sign}
\DeclareMathOperator{\Supp}{Supp}
\DeclareMathOperator{\tr}{tr}
\DeclareMathOperator{\Tr}{Tr}
\DeclareMathOperator{\Det}{Det}
\DeclareMathOperator{\Log}{Log}
\DeclareMathOperator{\rank}{rank}
\DeclareMathOperator{\diag}{diag}
\DeclareMathOperator{\corank}{corank}
\DeclareMathOperator{\Res}{Res}
\DeclareMathOperator{\Ker}{Ker}
\DeclareMathOperator{\coker}{coker}
\DeclareMathOperator{\Coker}{Coker}
\DeclareMathOperator{\Var}{Var}
\DeclareMathOperator{\Cov}{Cov}
\DeclareMathOperator{\sech}{sech}
\DeclareMathOperator{\csch}{csch}
\DeclareMathOperator{\arcsec}{arcsec}
\DeclareMathOperator{\arccot}{arccot}
\DeclareMathOperator{\arccsc}{arccsc}
\DeclareMathOperator{\arccosh}{arccosh}
\DeclareMathOperator{\arcsinh}{arcsinh}
\DeclareMathOperator{\arctanh}{arctanh}
\DeclareMathOperator{\arcsech}{arcsech}
\DeclareMathOperator{\arccsch}{arccsch}
\DeclareMathOperator{\arccoth}{arccoth}
\DeclareMathOperator{\grad}{grad}
\let\div\relax
\DeclareMathOperator{\div}{div}
\DeclareMathOperator{\rot}{rot}
%\DeclareMathOperator{\GL}{GL} % ★消去 : ここから↓
%\DeclareMathOperator{\SL}{SL}
%\DeclareMathOperator{\Sym}{Sym}
%\DeclareMathOperator{\Aut}{Aut}
%\DeclareMathOperator{\Inn}{Inn}
%\DeclareMathOperator{\Out}{Out}
%\DeclareMathOperator{\id}{id}
%\DeclareMathOperator{\pr}{pr}
%\DeclareMathOperator{\supp}{supp}
%\DeclareMathOperator{\diam}{diam}
%\DeclareMathOperator{\End}{End}
%\DeclareMathOperator{\Cl}{Cl}
%\DeclareMathOperator{\Hom}{Hom} % ★消去 : ここまで↑
%limit type
\DeclareMathOperator*{\argmin}{arg~min}
\DeclareMathOperator*{\argmax}{arg~max}
%%%
%%%定理
\usepackage{amsthm}
\theoremstyle{definition}
\newtheorem{lem}{補題}
\newtheorem*{lem*}{補題}
\newtheorem{prf}{証明}
\newtheorem*{prf*}{証明}
\newtheorem*{ex*}{Example}
\newtheorem*{rem*}{Remark}
\newenvironment{prb}[1]%
{\begin{itembox}[l]{\textbf{問題 #1}}}%
{\end{itembox}}
\newenvironment{sol}[2]%
{\setcounter{lem}{0}
\setcounter{prf}{0}
\par\noindent\textbf{解答 #1} (#2)\par}%
{\par\normalfont}

\renewcommand{\refname}{Reference}


%%%%%%%%%%%%%%%%%%%%%
\numberwithin{equation}{section}
%%%%%%%%%%%%%%%%%%%%%%

\newcounter{boxeddefcounter}
\newenvironment{problem}
{\refstepcounter{boxeddefcounter}\begin{itembox}[l]{問\theboxeddefcounter}}
{\end{itembox}}

%\usepackage[hang,small,bf]{caption}
%\usepackage[subrefformat=parens]{subcaption}
\captionsetup{compatibility=false}


\newcommand{\D}{^\circ\text{C}}
\newcommand{\ka}{\textasciitilde}


\pagestyle{myheadings}
\title{BET法による吸着表面積の決定}
\date{実験日:2025/12/2}
\author{報告者: No.7 05253011 Fumiya Kashiwai / 柏井史哉\\
共同実験者:No.5 伊藤}
\begin{document}
\maketitle
\markboth{Physics experiment No.7 05253011 Fumiya Kashiwai / 柏井史哉} {Physics experiment No.7 05253011 Fumiya Kashiwai / 柏井史哉}
%%ここまでタイトル

\section{Introduction}
BET法により、活性炭の吸着表面積を測定する。

BET法では、固体表面への多層の吸着を仮定する。この過程のもとで、次の理論式が導かれる。
窒素分圧$x=P_{N2}/P_0$の時の吸着した気体の体積$v$の間に、
\begin{equation}
\frac{1}{v(1-x)} = \frac{1}{v_m}+\frac{1}{v_m c} \lr{\frac{1-x}{x}}
\end{equation}
ただし、$v_m, c$は物質に固有の定数である。この式へのフィッティングにより、定数を求める。

ただし、BET法の適用範囲は$0.05 < x < 0.35$とされていることに注意してプロットを行った。

\section{Experimental}
\begin{figure}[htbp]
\begin{center}
\includegraphics[width = 10 cm]{moshikizu.png}
\caption{実験装置の模式図、テキストより引用}
\label{moshikizu}
\end{center}
\end{figure}

模式図に示したような真空ラインを用いて実験を行った。トラップは常に十分量の液体窒素で冷却した。
\begin{enumerate}
    \item 吸着管を取り付け、ガス球以外の真空ライン内部を10 min程度空引きした。内部圧力は36 Paまで低下した。以下、ゼロ点での圧力としてこの値を用いる。
    \item 空間V (C7、C8、C9 と圧力計の間) に窒素を導入したのち、平衡状態となった時の圧力を$P_1$として記録した。
    \item コックC9を開け、吸着管に窒素を導入した。平衡状態となった時の圧力を$P_2$として記録した。
    \item この測定を4回繰り返し、測定結果を表\ref{dead}にまとめた。
    \item 吸着管を大気圧に開放したのち、82.6 mgの活性炭を吸着管に入れた。
    \item 吸着管を取り付け、真空引きしたのちに液体窒素で冷却した。ガス球以外の真空ライン内部を10 min程度空引きした。
    \item 空間Vに窒素を導入し、平衡状態となった時の圧力を$P_{10}$とした。
    \item コックC9を開け、吸着管に窒素を導入した。平衡状態となった時の圧力を$P_1$として記録した。
    \item この測定を20回繰り返し、$n$回目の測定圧力を$P_{n0}$および$P_n$として記録し、測定結果を表\ref{active}にまとめた。
\end{enumerate}


\section{Results and Discussion}
\begin{table}[h]
\centering
\caption{$P-V$ Data}
\begin{tabular}{rrrrr}
\toprule
$P_1$ ($\mathrm{Pa}$) & $P_2$ ($\mathrm{Pa}$) & $P_1 - P_0$ ($\mathrm{Pa}$) & $P_2 - P_0$ ($\mathrm{Pa}$) & $V_A$ ($\mathrm{cm^3}$) \\
\midrule
12500 & 4260 & 12500 & 4230 & 82.9 \\
7130  & 2450 & 7090  & 2410 & 82.7 \\
4070  & 1410 & 4030  & 1370 & 82.8 \\
2340  & 820  & 2300  & 784  & 82.3 \\
\bottomrule
\end{tabular}
\end{table}

\subsection{死体積の決定}
テキストの式により、4回の測定結果から、$V_A = 82.7 \pm 0.3$cm$^3$と決定された。テキストに従い、活性炭の体積は無視して、これを死体積として扱う。




\subsection{BETへのプロット}


\begin{figure}[htbp]
\begin{center}
\includegraphics[width = 10 cm]{plot1.png}
\caption{実験結果のプロット}
\label{result}
\end{center}
\end{figure}
測定結果は図\ref{result}に記した。吸着した窒素の体積$v$、平衡圧$P$は測定結果をもとに、テキストの処理に従い求めた。その際、圧力のゼロ点として36 Paを引いた値を用いた。

BET法を適用するにあたり、$x$に対して$v(1-x)$をプロットした。この時、短調増加(傾きが正)の領域に限ってBET法は適用可能である\cite{shimadzu}が、この条件を満たしているのは、プロット\ref{Rouquerol}より、$x<0.06$の範囲である。

\begin{figure}[htbp]
\begin{center}
\includegraphics[width = 10 cm]{Rouquerol.png}
\caption{Rouquerol plot}
\label{Rouquerol}
\end{center}
\end{figure}


そこで、$0.01 < x < 0.06$の範囲の5点を用いてプロットを行い、傾きおよび切片を最小二乗法により決定した。

\begin{figure}[htbp]
\begin{center}
\includegraphics[width = 10 cm]{BET.png}
\caption{BET plot}
\label{BET}
\end{center}
\end{figure}

傾き$7.28 \times 10^{-5}$、切片$5.40 \times 10^{-2}$と決定され、これより$v_m = 18.5$[cm$^3$], $c = 741$と計算された。($c$は無単位)

\subsection{窒素分子の吸着表面積}
分子一個が占める体積$v$は
\begin{equation}
v = \frac{28.013 \text{g/mol}}{\rho N_A} = 5.76 \times 10^{-23} \text{[cm$^3$]} 
\end{equation}
FCC構造を作っていると仮定すると、分子の直径を$d$として、$v = \frac{d^3}{\sqrt{2}}$となる。

故に、窒素1分子が占める面積$\sigma$は、
\begin{equation}
    \sigma = \frac{\sqrt{3}}{2}d^2 = 0.163 \text{nm$^2$}
\end{equation}

\subsection{活性炭の吸着表面積}
単分子吸着量$V_m = 18.53$cm$^3$/g より、まず吸着分子数は$\frac{V_m}{22414}$によって決まり、表面積は先ほどの$\sigma$を用いて
\begin{equation}
    S=\frac{V_m}{22414}\times N_A \times \sigma = 81.1 \text{m$^2$/g}
\end{equation}
と決定される。

また、$E_1$については式(9)の係数を1として計算し、

\begin{equation}
    E_1 - E_L =  + RT\ln{c} = 4.23 \text{kJ/mol}
\end{equation}
と決定される。文献値$E_L = 5.6 \text{kJ/mol}$を用いると、$E_1 = 9.8 \text{kJ/mol}$となる。

$c$の値については、文献によりばらつきが見られるが、一般に50~300をとり、500を超える場合にはマイクロポアが疑われる\cite{shimadzu2}。実際、今回の解析ではマイクロポアあるとして、Rouquerol plotを用いる必要があったため、通常の多層の吸着と比較して、マイクロポアが存在し、$c$が大きいするのが合理的である。

\begin{thebibliography}{99}
\bibitem{shimadzu}
\url{https://www.an.shimadzu.co.jp/service-support/technical-support/analysis-basics/powder/lecture/practice/p02/lesson16/index.html}
\bibitem{shimadzu2}
\url{https://www.an.shimadzu.co.jp/service-support/technical-support/analysis-basics/powder/lecture/practice/p02/lesson15/index.html}
\end{thebibliography}
\end{document}
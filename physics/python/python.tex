\documentclass{ltjsarticle}
%%%package読み込み
\usepackage{amsmath}
\usepackage{amssymb}
\usepackage{amsfonts}
\usepackage{mathtools}
\usepackage{bm}
% \usepackage{tikz} % ★消去: 代わりに graphicx 追加
% \usetikzlibrary{cd}
\usepackage{url}
\usepackage{graphicx} % ★追加: 図を挿入するため
\usepackage{float} % ★追加: 図の位置を制御するため
\usepackage{caption} % ★追加: 図のキャプションを柔軟に扱うため
%\usepackage{xcolor}
\usepackage{ascmac}
\usepackage{tcolorbox}
%\usepackage[dvipdfmx, setpagesize=false, bookmarks=true, bookmarksdepth=tocdepth, bookmarksnumbered=true, colorlinks=true, linkcolor=red]
\usepackage{hyperref}
\usepackage[version=4]{mhchem}
\usepackage{braket} % 追加した
\usepackage{booktabs}
\usepackage{bookmark}
%\usepackage[textwidth=45zw,lines=44]{geometry}
%\usepackage{pxjahyper}
%%%黒板太字
\newcommand{\N}{\mathbb{N}}
\newcommand{\Z}{\mathbb{Z}}
\newcommand{\Q}{\mathbb{Q}}
\newcommand{\R}{\mathbb{R}}
\newcommand{\C}{\mathbb{C}}
\newcommand{\F}{\mathbb{F}}
%%%約物
\newcommand{\abs}[1]{\left|#1\right|}
\newcommand{\lr}[1]{\left(#1\right)}
\newcommand{\st}{\; \mathrm{s.t.}\; }
\newcommand{\Ae}{\textrm{-a.e.}} 
%%%繰り返し
\newcommand{\pluss}[3]{#1_{#2}+\cdots+#1_{#3}}
\newcommand{\minuss}[3]{#1_{#2}-\cdots-#1_{#3}}
\newcommand{\timess}[3]{#1_{#2}\times\cdots\times #1_{#3}}
\newcommand{\leqs}[3]{#1_{#2}\leq\cdots\leq #1_{#3}}
\newcommand{\geqs}[3]{#1_{#2}\geq\cdots\geq #1_{#3}}
\newcommand{\opluss}[3]{#1_{#2}\oplus\cdots\oplus #1_{#3}}
\newcommand{\otimess}[3]{#1_{#2}\otimes\cdots\otimes #1_{#3}}
\newcommand{\commas}[3]{#1_{#2},\ldots,#1_{#3}}
%%%微分
\newcommand{\dx}[1]{\mathrm{d}#1}
\newcommand{\ddx}[1]{\frac{\mathrm{d}}{\mathrm{d}#1}}
\newcommand{\dydx}[2]{\frac{\mathrm{d}#1}{\mathrm{d}#2}}
\newcommand{\dydxn}[3]{\frac{\mathrm{d}^{#3}#1}{\mathrm{d}#2^{#3}}}
\newcommand{\del}[2]{\frac{\partial#1}{\partial#2}}
\newcommand{\dell}[2]{\frac{\partial^2#1}{{\partial#2}^2}}
\newcommand{\deln}[3]{\frac{\partial^{#3}#1}{{\partial#2}^{#3}}}
%%%
%%%演算子
%log type
\let\Re\relax
\DeclareMathOperator{\Re}{Re}
\let\Im\relax
\DeclareMathOperator{\Im}{Im}
\DeclareMathOperator{\sgn}{sgn}
\DeclareMathOperator{\sign}{sign}
\DeclareMathOperator{\Supp}{Supp}
\DeclareMathOperator{\tr}{tr}
\DeclareMathOperator{\Tr}{Tr}
\DeclareMathOperator{\Det}{Det}
\DeclareMathOperator{\Log}{Log}
\DeclareMathOperator{\rank}{rank}
\DeclareMathOperator{\diag}{diag}
\DeclareMathOperator{\corank}{corank}
\DeclareMathOperator{\Res}{Res}
\DeclareMathOperator{\Ker}{Ker}
\DeclareMathOperator{\coker}{coker}
\DeclareMathOperator{\Coker}{Coker}
\DeclareMathOperator{\Var}{Var}
\DeclareMathOperator{\Cov}{Cov}
\DeclareMathOperator{\sech}{sech}
\DeclareMathOperator{\csch}{csch}
\DeclareMathOperator{\arcsec}{arcsec}
\DeclareMathOperator{\arccot}{arccot}
\DeclareMathOperator{\arccsc}{arccsc}
\DeclareMathOperator{\arccosh}{arccosh}
\DeclareMathOperator{\arcsinh}{arcsinh}
\DeclareMathOperator{\arctanh}{arctanh}
\DeclareMathOperator{\arcsech}{arcsech}
\DeclareMathOperator{\arccsch}{arccsch}
\DeclareMathOperator{\arccoth}{arccoth}
\DeclareMathOperator{\grad}{grad}
\let\div\relax
\DeclareMathOperator{\div}{div}
\DeclareMathOperator{\rot}{rot}
%\DeclareMathOperator{\GL}{GL} % ★消去 : ここから↓
%\DeclareMathOperator{\SL}{SL}
%\DeclareMathOperator{\Sym}{Sym}
%\DeclareMathOperator{\Aut}{Aut}
%\DeclareMathOperator{\Inn}{Inn}
%\DeclareMathOperator{\Out}{Out}
%\DeclareMathOperator{\id}{id}
%\DeclareMathOperator{\pr}{pr}
%\DeclareMathOperator{\supp}{supp}
%\DeclareMathOperator{\diam}{diam}
%\DeclareMathOperator{\End}{End}
%\DeclareMathOperator{\Cl}{Cl}
%\DeclareMathOperator{\Hom}{Hom} % ★消去 : ここまで↑
%limit type
\DeclareMathOperator*{\argmin}{arg~min}
\DeclareMathOperator*{\argmax}{arg~max}
%%%
%%%定理
\usepackage{amsthm}
\theoremstyle{definition}
\newtheorem{lem}{補題}
\newtheorem*{lem*}{補題}
\newtheorem{prf}{証明}
\newtheorem*{prf*}{証明}
\newtheorem*{ex*}{Example}
\newtheorem*{rem*}{Remark}
\newenvironment{prb}[1]%
{\begin{itembox}[l]{\textbf{問題 #1}}}%
{\end{itembox}}
\newenvironment{sol}[2]%
{\setcounter{lem}{0}
\setcounter{prf}{0}
\par\noindent\textbf{解答 #1} (#2)\par}%
{\par\normalfont}

\renewcommand{\refname}{Reference}


%%%%%%%%%%%%%%%%%%%%%
\numberwithin{equation}{section}
%%%%%%%%%%%%%%%%%%%%%%

\newcounter{boxeddefcounter}
\newenvironment{problem}
{\refstepcounter{boxeddefcounter}\begin{itembox}[l]{問\theboxeddefcounter}}
{\end{itembox}}


\usepackage{caption}
\usepackage[subrefformat=parens]{subcaption}
\captionsetup{format=hang, font=small, labelfont=bf, compatibility=false}

\newcommand{\D}{^\circ\text{C}}
\newcommand{\ka}{\textasciitilde}


\pagestyle{myheadings}
\title{計算機課題/Pythonプログラミング}
\date{\today}
\author{Author: No.7 05253011 Fumiya Kashiwai / 柏井史哉}
\begin{document}
\maketitle
\markboth{Physics experiment No.7 05253011 Fumiya Kashiwai / 柏井史哉} {Physics experiment No.7 05253011 Fumiya Kashiwai / 柏井史哉}
%%ここまでタイトル

\newpage
\part{ODE-3 : Belousov-Zhabotinsky Reaction}
\section{Purpose and Background}
Pythonプログラミングを用いて、複雑な反応速度式を有するBelousov-Zhabotinsky (BZ) 反応における、各化学種の濃度変化を数値的に解く。


\section{基本の条件}
まず、テキストに記載された速度定数、初期濃度を用いて計算を行った。[\ce{HBrO2}], [\ce{Br-}], [\ce{Ce^{4+}}]の初期濃度を$[0, 0.0015, 0.0015]$ (M) とした(条件1)。
\begin{figure}[htbp]
\begin{center}
\includegraphics[width = 10 cm]{ODE3_normal.png}
\caption{初濃度$[0, 0.0015, 0.0015]$、$\Delta t = 0.001$ s}
\label{Naphthalene_all}
\end{center}
\end{figure}

条件1では、100 s程度の周期での振動が見られた。

\section{初期濃度の変更}

\begin{figure}[htbp]
 \begin{minipage}[b]{0.3\linewidth}
  \centering
  \includegraphics[keepaspectratio, scale=0.3]{ODE3_HBrO2_0.png}
  \subcaption{}
 \end{minipage}
 \begin{minipage}[b]{0.3\linewidth}
  \centering
  \includegraphics[keepaspectratio, scale=0.3]{ODE3_HBrO2_3.png}
  \subcaption{}
 \end{minipage}
  \begin{minipage}[b]{0.3\linewidth}
  \centering
  \includegraphics[keepaspectratio, scale=0.3]{ODE3_HBrO2_6.png}
  \subcaption{}
 \end{minipage}
 \caption{[\ce{HBrO2}]を変化させた時の挙動}
 \label{ODE3_HBrO2}
\end{figure}


\begin{figure}[htbp]
 \begin{minipage}[b]{0.3\linewidth}
  \centering
  \includegraphics[keepaspectratio, scale=0.3]{ODE3_Br_0.png}
  \subcaption{}
 \end{minipage}
 \begin{minipage}[b]{0.3\linewidth}
  \centering
  \includegraphics[keepaspectratio, scale=0.3]{ODE3_Br_3.png}
  \subcaption{}
 \end{minipage}
  \begin{minipage}[b]{0.3\linewidth}
  \centering
  \includegraphics[keepaspectratio, scale=0.3]{ODE3_Br_6.png}
  \subcaption{}
 \end{minipage}
 \caption{[\ce{Br-}]を変化させた時の挙動}
 \label{ODE3_Br}
\end{figure}

[\ce{HBrO2}], [\ce{Br-}]の初期濃度を変化させた時、初期の挙動には違いが見られものの、振動周期はほぼ変化しなかった。これは、\ce{Br}種のうち、\ce{BrO3-}の濃度が非常に大きいため、[\ce{HBrO2}], [\ce{Br-}]の変化はほぼ影響しないためと考えられる。

対して、\ce{Ce^{4+}}の初期濃度を大きくすると、振動周期の増大が確認された。これは、(定性的には)\ce{Ce^{4+}}を消費するのに時間をより要することから説明される。


\begin{figure}[htbp]
 \begin{minipage}[b]{0.3\linewidth}
  \centering
  \includegraphics[keepaspectratio, scale=0.3]{ODE3_Ce_0.png}
  \subcaption{}
 \end{minipage}
 \begin{minipage}[b]{0.3\linewidth}
  \centering
  \includegraphics[keepaspectratio, scale=0.3]{ODE3_Ce_3.png}
  \subcaption{}
 \end{minipage}
  \begin{minipage}[b]{0.3\linewidth}
  \centering
  \includegraphics[keepaspectratio, scale=0.3]{ODE3_Ce_6.png}
  \subcaption{}
 \end{minipage}
 \caption{[\ce{Ce^{4+}}]を変化させた時の挙動}
 \label{ODE3_Ce}
\end{figure}

このモデルでは、簡略化に伴い\ce{Ce4+}の初期濃度が0の時にも、$\dydx{[\ce{Ce4+}]}{t} \neq 0$となってしまう。そのため、本来振動しないはずの状況でも、振動している計算結果が出力されることが確認された。


\section{反応速度定数の変更}



\section{Runge-Kutta法とEuler法の比較}
ともに$\Delta t = 0.0001$ sとして、Runge-Kutta法とEuler法それぞれで計算をした。

\begin{figure}[htbp]
\begin{center}
\includegraphics[width = 10 cm]{ODE3_compare.png}
\caption{Runge-Kutta法とEuler法の計算値}
\label{compare}
\end{center}
\end{figure}

\begin{figure}[htbp]
\begin{center}
\includegraphics[width = 10 cm]{ODE3_relative_error.png}
\caption{Runge-Kutta法とEuler法の計算値の差}
\label{error}
\end{center}
\end{figure}

\section{Appendix}



%%参考文献
\if0
\begin{thebibliography}{99}
\bibitem{nmr_solvent}
Hugo E. Gottlieb, Vadim Kotlyar, and
Abraham Nudelman, 
NMR Chemical Shifts of Common
Laboratory Solvents as Trace Impurities
J. Org. Chem. 1997, 62, 7512-7515
\bibitem{Wurtz}
\url{https://www.chem-station.com/odos/2009/07/wurtz-wurtz-reaction.html}
\bibitem{biphenyl}
\url{https://www.rsc.org/suppdata/cc/c3/c3cc45132a/c3cc45132a.pdf}
\bibitem{o}
\url{https://www.chemspider.com/Chemical-Structure.15646.html}
\end{thebibliography}
\fi
\end{document}
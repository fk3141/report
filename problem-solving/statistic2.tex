\documentclass{ltjsarticle}
%%%package読み込み
\usepackage{amsmath}
\usepackage{amssymb}
\usepackage{amsfonts}
\usepackage{mathtools}
\usepackage{bm}
% \usepackage{tikz} % ★消去: 代わりに graphicx 追加
% \usetikzlibrary{cd}
\usepackage{url}
\usepackage{graphicx} % ★追加: 図を挿入するため
\usepackage{float} % ★追加: 図の位置を制御するため
\usepackage{caption} % ★追加: 図のキャプションを柔軟に扱うため
%\usepackage{xcolor}
\usepackage{ascmac}
\usepackage{tcolorbox}
%\usepackage[dvipdfmx, setpagesize=false, bookmarks=true, bookmarksdepth=tocdepth, bookmarksnumbered=true, colorlinks=true, linkcolor=red]
\usepackage{hyperref}
\usepackage[version=4]{mhchem}
\usepackage{braket} % 追加した
\usepackage{booktabs}
\usepackage{bookmark}
%\usepackage[textwidth=45zw,lines=44]{geometry}
%\usepackage{pxjahyper}
%%%黒板太字
\newcommand{\N}{\mathbb{N}}
\newcommand{\Z}{\mathbb{Z}}
\newcommand{\Q}{\mathbb{Q}}
\newcommand{\R}{\mathbb{R}}
\newcommand{\C}{\mathbb{C}}
\newcommand{\F}{\mathbb{F}}
%%%約物
\newcommand{\abs}[1]{\left|#1\right|}
\newcommand{\lr}[1]{\left(#1\right)}
\newcommand{\st}{\; \mathrm{s.t.}\; }
\newcommand{\Ae}{\textrm{-a.e.}} 
%%%繰り返し
\newcommand{\pluss}[3]{#1_{#2}+\cdots+#1_{#3}}
\newcommand{\minuss}[3]{#1_{#2}-\cdots-#1_{#3}}
\newcommand{\timess}[3]{#1_{#2}\times\cdots\times #1_{#3}}
\newcommand{\leqs}[3]{#1_{#2}\leq\cdots\leq #1_{#3}}
\newcommand{\geqs}[3]{#1_{#2}\geq\cdots\geq #1_{#3}}
\newcommand{\opluss}[3]{#1_{#2}\oplus\cdots\oplus #1_{#3}}
\newcommand{\otimess}[3]{#1_{#2}\otimes\cdots\otimes #1_{#3}}
\newcommand{\commas}[3]{#1_{#2},\ldots,#1_{#3}}
%%%微分
\newcommand{\dx}[1]{\mathrm{d}#1}
\newcommand{\ddx}[1]{\frac{\mathrm{d}}{\mathrm{d}#1}}
\newcommand{\dydx}[2]{\frac{\mathrm{d}#1}{\mathrm{d}#2}}
\newcommand{\dydxn}[3]{\frac{\mathrm{d}^{#3}#1}{\mathrm{d}#2^{#3}}}
\newcommand{\del}[2]{\frac{\partial#1}{\partial#2}}
\newcommand{\dell}[2]{\frac{\partial^2#1}{{\partial#2}^2}}
\newcommand{\deln}[3]{\frac{\partial^{#3}#1}{{\partial#2}^{#3}}}
%%%
%%%演算子
%log type
\let\Re\relax
\DeclareMathOperator{\Re}{Re}
\let\Im\relax
\DeclareMathOperator{\Im}{Im}
\DeclareMathOperator{\sgn}{sgn}
\DeclareMathOperator{\sign}{sign}
\DeclareMathOperator{\Supp}{Supp}
\DeclareMathOperator{\tr}{tr}
\DeclareMathOperator{\Tr}{Tr}
\DeclareMathOperator{\Det}{Det}
\DeclareMathOperator{\Log}{Log}
\DeclareMathOperator{\rank}{rank}
\DeclareMathOperator{\diag}{diag}
\DeclareMathOperator{\corank}{corank}
\DeclareMathOperator{\Res}{Res}
\DeclareMathOperator{\Ker}{Ker}
\DeclareMathOperator{\coker}{coker}
\DeclareMathOperator{\Coker}{Coker}
\DeclareMathOperator{\Var}{Var}
\DeclareMathOperator{\Cov}{Cov}
\DeclareMathOperator{\sech}{sech}
\DeclareMathOperator{\csch}{csch}
\DeclareMathOperator{\arcsec}{arcsec}
\DeclareMathOperator{\arccot}{arccot}
\DeclareMathOperator{\arccsc}{arccsc}
\DeclareMathOperator{\arccosh}{arccosh}
\DeclareMathOperator{\arcsinh}{arcsinh}
\DeclareMathOperator{\arctanh}{arctanh}
\DeclareMathOperator{\arcsech}{arcsech}
\DeclareMathOperator{\arccsch}{arccsch}
\DeclareMathOperator{\arccoth}{arccoth}
\DeclareMathOperator{\grad}{grad}
\let\div\relax
\DeclareMathOperator{\div}{div}
\DeclareMathOperator{\rot}{rot}
%\DeclareMathOperator{\GL}{GL} % ★消去 : ここから↓
%\DeclareMathOperator{\SL}{SL}
%\DeclareMathOperator{\Sym}{Sym}
%\DeclareMathOperator{\Aut}{Aut}
%\DeclareMathOperator{\Inn}{Inn}
%\DeclareMathOperator{\Out}{Out}
%\DeclareMathOperator{\id}{id}
%\DeclareMathOperator{\pr}{pr}
%\DeclareMathOperator{\supp}{supp}
%\DeclareMathOperator{\diam}{diam}
%\DeclareMathOperator{\End}{End}
%\DeclareMathOperator{\Cl}{Cl}
%\DeclareMathOperator{\Hom}{Hom} % ★消去 : ここまで↑
%limit type
\DeclareMathOperator*{\argmin}{arg~min}
\DeclareMathOperator*{\argmax}{arg~max}
%%%
%%%定理
\usepackage{amsthm}
\theoremstyle{definition}
\newtheorem{lem}{補題}
\newtheorem*{lem*}{補題}
\newtheorem{prf}{証明}
\newtheorem*{prf*}{証明}
\newtheorem*{ex*}{Example}
\newtheorem*{rem*}{Remark}
\newenvironment{prb}[1]%
{\begin{itembox}[l]{\textbf{問題 #1}}}%
{\end{itembox}}
\newenvironment{sol}[2]%
{\setcounter{lem}{0}
\setcounter{prf}{0}
\par\noindent\textbf{解答 #1} (#2)\par}%
{\par\normalfont}

\renewcommand{\refname}{Reference}


%%%%%%%%%%%%%%%%%%%%%
\numberwithin{equation}{section}
%%%%%%%%%%%%%%%%%%%%%%

\newcounter{boxeddefcounter}
\newenvironment{problem}
{\refstepcounter{boxeddefcounter}\begin{itembox}[l]{問\theboxeddefcounter}}
{\end{itembox}}


\usepackage{caption}
\usepackage[subrefformat=parens]{subcaption}
\captionsetup{format=hang, font=small, labelfont=bf, compatibility=false}

\newcommand{\D}{^\circ\text{C}}
\newcommand{\ka}{\textasciitilde}


\pagestyle{myheadings}
\title{統計力学2 レポート}
\date{\today}
\author{理学部化学科 3年 05253011 Fumiya Kashiwai / 柏井史哉}
\begin{document}
\maketitle
\markboth{統計力学2レポート 理学部化学科3年 05253011 Fumiya Kashiwai / 柏井史哉} {統計力学2レポート 理学部化学科3年 05253011 Fumiya Kashiwai / 柏井史哉}
%%ここまでタイトル

\section{平均場近似の変分法としての定式化}
(a) 
$t>0$に対して
\begin{equation}
    \ln{\frac{1}{t}} \ge 1-t
\end{equation}
である。よって、
\begin{equation}
    D_{\text{KL}}(P||Q) 
    = \sum_{S \in \Omega} P(S) \ln{\frac{P(S)}{Q(S)}  }
    \ge \sum_{S \in \Omega} P(S) \lr{1-\frac{Q(S)}{P(S)}} 
    = \sum_{S \in \Omega} \lr{P(S) - Q(S)}
    = 0
\end{equation}

(b)
同様に、量子系の密度演算子について
\begin{align}
    \rho = \sum_j p_j \ket{i}\bra{i}, \, \sigma = \sum_i q_i \ket{j}\bra{j}
\end{align}
とスペクトル分解できるとする。この時、$\ket{i}, \ket{j}$は必ずしも同じではない正規直交基底である。

これより
\begin{align}
    D_{\text{KL}}(\rho||\sigma) 
    &= \Tr(\rho \ln{\rho}) - \Tr(\rho \ln{\sigma}) = \sum_i p_i \ln{p_i}  - \sum_i \bra{i}\rho \ln{\sigma}\ket{i} \notag \\
    &= \sum_i p_i \ln{p_i} - \sum_i \sum_j p_i \braket{i|j}\bra{j} \ln{\sigma}\ket{i}\notag \\
    &= \sum_i p_i \ln{p_i} - \sum_i \sum_j p_i |\braket{i|j}|^2\ln{q_j}\notag \\
    &= \sum_i \sum_j p_i |\braket{i|j}|^2 \lr{\ln{p_i} - \ln{q_j}}\notag\\
    & \ge \sum_i \sum_j \lr{p_i - q_j} |\braket{i|j}|^2\notag \\
    &= \sum_i p_i \lr{ \sum_j |\braket{i|j}|^2 } - \sum_j q_j \lr{ \sum_i |\braket{i|j}|^2 } \notag\\
    &= 1-1 = 0
\end{align}
が従う。
% p_j = 0 のケア

(c)
まず、古典系の場合、
\begin{align}
    Z = \int \dx{S} e^{-\beta H(S)}, \, Z_0 = \int \dx{S} e^{-\beta H_0(S)}
\end{align}
である。この時、状態空間$S\in \Omega$上の二つの確率変数
\begin{align}
    P(S) = \frac{e^{-\beta H(S)}}{Z},\, P_0(S) = \frac{e^{-\beta H_0(S)}}{Z}
\end{align}
に対して(a)を用いると
\begin{align}
    D_{\text{KL}}(P_0||P) &= \sum_{S\in \Omega} \frac{e^{-\beta H_0(S)}}{Z_0} \ln{\frac{e^{-\beta H_0(S)}}{Z_0}\frac{Z}{e^{-\beta H(S)}}}\notag \\
    &= \ln{Z} - \ln{Z_0} + \sum_{S\in \Omega} \frac{e^{-\beta H_0(S)}}{Z_0} \beta (H(S)-H_0(S))\notag \\
    &= \ln{Z} - \ln{Z_0} +\beta \braket{H-H_0}_0 \ge 0
\end{align}
$\displaystyle F=-\frac{\ln{Z}}{\beta}$より
\begin{equation}
    -F + F_0 + \braket{H-H_0}_0 \ge 0 
\end{equation}
よって、次の式が従う。
\begin{equation}
    F_\nu = F_0 + \braket{H-H_0}_0 \ge F
\end{equation}

%%%%%%%%%%%%%%%%%%
量子系の場合にも、同様にKL-情報量の非負性から従う。
%

(d) 状態空間$S \in \Omega$の要素の総数は$2^N$である。
\begin{align}
    Z_0 &= \sum_{S \in \Omega} e^{\beta H_0} = \sum_{S \in \Omega} \prod_i e^{\beta \Lambda S_i}\notag \\
    &=e^{-\beta \Lambda N} \sum_{S \in \Omega} \prod_i e^{\beta \Lambda S_i+1}\notag \\
    &= e^{-\beta \Lambda N} \sum_{k=0}^N 
        \begin{pmatrix}
            N\\k
        \end{pmatrix}
    e^{2\beta\Lambda k} \notag\\
    &= e^{-\beta \Lambda N} \lr{1+e^{2\beta\Lambda k}}^N = \lr{2\cosh{\beta \Lambda}}^N
\end{align}

よって、
\begin{align}
    \braket{H_0}_0 &= \frac{1}{Z_0} \sum_{S \in \Omega} H_0 e^{-\beta H_0}\notag\\ 
    &= \frac{\Lambda}{\lr{2\cosh{\beta \Lambda}}^N} \sum_{k=0}^N 
        \begin{pmatrix}
            N\\k
        \end{pmatrix}
    (2k-N) e^{-2\beta \Lambda k}\notag\\
    &= \frac{\Lambda e^{\beta \Lambda N}}{\lr{2\cosh{\beta \Lambda}}^N} \lr{2N e^{-2\beta \Lambda} \lr{1+e^{-2\beta\Lambda}}^{N-1}-N\lr{1+e^{-2\beta\Lambda}}^N} \notag\\
    &= \frac{\Lambda Ne^{\beta \Lambda }}{\lr{2\cosh{\beta \Lambda}}}\lr{2e^{-2\beta \Lambda} - \lr{1+e^{-2\beta\Lambda}}} \notag\\
    &= -\frac{\Lambda N}{\lr{2\cosh{\beta \Lambda}}}\lr{e^{\beta \Lambda} - e^{-\beta\Lambda}}\\
    &= -N\Lambda \tanh{\beta \Lambda}
\end{align}
$\braket{H_0}_0 = -\Lambda N m$より 
\begin{equation}
    m = \tanh{\beta \Lambda}
\end{equation}

上の式を満たす$m = m_0$とする。
この時、
\begin{align}
    \braket{H - H_0}_0 &= -J \frac{Nz}{2} m^2 - (B-\Lambda) N m\\
    F_\nu &= - \frac{NzJ}{2} m^2 - (B-\Lambda) N m + F_0 
\end{align}

と表せる。ここで、
\begin{align}
    m &= \tanh{\beta \Lambda} = \frac{e^{\beta \Lambda}-e^{-\beta \Lambda}}{e^{\beta \Lambda} + e^{-\beta \Lambda}}\\
    \Lambda &= \frac{1}{2\beta}\ln{\frac{1+m}{1-m}}
\end{align}
である。

$\displaystyle F_0 = -\frac{\ln{Z_0}}{\beta} = -N \frac{\ln{\lr{\cosh{\beta \Lambda}}}}{\beta}$であり、
この時、

\begin{align}
    F_\nu &= - \frac{NzJ}{2} m^2 - \lr{B- \frac{1}{2\beta}\ln{\frac{1+m}{1-m}}} N m - N \frac{\ln{\lr{\cosh{\beta \Lambda}}}}{\beta}\\
    \frac{1}{\cosh^2{\beta\Lambda}} &= 1-\tanh^2{\beta \Lambda} = 1-m^2
\end{align}

であるので、
\begin{align}
    f_\nu &= - \frac{zJ}{2} m^2 - \lr{B- \frac{1}{2\beta}\ln{\frac{1+m}{1-m}}}  m + \frac{\ln{\lr{1-m^2}}}{2\beta}\notag\\
    &= -\frac{zJ}{2} m^2 - Bm - \frac{1}{\beta} \lr{\frac{1+m}{2}\ln{\frac{1+m}{2}} +\frac{1-m}{2}\ln{\frac{1-m}{2}} }\notag\\
    &= -\frac{zJ}{2} m^2 - Bm - T\sigma(m)
\end{align}

を得る。

\begin{align}
    \del{f_\nu}{m} &= -zJm - B - \frac{1}{\beta} \lr{\frac{1}{2}\ln{\frac{1+m}{2}} - \frac{1}{2}\frac\ln{{1-m}{2}} + 1 - 1}\notag \\
    &= -zJm - B - \frac{1}{\beta} \lr{\frac{1}{2}\ln{\frac{1+m}{1-m}}} = 0 \\
    \frac{1}{2\beta}\ln{\frac{1-m}{1+m}} &= zJm + B\\
    m &= \tanh{\beta\lr{zJm + B}}
\end{align}
が成立する。

・長所



\section{階層格子上のIsing 模型と実空間繰り込み群}

(a)

\begin{align}
    Ae^{K'e_i e_j} &= \sum_{e_1,e_2 = \pm 1} e^{K\lr{\lr{\sigma_i + \sigma_j}\lr{\sigma_1 + \sigma_2} + \sigma_1 \sigma_2}}\notag\\
     &= e^{2K\lr{\sigma_i + \sigma_j} + K} + 2e^{-K} + e^{-2K\lr{\sigma_i + \sigma_j}+K}
\end{align}
である。また、$\{ e_i, e_j\} = \{1,1\}, \{1, -1\}$の場合をそれぞれ考えることにより
\begin{align}
    Ae^{K'} &= e^{5K} + 2e^{-K} + e^{-3K}\\
    Ae^{-K'} &= 2\lr{e^{K} + e^{-K}}
\end{align}
を得る。これより
\begin{align}
    e^{2K'} = \frac{Ae^{K'}}{Ae^{-K'}} = \frac{e^{5K} + 2e^{-K} + e^{-3K}}{2\lr{e^{K} + e^{-K}}}\\
    e^{2K} = \frac{1+t}{1-t}
\end{align}
を用いて変形し、
\begin{equation}
    t' = \frac{2t^2}{1-t+t^2+t^3}
\end{equation}
を得る。ただし、$t' = \tanh{K'}, t = \tanh{K}$とした。
$t' = t = t_0$として $t_0 = 0, 1 , -1 \pm \sqrt{2}$を得る。

$-1 \le t_0 \le 1$であるので、$t_0 = 0,1, \sqrt{2}-1$となる。
\begin{equation}
     \frac{2t^2}{1-t+t^2+t^3}-t
\end{equation}
のグラフを考えると、$t = 0,1$の近傍で傾き負、$t = t_c = 
\sqrt{2}-1$の近傍で傾き正である。

ゆえに、$t = 0,1$は安定点、$t = t_c$は不安定点である。
$\tanh{K_c} = t_c = \sqrt{2}-1$とすると、
\begin{align}
    e^{2K_c} = \frac{1+t}{1-t} = \sqrt{2}+1\\ 
    K_c = \frac{1}{2} \ln{\lr{\sqrt{2}+1}}
\end{align}

固定点の近傍において、
\begin{align}
    t = \tanh{K} = \sqrt{2}-1 + \frac{1}{\cosh^2{K_c}}(K-K_c)
\end{align}
と線形化することにすると、
\begin{align}
    \lambda_t = 
    \left. \dydx{K'}{K} \right|_{K=K_c} = \left.\dydx{t'}{t} \right|_{t=t_c} 
    &= \frac{4t_c(1-t_c+t_c^2+t_c^3)-2t_c^2(-1+2t_c+3t_c^2)}{\lr{1-t_c+t_c^2+t_c^3}^2} \notag\\
    &= \frac{4t_c\cdot 2t_c - 2t_c^2(-1+2t_c+3t_c^2)}{4t_c^2}\notag\\
    &= \frac{5-2t_c-3t_c^2}{2}\notag\\
    &= 1+ 2t_c\notag\\
    &= 2\sqrt{2}-1 \simeq 1.83\\
    y_t &=\frac{\ln{\lambda_t}}{\ln{2}} =\frac{\ln{\lr{2\sqrt{2}-1}}}{\ln{2}} \simeq 0.87\\
    \nu &= \frac{1}{y_t} \simeq 1.15
\end{align}


$K \propto 1/T$であることから、$T_c,K_c$の相対誤差の値は等しく、0である。
\begin{align}
    \frac{\nu^{\text{MK}}-\nu^{\text{exact}}}{\nu^{\text{exact}}} \simeq 15\%
\end{align}
となる。$K_c$あるいは臨界温度$T_c$は正確に計算されており、相関長臨界係数は誤差15\%程度であることがわかった。

(d)
\begin{equation}
    t = \tanh^2{\lr{4 \arctanh{(t)}}}
\end{equation}
を数値的に(Wolfram Alphaを使って)解くと、

$t = 0, 0.0655642580585243..., 0.992326060468120...$ を得る。非自明な固定点$t_c = 0.06556$について、これも不安定な固定点であり、
\begin{align}
    K_c^{\text{(MK)}} = \arctanh{0.06556} &= 0.06565\\
    y_t^{\text{(MK)}} &= \left. \dydx{t'}{t} \right|_{t=t_c} = 1.9058\\
    \nu^{\text{(MK)}} &= 0.5247
\end{align}
となる。

(e)
まだ

\section{量子Ising 模型とMajorana chain}
(a)
仮定の下で
\begin{equation}
    H = -iJ \sum_{j=0}^{2N} \gamma_{i}\gamma_{i+1} = \sum_{k=1}^{N} \varepsilon_k \lr{d_k^\dagger d_k -\frac{1}{2}}
\end{equation}
となる。ただし、$\gamma_0 = \gamma_{2N+1} = 0$とする。
\begin{equation}
    d_k = \sum_{j=0}^{2N+1} w_{k,j} \gamma_j 
\end{equation}
となるように要請する。ただし$w_{k,j} \in \C$とする。
この時、Heisenberg方程式を考えると
\begin{align}
    [\gamma_i , H ] &= 2i J \lr{\gamma_{i+1}-\gamma_{i-1}}\\
    [d_k, H ] &= \varepsilon_k d_k
\end{align}
となる。$d_k$と$\gamma_k$の間の関係式を用いると、
\begin{equation}
    2i J \sum_{j=0}^{2N+1} w_{k,j}\lr{\gamma_{j+1}-\gamma_{j-1}}=  \sum_{j=0}^{2N+1} w_{k,j} [\gamma_j , H ] = \varepsilon_k \sum_{j=0}^{2N+1} w_{k,j} \gamma_j 
\end{equation}
を得る。$\gamma_i$の係数を比較することによって
\begin{align}
    2iJ (w_{k,j-1} - w_{k,j+1}) = \varepsilon_k w_{k,j}
\end{align}
$i^j u_{k,j} =  w_{k,j}$と改めておくことによって
\begin{align}
    u_{k,j-1} + u_{k,j+1} = \frac{\varepsilon_k}{2J} u_{k,j}
\end{align}
となる。境界条件として、$u_{k,0} = u_{k,2N+1} = 0$が課される。この差分方程式の解として$u_{k,j} \propto e^{i\alpha_k j}$をとると、
\begin{align}
    \lr{e^{-i\alpha_k} + e^{i\alpha_k}} &= \frac{\varepsilon_k}{2J}\\
    \varepsilon_k &= 4J\cos{\alpha_k}
\end{align}
が要請される。さらに、境界条件から、$u_{k,j}$は$\sin{}$型の関数であって、
\begin{align}
    u_{k,j} &\propto \sin{\alpha_k}\\
    \sin{(2N+1)\alpha_k} &= 0\\
    \alpha_{k} &= \frac{k\pi}{2N+1}
\end{align}
となるので、
\begin{equation}
    \varepsilon_k = 4J \cos{\frac{k\pi}{2N+1}}
\end{equation}
となる。

(b)
$d_k$は、励起による準粒子の fermion演算子であり、基底状態においてはすべての$k$で$d_k^\dagger d_k =0$となる。

よって基底状態は
\begin{equation}
    E_0 = -2J \sum_{k=1}^N \cos{\frac{k\pi}{2N+1}} = -2J \frac{\cos{\frac{N\pi}{2(2N+1)}}\sin{\frac{(N+1)\pi}{2(2N+1)}}}{\sin{\frac{\pi}{2(2N+1)}}-1} = -J \lr{\frac{1}{\sin{\frac{\pi}{2(2N+1)}}} -1 }¥
\end{equation}

$N \to \infty$において$E_0$を展開する。
$\sin{x} = x- \frac{x^3}{6} O(x^5)$であるので、$\displaystyle x = \frac{\pi}{2(2N+1)}$として
\begin{align}
    E_0 &\simeq -J \lr{\frac{1}{x(1-\frac{x^2}{6})}-1} \simeq -J \frac{1}{x} \lr{1+\frac{x^2}{6} - x} = -J \lr{\frac{1}{x} + \frac{x}{6} - 1}\notag\\
    &= - J \lr{\frac{4N + 2}{\pi} + \frac{\pi}{12(2N+1)} - 1}\notag\\
    &\simeq - J \lr{\frac{4N + 2}{\pi} + \frac{\pi}{2} + \frac{\pi}{24N} \lr{1-\frac{1}{2N}} - 1}\notag\\
    &= -J \lr{\frac{4N}{\pi} + \frac{2}{\pi} - 1 + \frac{\pi}{24N} + O(N^{-2})
    }\\
    \frac{E_0}{N} &= -J \lr{\frac{4}{\pi} + \frac{1}{N}\lr{\frac{2}{\pi } - 1} + \frac{\pi}{24N^2} + O(N^{-3})}
\end{align}
を得る。よって
\begin{align}
    e_0 = -\frac{4J}{\pi}, \, e_1 = -J \lr{\frac{2}{\pi } - 1},\, e_2 = -\frac{\pi J}{24}
\end{align}

(d) 


\end{document}

\begin{align}
    Z = \int \dx{S} e^{-\beta H(S)} = \int \dx{S} e^{-\beta H_0(S)} e^{\beta \lr{H(S) - H_0(S)}}\\
    = Z_0 \int \dx{S} \frac{e^{-\beta H_0(S)}}{Z_0} e^{\beta \lr{H(S) - H_0(S)}}
    = Z_0 \braket{H-H_0}_0
\end{align}
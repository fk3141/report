\documentclass{ltjsarticle}
%%%package読み込み
\usepackage{amsmath}
\usepackage{amssymb}
\usepackage{amsfonts}
\usepackage{mathtools}
\usepackage{bm}
% \usepackage{tikz} % ★消去: 代わりに graphicx 追加
% \usetikzlibrary{cd}
\usepackage{url}
\usepackage{graphicx} % ★追加: 図を挿入するため
\usepackage{float} % ★追加: 図の位置を制御するため
\usepackage{caption} % ★追加: 図のキャプションを柔軟に扱うため
%\usepackage{xcolor}
\usepackage{ascmac}
\usepackage{tcolorbox}
%\usepackage[dvipdfmx, setpagesize=false, bookmarks=true, bookmarksdepth=tocdepth, bookmarksnumbered=true, colorlinks=true, linkcolor=red]
\usepackage{hyperref}
\usepackage{booktabs}
\usepackage{siunitx}
\usepackage{multirow}
\usepackage[version=4]{mhchem}
\usepackage{braket} % 追加した
\usepackage{booktabs}
\usepackage{bookmark}
\usepackage{caption}
%\usepackage[textwidth=45zw,lines=44]{geometry}
%\usepackage{pxjahyper}
%%%黒板太字
\newcommand{\N}{\mathbb{N}}
\newcommand{\Z}{\mathbb{Z}}
\newcommand{\Q}{\mathbb{Q}}
\newcommand{\R}{\mathbb{R}}
\newcommand{\C}{\mathbb{C}}
\newcommand{\F}{\mathbb{F}}
%%%約物
\newcommand{\abs}[1]{\left|#1\right|}
\newcommand{\lr}[1]{\left(#1\right)}
\newcommand{\st}{\; \mathrm{s.t.}\; }
\newcommand{\Ae}{\textrm{-a.e.}} 
%%%繰り返し
\newcommand{\pluss}[3]{#1_{#2}+\cdots+#1_{#3}}
\newcommand{\minuss}[3]{#1_{#2}-\cdots-#1_{#3}}
\newcommand{\timess}[3]{#1_{#2}\times\cdots\times #1_{#3}}
\newcommand{\leqs}[3]{#1_{#2}\leq\cdots\leq #1_{#3}}
\newcommand{\geqs}[3]{#1_{#2}\geq\cdots\geq #1_{#3}}
\newcommand{\opluss}[3]{#1_{#2}\oplus\cdots\oplus #1_{#3}}
\newcommand{\otimess}[3]{#1_{#2}\otimes\cdots\otimes #1_{#3}}
\newcommand{\commas}[3]{#1_{#2},\ldots,#1_{#3}}
%%%微分
\newcommand{\dx}[1]{\mathrm{d}#1}
\newcommand{\ddx}[1]{\frac{\mathrm{d}}{\mathrm{d}#1}}
\newcommand{\dydx}[2]{\frac{\mathrm{d}#1}{\mathrm{d}#2}}
\newcommand{\dydxn}[3]{\frac{\mathrm{d}^{#3}#1}{\mathrm{d}#2^{#3}}}
\newcommand{\del}[2]{\frac{\partial#1}{\partial#2}}
\newcommand{\dell}[2]{\frac{\partial^2#1}{{\partial#2}^2}}
\newcommand{\deln}[3]{\frac{\partial^{#3}#1}{{\partial#2}^{#3}}}
%%%
%%%演算子
%log type
\let\Re\relax
\DeclareMathOperator{\Re}{Re}
\let\Im\relax
\DeclareMathOperator{\Im}{Im}
\DeclareMathOperator{\sgn}{sgn}
\DeclareMathOperator{\sign}{sign}
\DeclareMathOperator{\Supp}{Supp}
\DeclareMathOperator{\tr}{tr}
\DeclareMathOperator{\Tr}{Tr}
\DeclareMathOperator{\Det}{Det}
\DeclareMathOperator{\Log}{Log}
\DeclareMathOperator{\rank}{rank}
\DeclareMathOperator{\diag}{diag}
\DeclareMathOperator{\corank}{corank}
\DeclareMathOperator{\Res}{Res}
\DeclareMathOperator{\Ker}{Ker}
\DeclareMathOperator{\coker}{coker}
\DeclareMathOperator{\Coker}{Coker}
\DeclareMathOperator{\Var}{Var}
\DeclareMathOperator{\Cov}{Cov}
\DeclareMathOperator{\sech}{sech}
\DeclareMathOperator{\csch}{csch}
\DeclareMathOperator{\arcsec}{arcsec}
\DeclareMathOperator{\arccot}{arccot}
\DeclareMathOperator{\arccsc}{arccsc}
\DeclareMathOperator{\arccosh}{arccosh}
\DeclareMathOperator{\arcsinh}{arcsinh}
\DeclareMathOperator{\arctanh}{arctanh}
\DeclareMathOperator{\arcsech}{arcsech}
\DeclareMathOperator{\arccsch}{arccsch}
\DeclareMathOperator{\arccoth}{arccoth}
\DeclareMathOperator{\grad}{grad}
\let\div\relax
\DeclareMathOperator{\div}{div}
\DeclareMathOperator{\rot}{rot}
%\DeclareMathOperator{\GL}{GL} % ★消去 : ここから↓
%\DeclareMathOperator{\SL}{SL}
%\DeclareMathOperator{\Sym}{Sym}
%\DeclareMathOperator{\Aut}{Aut}
%\DeclareMathOperator{\Inn}{Inn}
%\DeclareMathOperator{\Out}{Out}
%\DeclareMathOperator{\id}{id}
%\DeclareMathOperator{\pr}{pr}
%\DeclareMathOperator{\supp}{supp}
%\DeclareMathOperator{\diam}{diam}
%\DeclareMathOperator{\End}{End}
%\DeclareMathOperator{\Cl}{Cl}
%\DeclareMathOperator{\Hom}{Hom} % ★消去 : ここまで↑
%limit type
\DeclareMathOperator*{\argmin}{arg~min}
\DeclareMathOperator*{\argmax}{arg~max}
%%%
%%%定理
\usepackage{amsthm}
\theoremstyle{definition}
\newtheorem{lem}{補題}
\newtheorem*{lem*}{補題}
\newtheorem{prf}{証明}
\newtheorem*{prf*}{証明}
\newtheorem*{ex*}{Example}
\newtheorem*{rem*}{Remark}
\newenvironment{prb}[1]%
{\begin{itembox}[l]{\textbf{問題 #1}}}%
{\end{itembox}}
\newenvironment{sol}[2]%
{\setcounter{lem}{0}
\setcounter{prf}{0}
\par\noindent\textbf{解答 #1} (#2)\par}%
{\par\normalfont}

\renewcommand{\refname}{Reference}


%%%%%%%%%%%%%%%%%%%%%
\numberwithin{equation}{section}
%%%%%%%%%%%%%%%%%%%%%%

\newcounter{boxeddefcounter}
\newenvironment{problem}
{\refstepcounter{boxeddefcounter}\begin{itembox}[l]{問\theboxeddefcounter}}
{\end{itembox}}

%\usepackage[hang,small,bf]{caption}
%\usepackage[subrefformat=parens]{subcaption}
\captionsetup{compatibility=false}


\newcommand{\D}{^\circ\text{C}}
\newcommand{\ka}{\textasciitilde}


\pagestyle{myheadings}
\title{Inorganic and Analytical experiment XRD}
\date{\today}
\author{Author: No.7 05253011 Fumiya Kashiwai / 柏井史哉}
\begin{document}
\maketitle
\markboth{Inorganic and Analytical experiment XRD  No.7 05253011 Fumiya Kashiwai / 柏井史哉} {Inorganic and Analytical experiment XRD   No.7 05253011 Fumiya Kashiwai / 柏井史哉}
%%ここまでタイトル

\newpage
\section{Purpose}
酸化チタン(IV)(\ce{TiO2}​)の多形であるルチル(Rutile)相とアナターゼ(Anatase)相の混合試料について、粉末X線回折(XRD)測定を行う。得られた回折パターンのピーク強度を解析し、用いたRutile試料中に微量に不純物として含まれる、Anataseの割合を定量分析する。

\section{Experimental}
\begin{enumerate}
    \item RutileとAnataseを、表\ref{sosei}の割合で混合した試料を、ガラス板に薄く広げた。
    \item 混合試料については、メノウ乳鉢上でアセトンを用いて均一になるまで粉砕した。
    \item $2\theta = 20-60°$, 0.01°ステップ、 10°/minでXRD測定を行った。
    \item Rutileの(101)、Anataseの(110)面に由来するピークについて、バックグラウンド除去後のピーク面積を算出し、Rutile試料に含まれるAnatase割合を算出した。
\end{enumerate}

\begin{table}[htbp]
  \centering
  \caption{各試料の組成: $\alpha$はRutile試料の割合を示す。}
  \label{sosei}
  \begin{tabular}{@{}lSSS@{}}
    \toprule
    ID & {Rutile (\si{g})} & {Anatase (\si{g})} & {$\alpha$} \\
    \midrule
    A & {--}   & {--}   & 1 \\
    B & {--}   & {--}   & 0 \\
    C & 0.1976 & 0.8059 & 0.197 \\
    D & 0.4    & 0.5986 & 0.401 \\
    E & 0.6008 & 0.4014 & 0.599 \\
    F & 0.7978 & 0.2    & 0.800 \\
    \bottomrule
  \end{tabular}
\end{table}


\section{Results}
図\ref{XRD}に示した。
\begin{figure}[htbp]
\begin{center}
\includegraphics[width = 15 cm]
{data_plot.png}
\caption{XRD測定結果}
\label{XRD}
\end{center}
\end{figure}

\newpage
\section{Discussion}

\subsection{課題1}
RutileとAnataseについて、結晶型や空間群を表\ref{crystal}に示した。ともにTetragonalであり、空間群が異なる。

\begin{table}[htbp]
  \centering
  \caption{結晶型、空間群、$Z$は単位結晶ない}
  \label{crystal}
  \begin{tabular}{@{}lllll@{}}
    \toprule
    Phase & \begin{tabular}[t]{@{}l@{}}Crystal System / \\ Space Group\end{tabular} & Lattice Parameters & $Z$ & Atomic Coordinates \\
    \midrule
    \multirow{3}{*}{Rutile} & Tetragonal & $a = \qty{4.594}{\angstrom}$ & \multirow{3}{*}{2} & Ti ($2a$): $(0, 0, 0)$ \\
     & $P4_2/mnm$ & $c = \qty{2.959}{\angstrom}$ & & O ($4f$): $(u, u, 0)$ \\
     & (No. 136) & & & $u\sim 0.305$\\
    \midrule
    \multirow{3}{*}{Anatase} & Tetragonal & $a = \qty{3.785}{\angstrom}$ & \multirow{3}{*}{4} & Ti ($4a$): $(0, 0, 0)$ \\
     & $I4_1/amd$ & $c = \qty{9.514}{\angstrom}$ & & O ($8e$): $(0, 0, u)$ \\
     & (No. 141) & & & $u\sim 0.208$\\
    \bottomrule
  \end{tabular}
\end{table}

\newpage
\subsection{課題2}
結晶の密度 $\rho$ ($\text{g/cm}^3$) は、単位格子体積 $V$ ($\text{\AA}^3$)、単位格子中の化学単位数 $Z$、モル質量 $M$ ($\text{g/mol}$)、およびアボガドロ定数 $N_A$ ($\text{mol}^{-1}$) を用いて以下の式で求められる。

\begin{equation}
\rho = \frac{Z \times M}{N_A \times V \times 10^{-24}}
\end{equation}

ここで、$\text{TiO}_2$ のモル質量 $M$ は、
\[
M = 79.865 \, \text{g/mol}
\]
また、アボガドロ定数 $N_A \approx 6.022 \times 10^{23} \, \text{mol}^{-1}$ 

\paragraph{1. Rutile (ルチル)}
(i)で得られた結晶学的情報より、格子定数 $a = 4.594 \, \text{\AA}$、$c = 2.959 \, \text{\AA}$、 $Z = 2$ とする。
単位格子体積 $V_R$ は次のように計算される。
\[
V_R = a^2 c = (4.594)^2 \times 2.959 \approx 62.45 \, \text{\AA}^3
\]
したがって、密度 $\rho_R$ は、
\begin{align}
\rho_R &= \frac{2 \times 79.865}{6.022 \times 10^{23} \times 62.45 \times 10^{-24}} \notag\\
&\approx 4.25 \, \text{g/cm}^3
\end{align}
となる。

\paragraph{2. Anatase (アナターゼ)}
同様に、格子定数 $a = 3.785 \, \text{\AA}$、$c = 9.514 \, \text{\AA}$、$Z = 4$ とする。
単位格子体積 $V_A$ は次のように計算される。
\[
V_A = a^2 c = (3.785)^2 \times 9.514 \approx 136.31 \, \text{\AA}^3
\]
したがって、密度 $\rho_A$ は、
\begin{align}
\rho_A &= \frac{4 \times 79.865}{6.022 \times 10^{23} \times 136.31 \times 10^{-24}} \notag\\
&\approx 3.89 \, \text{g/cm}^3
\end{align}
となる。

\subsection{課題3}
Rutile 試薬中に含まれる Anatase の質量分率(不純物濃度)を $x$ とし、混合試薬中の Rutile 試薬の質量分率(添加率)を $\alpha$ とする。なお、Anatase相にはRutileが含まれないものとする。

Rutile 相の質量分率 ($w_{\text{rutile}}$):
\begin{equation}
w_{\text{rutile}} = \alpha (1 - x)
\end{equation}
となる。

Anatase 相の質量分率 ($w_{\text{anatase}}$):
\begin{align}
w_{\text{anatase}} &= (1 - \alpha) + \alpha x 
= 1 - \alpha + \alpha x 
= 1 - \alpha (1 - x)
\end{align}
となる。

以上より、混合物中の各相の質量分率は $\alpha$ と $x$ を用いて次のように表される。
\begin{equation}
\begin{cases}
w_{\text{rutile}} &= \alpha (1 - x) \\
w_{\text{anatase}} &= 1 - \alpha (1 - x)
\end{cases}
\end{equation}

\subsection{課題4}
図2のように、それぞれの結晶に対する面指数が帰属される。特に吸収が大きい、Anataseの(101)、Rutileの(110)に対応する反射を用いることで、精度良く解析ができると考えられる。さらに、他のピークと重なっておらず、面積が正確に求められると期待される。そのため、これらのピークを用いる。

\begin{figure}[htbp]
\label{xrd_attribute}
 \begin{minipage}[b]{0.5\linewidth}
  \centering
  \includegraphics[keepaspectratio, scale=0.3]{Anatase_xrd.png}
 \end{minipage}
 \begin{minipage}[b]{0.5\linewidth}
  \centering
  \includegraphics[keepaspectratio, scale=0.3]{Rutile_xrd.png}
 \end{minipage}
 \caption{左:Anatase、右: Ruliteの吸収ピークと面指数との対応}
\end{figure}

\subsection{課題5}
ピーク強度を、台形近似により計算した。
なお、理由は不明だがAのみ4サンプルのデータがあるため、すべてのデータを用いた。
\begin{table}[htbp]
  \centering
  \caption{各試料のAnatase,Rutileの吸収ピークの面積}
  \label{tab:intensity_stats}
  \begin{tabular}{@{}lccc@{}}
    \toprule
    Condition & Sample Size ($N$) & \begin{tabular}[c]{@{}c@{}}$I_{\text{A}}$(Mean $\pm$ SD)\end{tabular} & \begin{tabular}[c]{@{}c@{}}$I_{\text{R}}$ (Mean $\pm$ SD)\end{tabular} \\
    \midrule
    A & 4 & $418.0 \pm 57.7$     & $21118.7 \pm 763.5$ \\
    B & 3 & $24409.7 \pm 3920.8$ & $88.2 \pm 18.4$ \\
    C & 3 & $22125.9 \pm 1171.8$ & $4044.8 \pm 124.3$ \\
    D & 3 & $18104.5 \pm 618.3$  & $8653.0 \pm 63.4$ \\
    E & 3 & $11370.0 \pm 928.0$  & $12278.7 \pm 1087.8$ \\
    F & 3 & $5534.0 \pm 321.9$   & $14834.0 \pm 666.1$ \\
    \bottomrule
  \end{tabular}
\end{table}

\subsection{課題6}
課題3より、次の理論式が成立する。
\begin{equation}
    \frac{I_{\text{R}}}{I_{\text{A}}} = K\frac{w_{\text{R}}}{w_{\text{A}}} = K\frac{1-\alpha(1-x)}{\alpha(1-x)} = \frac{K}{1-x}\frac{1}{\alpha} -K
\end{equation}
ただし、$K$は定数。
これより、
$\frac{I_{\text{R}}}{I_{\text{A}}}$に対して$\frac{1}{\alpha}$をプロットした傾きおよび切片によって、不純物の割合$x$を決定する。図\ref{prb6}に示したプロットにより、傾き1.3363、切片-1.2961である($R^2 = 0.999$)。よって
\begin{equation}
   x = 1 + (\text{切片}/\text{傾き}) = 0.0300
\end{equation}
より、不純物濃度は3\% 程度と計算された。


\begin{figure}[htbp]
\begin{center}
\includegraphics[width = 15 cm]
{problem6.png}
\caption{不純物濃度決定のためのプロット}
\label{prb6}
\end{center}
\end{figure}

\begin{thebibliography}{99}
\bibitem{Rutile}
Materials Project. (2020). Materials Data for TiO2 (mp-2657) [データセット]. \url{https://doi.org/10.17188/1184648}
\bibitem{Anatase}
Materials Project. (2020). Materials Data for TiO2 (mp-390) [データセット]. \url{https://doi.org/10.17188/1207597}
\end{thebibliography}


\end{document}
\documentclass{ltjsarticle}
%%%package読み込み
\usepackage{amsmath}
\usepackage{amssymb}
\usepackage{amsfonts}
\usepackage{mathtools}
\usepackage{bm}
% \usepackage{tikz} % ★消去: 代わりに graphicx 追加
% \usetikzlibrary{cd}
\usepackage{url}
\usepackage{graphicx} % ★追加: 図を挿入するため
\usepackage{float} % ★追加: 図の位置を制御するため
\usepackage{caption} % ★追加: 図のキャプションを柔軟に扱うため
%\usepackage{xcolor}
\usepackage{ascmac}
\usepackage{tcolorbox}
%\usepackage[dvipdfmx, setpagesize=false, bookmarks=true, bookmarksdepth=tocdepth, bookmarksnumbered=true, colorlinks=true, linkcolor=red]
\usepackage{hyperref}
\usepackage{booktabs}
\usepackage{siunitx}
\usepackage{multirow}
\usepackage[version=4]{mhchem}
\usepackage{braket} % 追加した
\usepackage{booktabs}
\usepackage{bookmark}
%\usepackage[textwidth=45zw,lines=44]{geometry}
%\usepackage{pxjahyper}
%%%黒板太字
\newcommand{\N}{\mathbb{N}}
\newcommand{\Z}{\mathbb{Z}}
\newcommand{\Q}{\mathbb{Q}}
\newcommand{\R}{\mathbb{R}}
\newcommand{\C}{\mathbb{C}}
\newcommand{\F}{\mathbb{F}}
%%%約物
\newcommand{\abs}[1]{\left|#1\right|}
\newcommand{\lr}[1]{\left(#1\right)}
\newcommand{\st}{\; \mathrm{s.t.}\; }
\newcommand{\Ae}{\textrm{-a.e.}} 
%%%繰り返し
\newcommand{\pluss}[3]{#1_{#2}+\cdots+#1_{#3}}
\newcommand{\minuss}[3]{#1_{#2}-\cdots-#1_{#3}}
\newcommand{\timess}[3]{#1_{#2}\times\cdots\times #1_{#3}}
\newcommand{\leqs}[3]{#1_{#2}\leq\cdots\leq #1_{#3}}
\newcommand{\geqs}[3]{#1_{#2}\geq\cdots\geq #1_{#3}}
\newcommand{\opluss}[3]{#1_{#2}\oplus\cdots\oplus #1_{#3}}
\newcommand{\otimess}[3]{#1_{#2}\otimes\cdots\otimes #1_{#3}}
\newcommand{\commas}[3]{#1_{#2},\ldots,#1_{#3}}
%%%微分
\newcommand{\dx}[1]{\mathrm{d}#1}
\newcommand{\ddx}[1]{\frac{\mathrm{d}}{\mathrm{d}#1}}
\newcommand{\dydx}[2]{\frac{\mathrm{d}#1}{\mathrm{d}#2}}
\newcommand{\dydxn}[3]{\frac{\mathrm{d}^{#3}#1}{\mathrm{d}#2^{#3}}}
\newcommand{\del}[2]{\frac{\partial#1}{\partial#2}}
\newcommand{\dell}[2]{\frac{\partial^2#1}{{\partial#2}^2}}
\newcommand{\deln}[3]{\frac{\partial^{#3}#1}{{\partial#2}^{#3}}}
%%%
%%%演算子
%log type
\let\Re\relax
\DeclareMathOperator{\Re}{Re}
\let\Im\relax
\DeclareMathOperator{\Im}{Im}
\DeclareMathOperator{\sgn}{sgn}
\DeclareMathOperator{\sign}{sign}
\DeclareMathOperator{\Supp}{Supp}
\DeclareMathOperator{\tr}{tr}
\DeclareMathOperator{\Tr}{Tr}
\DeclareMathOperator{\Det}{Det}
\DeclareMathOperator{\Log}{Log}
\DeclareMathOperator{\rank}{rank}
\DeclareMathOperator{\diag}{diag}
\DeclareMathOperator{\corank}{corank}
\DeclareMathOperator{\Res}{Res}
\DeclareMathOperator{\Ker}{Ker}
\DeclareMathOperator{\coker}{coker}
\DeclareMathOperator{\Coker}{Coker}
\DeclareMathOperator{\Var}{Var}
\DeclareMathOperator{\Cov}{Cov}
\DeclareMathOperator{\sech}{sech}
\DeclareMathOperator{\csch}{csch}
\DeclareMathOperator{\arcsec}{arcsec}
\DeclareMathOperator{\arccot}{arccot}
\DeclareMathOperator{\arccsc}{arccsc}
\DeclareMathOperator{\arccosh}{arccosh}
\DeclareMathOperator{\arcsinh}{arcsinh}
\DeclareMathOperator{\arctanh}{arctanh}
\DeclareMathOperator{\arcsech}{arcsech}
\DeclareMathOperator{\arccsch}{arccsch}
\DeclareMathOperator{\arccoth}{arccoth}
\DeclareMathOperator{\grad}{grad}
\let\div\relax
\DeclareMathOperator{\div}{div}
\DeclareMathOperator{\rot}{rot}
%\DeclareMathOperator{\GL}{GL} % ★消去 : ここから↓
%\DeclareMathOperator{\SL}{SL}
%\DeclareMathOperator{\Sym}{Sym}
%\DeclareMathOperator{\Aut}{Aut}
%\DeclareMathOperator{\Inn}{Inn}
%\DeclareMathOperator{\Out}{Out}
%\DeclareMathOperator{\id}{id}
%\DeclareMathOperator{\pr}{pr}
%\DeclareMathOperator{\supp}{supp}
%\DeclareMathOperator{\diam}{diam}
%\DeclareMathOperator{\End}{End}
%\DeclareMathOperator{\Cl}{Cl}
%\DeclareMathOperator{\Hom}{Hom} % ★消去 : ここまで↑
%limit type
\DeclareMathOperator*{\argmin}{arg~min}
\DeclareMathOperator*{\argmax}{arg~max}
%%%
%%%定理
\usepackage{amsthm}
\theoremstyle{definition}
\newtheorem{lem}{補題}
\newtheorem*{lem*}{補題}
\newtheorem{prf}{証明}
\newtheorem*{prf*}{証明}
\newtheorem*{ex*}{Example}
\newtheorem*{rem*}{Remark}
\newenvironment{prb}[1]%
{\begin{itembox}[l]{\textbf{問題 #1}}}%
{\end{itembox}}
\newenvironment{sol}[2]%
{\setcounter{lem}{0}
\setcounter{prf}{0}
\par\noindent\textbf{解答 #1} (#2)\par}%
{\par\normalfont}

\renewcommand{\refname}{Reference}


%%%%%%%%%%%%%%%%%%%%%
\numberwithin{equation}{section}
%%%%%%%%%%%%%%%%%%%%%%

\newcounter{boxeddefcounter}
\newenvironment{problem}
{\refstepcounter{boxeddefcounter}\begin{itembox}[l]{問\theboxeddefcounter}}
{\end{itembox}}

%\usepackage[hang,small,bf]{caption}
%\usepackage[subrefformat=parens]{subcaption}
\captionsetup{compatibility=false}


\newcommand{\D}{^\circ\text{C}}
\newcommand{\ka}{\textasciitilde}


\pagestyle{myheadings}
\title{Inorganic and Analyutical experiment unit.2}
\date{\today}
\author{Author: No.7 05253011 Fumiya Kashiwai / 柏井史哉}
\begin{document}
\maketitle
\markboth{Inorganic and Analytical experiment Unit.2   No.7 05253011 Fumiya Kashiwai / 柏井史哉} {Inorganic and Analytical experiment Unit.2   No.7 05253011 Fumiya Kashiwai / 柏井史哉}
%%ここまでタイトル

\newpage
\section{Purpose}
同位体希釈法により、未知試料中のリン酸濃度を決定する。同位体希釈法では、既知濃度のリン酸溶液との放射線量を比較することによって、資料を100\% 回収せずとも未知試料中のリン酸濃度を決定することができる。

\section{Experimental}
\begin{enumerate}
    \item ブランチェットの風袋の質量を計測した。
    \item 2本の遠沈管に濃度既知、未知のリン酸ナトリウム水溶液を1.0 mL分取し、蒸留水を3.0 mL加えて希釈した。
    \item 放射性\ce{^32 P}標識のリン酸水溶液0.1 mLを加え、よく混合した。
    \item 沈澱が生じなくなるまで\ce{CaCl2}水溶液を徐々に加えた。
    \item 生じた沈澱を遠心分離 (2000 rpm, 2 min)して、上澄を除去した。
    \item 沈殿に2.0 mLの蒸留水を加え、voltexして施錠した。再び遠心分離 (2000 rpm, 2 min)して、上澄を除去した。
    \item 沈澱の大部分をブランチェットに移し、ホットプレートと赤外ランプで乾燥させた。
    \item ブランチェットの質量を計測し、沈澱の質量を求めた。
    \item GM管の測定電圧を、unit.1で決定した1100 Vとして、両方の試料の放射線の測定を5 min行った。
    \item backgroundの放射線の測定を5 min行った。
\end{enumerate}

\section{Results}
測定結果は次のようになった。


\begin{table}[htbp]
  \centering
  \caption{Measurement Data: Weight, Count Rates, and Concentration}
  \label{tab:measurement_data_4}
  \begin{tabular}{@{}lccccccc@{}}
    \toprule
    Sample & \begin{tabular}[c]{@{}c@{}}Weight\\ (\si{g})\end{tabular} & Count & \begin{tabular}[c]{@{}c@{}}Time\\ (\si{min})\end{tabular} & cpm & \begin{tabular}[c]{@{}c@{}}Net cpm\\ (\si{cpm})\end{tabular} & \begin{tabular}[c]{@{}c@{}}S\\ (\si{cpm.g^{-1}})\end{tabular} & \begin{tabular}[c]{@{}c@{}}Concentration\\ (\si{mM})\end{tabular} \\
    \midrule
    Unknown    & 0.0138 & 63231 & 5 & 12646.2 & 12593.8 & 912594.20 & 89.35 \\
    Known      & 0.0209 & 71269 & 5 & 14253.8 & 14201.4 & 679492.82 & 120   \\
    Background & --     & 262   & 5 & 52.4    & --      & --        & --    \\
    \bottomrule
  \end{tabular}
\end{table}

\section{Discussion}
\subsection{課題1}
既知試料の濃度120 mMを用いる。また、未知資料のリン酸濃度を$x$ mMとする。
両方の試料に、同量のトレーサー\ce{^32 P}を加えているので、総放射能は等しい。このため、
\begin{equation}
  x S_{\text{unknown}} = 120 \text{mM} \times S_{\text{known}}
\end{equation}
が成立する。測定結果を用いると、$x = 89.3 $ mMと決定される。


\subsection{課題2}
別法としては、モリブデンブルー法が挙げられる。酸性条件下でモリブデン酸アンモニウムを加えると、黄色のりんモリブデン酸が生じ、これを還元すると青色のモリブデンブルーの錯体が生じる。この吸光度を測定することで、濃度を決定することができる。

\paragraph{長所}
放射線を扱わないために操作や廃液の処理が楽である。また、吸光度分析は簡便かつ高感度であり、低濃度の溶液であっても十分な精度で決定できると期待される。

\paragraph{短所}
妨害物質が存在している場合、測定値に誤差を与えることになってしまう。また、定量的な分析を行う必要があるため、試料の希釈時などに、容量などを正確に計る必要がある。また、高濃度過ぎるばあいにはそのままでは吸光度分析ができず、適切な割合に希釈する必要がある。

\paragraph{同位体希釈法}
放射性同位体を扱うために手続きや実験操作が煩雑であるが、100\%の回収をせずとも相対比から、定量分析が可能である点が長所である。


\begin{thebibliography}{99}
\bibitem{nmr_solvent}
中村 道徳(1962). リン酸の比色定量法II. 化学と生物, 3(2), 91-98. https://doi.org/10.1271/kagakutoseibutsu1962.3.91
\end{thebibliography}


\end{document}
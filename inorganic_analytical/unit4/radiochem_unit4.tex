\documentclass{ltjsarticle}
%%%package読み込み
\usepackage{amsmath}
\usepackage{amssymb}
\usepackage{amsfonts}
\usepackage{mathtools}
\usepackage{bm}
% \usepackage{tikz} % ★消去: 代わりに graphicx 追加
% \usetikzlibrary{cd}
\usepackage{url}
\usepackage{graphicx} % ★追加: 図を挿入するため
\usepackage{float} % ★追加: 図の位置を制御するため
\usepackage{caption} % ★追加: 図のキャプションを柔軟に扱うため
%\usepackage{xcolor}
\usepackage{ascmac}
\usepackage{tcolorbox}
%\usepackage[dvipdfmx, setpagesize=false, bookmarks=true, bookmarksdepth=tocdepth, bookmarksnumbered=true, colorlinks=true, linkcolor=red]
\usepackage{hyperref}
\usepackage{booktabs}
\usepackage{siunitx}
\usepackage{multirow}
\usepackage[version=4]{mhchem}
\usepackage{braket} % 追加した
\usepackage{booktabs}
\usepackage{bookmark}
%\usepackage[textwidth=45zw,lines=44]{geometry}
%\usepackage{pxjahyper}
%%%黒板太字
\newcommand{\N}{\mathbb{N}}
\newcommand{\Z}{\mathbb{Z}}
\newcommand{\Q}{\mathbb{Q}}
\newcommand{\R}{\mathbb{R}}
\newcommand{\C}{\mathbb{C}}
\newcommand{\F}{\mathbb{F}}
%%%約物
\newcommand{\abs}[1]{\left|#1\right|}
\newcommand{\lr}[1]{\left(#1\right)}
\newcommand{\st}{\; \mathrm{s.t.}\; }
\newcommand{\Ae}{\textrm{-a.e.}} 
%%%繰り返し
\newcommand{\pluss}[3]{#1_{#2}+\cdots+#1_{#3}}
\newcommand{\minuss}[3]{#1_{#2}-\cdots-#1_{#3}}
\newcommand{\timess}[3]{#1_{#2}\times\cdots\times #1_{#3}}
\newcommand{\leqs}[3]{#1_{#2}\leq\cdots\leq #1_{#3}}
\newcommand{\geqs}[3]{#1_{#2}\geq\cdots\geq #1_{#3}}
\newcommand{\opluss}[3]{#1_{#2}\oplus\cdots\oplus #1_{#3}}
\newcommand{\otimess}[3]{#1_{#2}\otimes\cdots\otimes #1_{#3}}
\newcommand{\commas}[3]{#1_{#2},\ldots,#1_{#3}}
%%%微分
\newcommand{\dx}[1]{\mathrm{d}#1}
\newcommand{\ddx}[1]{\frac{\mathrm{d}}{\mathrm{d}#1}}
\newcommand{\dydx}[2]{\frac{\mathrm{d}#1}{\mathrm{d}#2}}
\newcommand{\dydxn}[3]{\frac{\mathrm{d}^{#3}#1}{\mathrm{d}#2^{#3}}}
\newcommand{\del}[2]{\frac{\partial#1}{\partial#2}}
\newcommand{\dell}[2]{\frac{\partial^2#1}{{\partial#2}^2}}
\newcommand{\deln}[3]{\frac{\partial^{#3}#1}{{\partial#2}^{#3}}}
%%%
%%%演算子
%log type
\let\Re\relax
\DeclareMathOperator{\Re}{Re}
\let\Im\relax
\DeclareMathOperator{\Im}{Im}
\DeclareMathOperator{\sgn}{sgn}
\DeclareMathOperator{\sign}{sign}
\DeclareMathOperator{\Supp}{Supp}
\DeclareMathOperator{\tr}{tr}
\DeclareMathOperator{\Tr}{Tr}
\DeclareMathOperator{\Det}{Det}
\DeclareMathOperator{\Log}{Log}
\DeclareMathOperator{\rank}{rank}
\DeclareMathOperator{\diag}{diag}
\DeclareMathOperator{\corank}{corank}
\DeclareMathOperator{\Res}{Res}
\DeclareMathOperator{\Ker}{Ker}
\DeclareMathOperator{\coker}{coker}
\DeclareMathOperator{\Coker}{Coker}
\DeclareMathOperator{\Var}{Var}
\DeclareMathOperator{\Cov}{Cov}
\DeclareMathOperator{\sech}{sech}
\DeclareMathOperator{\csch}{csch}
\DeclareMathOperator{\arcsec}{arcsec}
\DeclareMathOperator{\arccot}{arccot}
\DeclareMathOperator{\arccsc}{arccsc}
\DeclareMathOperator{\arccosh}{arccosh}
\DeclareMathOperator{\arcsinh}{arcsinh}
\DeclareMathOperator{\arctanh}{arctanh}
\DeclareMathOperator{\arcsech}{arcsech}
\DeclareMathOperator{\arccsch}{arccsch}
\DeclareMathOperator{\arccoth}{arccoth}
\DeclareMathOperator{\grad}{grad}
\let\div\relax
\DeclareMathOperator{\div}{div}
\DeclareMathOperator{\rot}{rot}
%\DeclareMathOperator{\GL}{GL} % ★消去 : ここから↓
%\DeclareMathOperator{\SL}{SL}
%\DeclareMathOperator{\Sym}{Sym}
%\DeclareMathOperator{\Aut}{Aut}
%\DeclareMathOperator{\Inn}{Inn}
%\DeclareMathOperator{\Out}{Out}
%\DeclareMathOperator{\id}{id}
%\DeclareMathOperator{\pr}{pr}
%\DeclareMathOperator{\supp}{supp}
%\DeclareMathOperator{\diam}{diam}
%\DeclareMathOperator{\End}{End}
%\DeclareMathOperator{\Cl}{Cl}
%\DeclareMathOperator{\Hom}{Hom} % ★消去 : ここまで↑
%limit type
\DeclareMathOperator*{\argmin}{arg~min}
\DeclareMathOperator*{\argmax}{arg~max}
%%%
%%%定理
\usepackage{amsthm}
\theoremstyle{definition}
\newtheorem{lem}{補題}
\newtheorem*{lem*}{補題}
\newtheorem{prf}{証明}
\newtheorem*{prf*}{証明}
\newtheorem*{ex*}{Example}
\newtheorem*{rem*}{Remark}
\newenvironment{prb}[1]%
{\begin{itembox}[l]{\textbf{問題 #1}}}%
{\end{itembox}}
\newenvironment{sol}[2]%
{\setcounter{lem}{0}
\setcounter{prf}{0}
\par\noindent\textbf{解答 #1} (#2)\par}%
{\par\normalfont}

\renewcommand{\refname}{Reference}


%%%%%%%%%%%%%%%%%%%%%
\numberwithin{equation}{section}
%%%%%%%%%%%%%%%%%%%%%%

\newcounter{boxeddefcounter}
\newenvironment{problem}
{\refstepcounter{boxeddefcounter}\begin{itembox}[l]{問\theboxeddefcounter}}
{\end{itembox}}

%\usepackage[hang,small,bf]{caption}
%\usepackage[subrefformat=parens]{subcaption}
\captionsetup{compatibility=false}


\newcommand{\D}{^\circ\text{C}}
\newcommand{\ka}{\textasciitilde}


\pagestyle{myheadings}
\title{Inorganic and Analyutical experiment unit.4}
\date{\today}
\author{Author: No.7 05253011 Fumiya Kashiwai / 柏井史哉}
\begin{document}
\maketitle
\markboth{Inorganic and Analytical experiment Unit.4   No.7 05253011 Fumiya Kashiwai / 柏井史哉} {Inorganic and Analytical experiment Unit.4   No.7 05253011 Fumiya Kashiwai / 柏井史哉}
%%ここまでタイトル

\newpage
\section{Purpose}
\ce{Al}板による$\beta$線の遮蔽を測定することにより、\ce{Al}板の厚さと遮蔽の関係を調べる。

\section{Experimental}
\begin{enumerate}
    \item GM管で、測定電圧はUnit.1の実験で決定した1100 Vとして、バックグラウンドを測定した。
    \item \ce{^{14}C}線源をGMスタンドの上から4段目にセットし、放射線を測定した。
    \item 様々な厚みの\ce{Al}板を線源の上におき、再び放射線を測定した。
    \item \ce{226 Ra}をGMスタンドの上から6段目にセットし、同様の測定を行った。
\end{enumerate}

\section{Results}
計測結果は、次の表\ref{table_Al}および表\ref{table_Ra}のようになった。

\begin{table}[htbp]
  \centering
  \caption{\ce{Al}による\ce{^14 C}の放射線の遮蔽}
  \label{table_Al}
  \begin{tabular}{@{}lcccccc@{}}
    \toprule
    No. & \begin{tabular}[c]{@{}c@{}}Thickness\\ (\si{mg.cm^{-2}})\end{tabular} & \begin{tabular}[c]{@{}c@{}}Time\\ (\si{min})\end{tabular} & count & cpm & \begin{tabular}[c]{@{}c@{}}Net count\\ (\si{cpm})\end{tabular} & \begin{tabular}[c]{@{}c@{}}s.d.\\ (\si{cpm})\end{tabular} \\
    \midrule
    BG & -- & 5 & 154 & 30.8 & -- & -- \\
    1  & 0    & 1 & 1488 & 1488.0 & 1457.2 & 38.65 \\
    2  & 4.32 & 1 & 698  & 698.0  & 667.2  & 26.54 \\
    3  & 20.2 & 3 & 86   & 28.7   & --     & 3.96 \\
    4  & 3.80  & 1 & 705  & 705.0  & 674.2  & 26.67 \\
    5  & 5.92 & 2 & 827  & 413.5  & 382.7  & 14.59 \\
    6  & 2.0    & 1 & 913  & 913.0  & 882.2  & 30.32 \\
    7  & 6.9  & 3 & 837  & 279.0  & 248.2  & 9.96 \\
    8  & 8.17 & 5 & 1181 & 236.2  & 205.4  & 7.31 \\
    9  & 13.3 & 5 & 287  & 57.4   & 26.6   & 4.20 \\
    \bottomrule
  \end{tabular}
\end{table}

\begin{table}[htbp]
  \centering
  \caption{\ce{Al}による\ce{^226 Ra}の放射線の遮蔽}
  \label{Table_Ra}
  \begin{tabular}{@{}lcccccc@{}}
    \toprule
    No. & \begin{tabular}[c]{@{}c@{}}Thickness\\ (\si{mg.cm^{-2}})\end{tabular} & \begin{tabular}[c]{@{}c@{}}Time\\ (\si{min})\end{tabular} & count & cpm & \begin{tabular}[c]{@{}c@{}}Net count\\ (\si{cpm})\end{tabular} & \begin{tabular}[c]{@{}c@{}}s.d.\\ (\si{cpm})\end{tabular} \\
    \midrule
    BG & -- & 5 & 154 & 30.8 & -- & -- \\
    1  & 0      & 1 & 23444 & 23444.0 & 23413.2 & 153.13 \\
    2  & 1111   & 1 & 1187  & 1187.0  & 1156.2  & 34.54 \\
    3  & 80.44  & 1 & 14103 & 14103.0 & 14072.2 & 118.78 \\
    4  & 135.77 & 1 & 10744 & 10744.0 & 10713.2 & 103.68 \\
    5  & 219.29 & 1 & 7467  & 7467.0  & 7436.2  & 86.45 \\
    6  & 401.24 & 1 & 3912  & 3912.0  & 3881.2  & 62.60 \\
    7  & 688.31 & 1 & 1744  & 1744.0  & 1713.2  & 41.83 \\
    8  & 815.31 & 1 & 1423  & 1423.0  & 1392.2  & 37.80 \\
    9  & 1926   & 1 & 918   & 918.0   & 887.2   & 30.40 \\
    \bottomrule
  \end{tabular}
\end{table}

これをグラフにすると下の図のようになる。
\begin{figure}[htbp]
\begin{center}
\includegraphics[width = 15 cm]
{Al.png}
\caption{\ce{Al}による\ce{^14 C}の放射線の遮蔽}
\label{grah_Al}
\end{center}
\end{figure}

\begin{figure}[htbp]
\begin{center}
\includegraphics[width = 15 cm]
{Ra.png}
\caption{\ce{Al}による\ce{^226 Ra}の放射線の遮蔽}
\label{grah_Ra}
\end{center}
\end{figure}

\section{Discussion}
\subsection{課題1}
グラフはresultsに示した。

\subsection{課題2}
\ce{^14 C}では、GM管で観測されるのはβ線のみである。そのため、\ce{Al}板の厚さに対して、放射線の強度は指数減衰する。

対して、\ce{^226 Ra}はγ線も発する。γ線は\ce{Al}板では遮蔽することができないため、厚さにかからわず観測器に到達する。そのため、非常に厚い条件であっても、一定の放射能が観測されると期待される。実際、グラフに示した通り、counts数は1000 cpm程度に収束しており、これがγ線によるものと考えられる。

\if0
\begin{thebibliography}{99}
\bibitem{nmr_solvent}
Hugo E. Gottlieb, Vadim Kotlyar, and
Abraham Nudelman, 
NMR Chemical Shifts of Common
Laboratory Solvents as Trace Impurities
J. Org. Chem. 1997, 62, 7512-7515
\bibitem{Wurtz}
\url{https://www.chem-station.com/odos/2009/07/wurtz-wurtz-reaction.html}
\bibitem{biphenyl}
\url{https://www.rsc.org/suppdata/cc/c3/c3cc45132a/c3cc45132a.pdf}
\bibitem{o}
\url{https://www.chemspider.com/Chemical-Structure.15646.html}
\end{thebibliography}

\fi
\end{document}
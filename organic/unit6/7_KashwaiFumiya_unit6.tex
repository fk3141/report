\documentclass{ltjsarticle}
%%%package読み込み
\usepackage{amsmath}
\usepackage{amssymb}
\usepackage{amsfonts}
\usepackage{mathtools}
\usepackage{bm}
% \usepackage{tikz} % ★消去: 代わりに graphicx 追加
% \usetikzlibrary{cd}
\usepackage{url}
\usepackage{graphicx} % ★追加: 図を挿入するため
\usepackage{float} % ★追加: 図の位置を制御するため
\usepackage{caption} % ★追加: 図のキャプションを柔軟に扱うため
%\usepackage{xcolor}
\usepackage{ascmac}
\usepackage{tcolorbox}
%\usepackage[dvipdfmx, setpagesize=false, bookmarks=true, bookmarksdepth=tocdepth, bookmarksnumbered=true, colorlinks=true, linkcolor=red]
\usepackage{hyperref}
\usepackage[version=4]{mhchem}
\usepackage{braket} % 追加した
\usepackage{booktabs}
\usepackage{bookmark}
%\usepackage[textwidth=45zw,lines=44]{geometry}
%\usepackage{pxjahyper}
%%%黒板太字
\newcommand{\N}{\mathbb{N}}
\newcommand{\Z}{\mathbb{Z}}
\newcommand{\Q}{\mathbb{Q}}
\newcommand{\R}{\mathbb{R}}
\newcommand{\C}{\mathbb{C}}
\newcommand{\F}{\mathbb{F}}
%%%約物
\newcommand{\abs}[1]{\left|#1\right|}
\newcommand{\lr}[1]{\left(#1\right)}
\newcommand{\st}{\; \mathrm{s.t.}\; }
\newcommand{\Ae}{\textrm{-a.e.}} 
%%%繰り返し
\newcommand{\pluss}[3]{#1_{#2}+\cdots+#1_{#3}}
\newcommand{\minuss}[3]{#1_{#2}-\cdots-#1_{#3}}
\newcommand{\timess}[3]{#1_{#2}\times\cdots\times #1_{#3}}
\newcommand{\leqs}[3]{#1_{#2}\leq\cdots\leq #1_{#3}}
\newcommand{\geqs}[3]{#1_{#2}\geq\cdots\geq #1_{#3}}
\newcommand{\opluss}[3]{#1_{#2}\oplus\cdots\oplus #1_{#3}}
\newcommand{\otimess}[3]{#1_{#2}\otimes\cdots\otimes #1_{#3}}
\newcommand{\commas}[3]{#1_{#2},\ldots,#1_{#3}}
%%%微分
\newcommand{\dx}[1]{\mathrm{d}#1}
\newcommand{\ddx}[1]{\frac{\mathrm{d}}{\mathrm{d}#1}}
\newcommand{\dydx}[2]{\frac{\mathrm{d}#1}{\mathrm{d}#2}}
\newcommand{\dydxn}[3]{\frac{\mathrm{d}^{#3}#1}{\mathrm{d}#2^{#3}}}
\newcommand{\del}[2]{\frac{\partial#1}{\partial#2}}
\newcommand{\dell}[2]{\frac{\partial^2#1}{{\partial#2}^2}}
\newcommand{\deln}[3]{\frac{\partial^{#3}#1}{{\partial#2}^{#3}}}
%%%
%%%演算子
%log type
\let\Re\relax
\DeclareMathOperator{\Re}{Re}
\let\Im\relax
\DeclareMathOperator{\Im}{Im}
\DeclareMathOperator{\sgn}{sgn}
\DeclareMathOperator{\sign}{sign}
\DeclareMathOperator{\Supp}{Supp}
\DeclareMathOperator{\tr}{tr}
\DeclareMathOperator{\Tr}{Tr}
\DeclareMathOperator{\Det}{Det}
\DeclareMathOperator{\Log}{Log}
\DeclareMathOperator{\rank}{rank}
\DeclareMathOperator{\diag}{diag}
\DeclareMathOperator{\corank}{corank}
\DeclareMathOperator{\Res}{Res}
\DeclareMathOperator{\Ker}{Ker}
\DeclareMathOperator{\coker}{coker}
\DeclareMathOperator{\Coker}{Coker}
\DeclareMathOperator{\Var}{Var}
\DeclareMathOperator{\Cov}{Cov}
\DeclareMathOperator{\sech}{sech}
\DeclareMathOperator{\csch}{csch}
\DeclareMathOperator{\arcsec}{arcsec}
\DeclareMathOperator{\arccot}{arccot}
\DeclareMathOperator{\arccsc}{arccsc}
\DeclareMathOperator{\arccosh}{arccosh}
\DeclareMathOperator{\arcsinh}{arcsinh}
\DeclareMathOperator{\arctanh}{arctanh}
\DeclareMathOperator{\arcsech}{arcsech}
\DeclareMathOperator{\arccsch}{arccsch}
\DeclareMathOperator{\arccoth}{arccoth}
\DeclareMathOperator{\grad}{grad}
\let\div\relax
\DeclareMathOperator{\div}{div}
\DeclareMathOperator{\rot}{rot}
%\DeclareMathOperator{\GL}{GL} % ★消去 : ここから↓
%\DeclareMathOperator{\SL}{SL}
%\DeclareMathOperator{\Sym}{Sym}
%\DeclareMathOperator{\Aut}{Aut}
%\DeclareMathOperator{\Inn}{Inn}
%\DeclareMathOperator{\Out}{Out}
%\DeclareMathOperator{\id}{id}
%\DeclareMathOperator{\pr}{pr}
%\DeclareMathOperator{\supp}{supp}
%\DeclareMathOperator{\diam}{diam}
%\DeclareMathOperator{\End}{End}
%\DeclareMathOperator{\Cl}{Cl}
%\DeclareMathOperator{\Hom}{Hom} % ★消去 : ここまで↑
%limit type
\DeclareMathOperator*{\argmin}{arg~min}
\DeclareMathOperator*{\argmax}{arg~max}
%%%
%%%定理
\usepackage{amsthm}
\theoremstyle{definition}
\newtheorem{lem}{補題}
\newtheorem*{lem*}{補題}
\newtheorem{prf}{証明}
\newtheorem*{prf*}{証明}
\newtheorem*{ex*}{Example}
\newtheorem*{rem*}{Remark}
\newenvironment{prb}[1]%
{\begin{itembox}[l]{\textbf{問題 #1}}}%
{\end{itembox}}
\newenvironment{sol}[2]%
{\setcounter{lem}{0}
\setcounter{prf}{0}
\par\noindent\textbf{解答 #1} (#2)\par}%
{\par\normalfont}

\renewcommand{\refname}{Reference}


%%%%%%%%%%%%%%%%%%%%%
\numberwithin{equation}{section}
%%%%%%%%%%%%%%%%%%%%%%

\newcounter{boxeddefcounter}
\newenvironment{problem}
{\refstepcounter{boxeddefcounter}\begin{itembox}[l]{問\theboxeddefcounter}}
{\end{itembox}}

%\usepackage[hang,small,bf]{caption}
%\usepackage[subrefformat=parens]{subcaption}
\captionsetup{compatibility=false}


\newcommand{\D}{^\circ\text{C}}
\newcommand{\ka}{\textasciitilde}


\pagestyle{myheadings}
\title{有機化学実験 Unit.6}
\date{10/7, 8. 9, 14, 15}
\author{実験者/報告者: No.7 05253011 Fumiya Kashiwai / 柏井史哉}
\begin{document}
\maketitle
\markboth{Organic experiment Unit.5 No.7 05253011 Fumiya Kashiwai / 柏井史哉} {Organic experiment Unit.5 No.7 05253011 Fumiya Kashiwai / 柏井史哉}
%%ここまでタイトル

\newpage
\section{3,6-dymethylphthalic anhydrideの合成}
\subsection{Purpose and Background}
図\ref{scheme_6-1}に示したスキームにより3,6-dymethylphthalic anhydrideを合成する。
\begin{figure}[htbp]
\begin{center}
\includegraphics[width = 15 cm]{scheme_6-1.png}
\caption{化合物\textbf{4}の合成スキーム}
\label{scheme_6-1}
\end{center}
\end{figure}

\subsection{Experimental}
\begin{enumerate}
\item ナスフラスコにアンバーリスト320 mg、化合物\textbf{1} 21.17 gをとり、マグネティックスターラーで撹拌しながら$130\D$のオイルバスで蒸留して留分を得た。留分は氷冷した。
\item 受けに用いていたフラスコが転倒し、産物をロスしたためあらためて操作を行った。
\item ナスフラスコにアンバーリスト220 mg、化合物\textbf{1} 21.00 gをとり、スターラーで撹拌しながら$130\D$のオイルバスで蒸留して留分を得た。留分は氷冷して得た。
蒸気温度が$75\D$で流出が開始し、蒸気温度が$82\D$まで上昇したのち、$76\D$程度まで低下し液体の流出が穏やかになったところで蒸留を停止した。
\item 得られた留分は下層 (透明) と、上層 (黄色) にほぼ同体積で分離していた。下の水層をピペットで大方取り除いたのち、無水\ce{Na2SO4}を加えて乾燥、ひだつき濾紙により濾過することにより化合物\textbf{2}の粗精製物 (9.93 g, 56\%) を得た。
\item 三角フラスコに乳鉢で粉砕した無水マレイン酸 8.09 g、\ce{Et2O} 6 mLを加え、マグネティックスターラーで激しく撹拌して懸濁させた。
\item 化合物\textbf{2}の粗精製物 9.70 gをフラスコに加え、アルミホイルで蓋をして、室温で撹拌したところ、黄色の均質な溶液となった。45 min程度撹拌を続けると白濁し、白色の沈澱が生じた。反応進行をTLC (hexane: EtOAc = 2:1) で追跡した。
\item 生じた沈澱を吸引濾過すると黄色がかった固体を得た。氷冷した\ce{Et2O}で洗浄し、白色の固体\textbf{3} (8.17 g, 48\%) を得た。この固体を風乾し、IRスペクトル(図\ref{IR_6-1-2})を測定した。また、この物質の融点の測定値は$71-72\D$ (文献値: $67-71\D$) であった。
\item 乾燥した三角フラスコに濃\ce{H2SO4} 40 mLをはかりとり、食塩を加えた氷浴により$-6\D$程度に冷却した。温度をモニターしながら、溶液の温度を$-3\D$前後に保って固体\textbf{3} 4.01 gを少しずつ加えた。1 h程度で固体を加え終わったのち、10 min撹拌して溶液が均質になった。
\item 氷浴を外して溶液を室温に戻したのち、300 mLの氷水にゆっくりと加えたところ、白濁した。
\item 吸引濾過ののち、水10 mLで洗浄し粘土状の白色固体を得た。
\item 得られた固体を$10\%$ \ce{NaOH}\textit{aq.} 30 mLに溶かし、\ce{AcOH} 5 mLを加えた。析出した固体をひだ付き濾紙を通して、取り除いた。
\item 濾液に濃塩酸4 mLを加えると白色の結晶が生じた。この結晶を吸引濾過によりろ別した。時間の都合上、結晶の状態で4 days風乾した。
\item 得られた固体をtoluene 50 mLに溶解し、常圧蒸留を行った。はじめ蒸気温度$80\D$程度の成分が流出し、蒸気温度が低下したところで一旦加熱を停止し、熱時濾過により固体を除去した。
\item さらに蒸留を行い、蒸気温度$102\D$程度の成分を得た。全量5 mL程度まで濃縮し、氷冷して析出した固体を固体を吸引濾過、風乾し、白色固体 (0.95 g, 26\%) を得た。得られた固体のIRスペクトル (図\ref{IR_6-1-3})、\ce{^1H NMR}スペクトル (図\ref{NMR_6-1-3}) を取得した。測定した融点は$145-146\D$ (文献値: $146-147\D$) であった。
\end{enumerate}

\subsection{Result and Discussion}
\subsubsection{反応機構}

図\ref{mechanism_6-1}に示した反応機構で進行すると推測される。酸触媒下での反応であり、分子間および分子内アルドール反応と目的の反応が競合すると予測される。しかしながら、アルドール反応は可逆であり、目的の反応は脱水を伴うため基本的には不可逆である。そのため最終生成物としては化合物\textbf{2}が優先的に生成すると考えられる。

その後、無水マレイン酸と化合物\textbf{2}の間でDiels-Alder (DA) 反応が生じる。求ジエン\textbf{2}は電子豊富、無水マレイン酸は電子不足であり、このDA反応は良好に進行すると期待される。実際、今回の実験では室温で45 minで完了したと考えられる。この際、endo付加体およびexo付加体が生じる可能性がある。しかしながら、後述のようにTLCは1種類の化合物のみの存在を支持しており、どちらかが選択的に生じたと考えられるが一般則からendo体がmajorであると考えられる。
その後、硫酸溶媒下で脱水反応を起こして化合物\textbf{4}を得る。この反応は(E2ではなく)E1機構で進行すると考えられ、なぜならmajorであると考えられる\textbf{3}のendo体では脱離すべき官能基がanti配座を取り得ないためである。

\textbf{3}→\textbf{4}の反応は脱水反応であり、生じた水は溶媒の硫酸との溶媒和で熱を生じる。加えて、\textbf{3}→\textbf{4}では芳香環が形成されるため、大きく安定化すると予測され、この反応は非常に発熱的であると期待される。実際、この反応は食塩を加えた氷浴により冷却しながら行ったが、化合物\textbf{3}の粉末を反応溶液に加えると溶液温度の上昇が見られた。冷却することにより、脱炭酸反応を抑制することができていると考えられる。

\subsubsection{精製過程について}
化合物\textbf{4}の粗精製物に含まれると考えられる副生成物としては、未反応の化合物\textbf{3}および、化合物\textbf{3}、\textbf{4}が加水分解されたジカルボン酸が挙げられる。
精製では、水で希釈した時に析出した固体を塩基性溶液に溶かし、弱酸性にした時に析出した固体を除いたのちに、強酸性にして白色の結晶を得た。化合物\textbf{4}は強酸性条件では水に溶解しない (白色固体として析出する) が、弱酸性-塩基性溶液には溶解していた。この性質を用いることで、他の弱酸性溶液での不純物をまず除くことができると期待される。また、化合物\textbf{4}は芳香環を有するため、tolueneへの溶解性が良いと期待される。対して化合物\textbf{3}は溶解性が悪いと考えられ、toluene溶液の熱時濾過により除去できると考えられる。toluene溶液の蒸留では、最初に$80\D$程度の留分が流出した。tolueneと水は$85\D$で共沸する\cite{solvent_bp}ことから、これにより水を除くことができたと考えられる。その後は$102\D$の留分としてtoluene (沸点$110.6\D$) が得られ、残渣を冷却すると\textbf{4}が析出した。\ce{^1H NMR}スペクトル (図\ref{NMR_6-1-3}) では不純物は観測されず、純度よく精製できたと考えられる。

\subsubsection{実際の実験について}
各ステップの収率は\textbf{1}→\textbf{2}: 56\%、\textbf{2}→\textbf{3}: 48\%、\textbf{3}→\textbf{4}: 26\%であった。IRスペクトルおよび\ce{^1H} NMRスペクトルの帰属結果も、Appendixに示した。IRは各目的化合物の存在を支持した。融点測定値は、どの化合物もほぼ一定の融点を示し、かつ文献値と概ね一致した。\ce{^1H} NMRスペクトル(図\ref{NMR_6-1-3}、帰属結果は表\ref{NMR_6-1_attribute})では化合物\textbf{4}に由来するピークのみが観測されており、高純度で合成することができたと考えられる。

\paragraph{\textbf{1}→\textbf{2}}
蒸留を加熱と同時に行うと、ほぼ同量の水と黄色油状の産物が得られた。反応機構に示した通り、この反応は形式的に脱水反応であり、系から水を除去することで反応の進行が促進すると考えられる。また、表\ref{properties_6-1}に示した通り、化合物\textbf{1}と化合物\textbf{2}の融点を比較すると後者の方が$100\D$程度も低く、蒸留を用いることで効率的に分離ができると同時に水の除去ができると考えられる。

\paragraph{\textbf{2}→\textbf{3}}
活性化されたDiels-Alder反応であり、室温での撹拌で進行した。
TLC (hexane:EtOAc = 2:1) で反応を追跡したところ、UV照射下で化合物\textbf{2}は$R_f= 0.26$でピンク色に呈色した。無水マレイン酸に対応するスポットは確認されなかった。反応進行に伴い、$R_f = 0.59$の新たなスポットが出現した。これが産物と考えられる。どちらのスポットも、リンモリブデン酸硫酸溶液を噴射し熱すると青色に呈色した。
反応開始後45 minに固体が急激に析出した。溶液の部分を展開したところ、$R_f= 0.26$の点のみが確認されたため、化合物\textbf{3}はほぼ完全に結晶として析出したと考えられる。

析出した固体は黄色がかっていたが、冷却した\ce{Et2O}で洗浄すると白色になった。析出した化合物\textbf{3}の結晶に、化合物\textbf{2}が吸着していたと考えられる。化合物\textbf{2}は\ce{Et2O}に可溶であるので、洗浄操作により除去ができたと考えられる。

\paragraph{\textbf{3}→\textbf{4}}
硫酸溶媒下で脱水反応を行った。上で示した精製過程を経て、白色固体を得た。IRスペクトルでは$1758$ cm$^{-1}$に\ce{C=O}伸縮に対応するピークが観測された (図\ref{IR_6-1-3_attribute})。また、化合物\textbf{3}と比較すると、芳香族\ce{C-H}伸縮に対応する1400-1450 cm$^{-1}$の領域に新たに吸収ピークが出現しており、図\ref{scheme_6-1}に示した反応様式を支持する。
\ce{^1H} NMR (図\ref{NMR_6-1-3}) では、化合物\textbf{4}に帰属されるピークのみが観測された。

%\subsection{Conclusion}

\section{3-methyl-3-(\textit{p-}tolylsulfonyl)-cyclopentanone の合成}
\subsection{Purpose and Background}
図\ref{scheme_6-2}に示したスキームにより、3-methyl-3-(\textit{p-}tolylsulfonyl)-cyclopentanone (\textbf{8}) を合成する。
Section.1と同じ出発物\textbf{1}を用い、塩基性の水を溶媒として(やや不安定な)化合物\textbf{5}を合成したのち、化合物\textbf{7}を酸性条件下で1,4-付加させて最終生成物の\textbf{7}を得る。
\begin{figure}[htbp]
\begin{center}
\includegraphics[width = 15 cm]{scheme_6-2.png}
\caption{化合物\textbf{7}の合成スキーム}%要修正
\label{scheme_6-2}
\end{center}
\end{figure}

\subsection{Experimental}
\begin{enumerate}
\item ミツ口フラスコに\ce{NaOH} 1.15 g、水 100 mLを加え、撹拌して均一な溶液とした。
\item 油浴で$120\D$に加熱し、穏やかに還流させながら滴下管から化合物\textbf{1} 12.60 gを5 minかけて滴下した。 
\item TLCにより反応の進行をモニターし、20 min後に出発物 ($R_f = 0.29
$) の消失を確認した。滴下終了後30 min撹拌し、反応を終了とした。
\item 氷冷により室温まで冷却したのち、\ce{NaCl} 10.56 gを加え、\ce{Et2O} 50 mLで3回抽出し、合わせた有機層をBrineで洗浄したのち、無水\ce{Na2SO4}で乾燥、濾別して得られたエーテル溶液を、エバポレーターによる減圧下 ($35\D$/300 mmHg) で溶媒を留去し、8.02 gに減容した。
\item 得られた黒色の液体を減圧蒸留した。はじめ減圧すると液体が沸騰し、沸騰がおさまったのちに加熱すると$44\D$/20 Torrで1つ目の留分が、$75\D$/19 Torrで2つ目の留分 (無色油状, 0.98 g, 9.2\%) が得られた。蒸気の温度が$50\D$まで低下したところで蒸留操作を終了した。
\item 得られた無色油状の2つ目の留分のGC、IRスペクトルを取得した。
\item 白色固体\textbf{7} 9.95 gを水に溶かし50 mLとした。
\item 三角フラスコに液体\textbf{6} 0.80 g、先ほど調整した\textbf{7}\textit{aq.}のうち 15 mL、1 M \ce{HCl}\textit{aq.} 10 mLを加えると色固体が生じた。
\item さらに撹拌を続けると、白色の濁りが消失したのち、10 min程度経過後に再び白色固体が得られた。
\item 合計15 minの撹拌後、得られた溶液を吸引濾過し、得られた白色固体を水、イソプロピルアルコール (IPA)、\ce{Et2O}の順に洗浄し、風乾したところ白色固体 (0.05 g) が得られた。この固体の融点の測定値は$88-89\D$ (文献値: $86-87\D$) であった。
\item しばらく減圧を続けたところ濾液が白濁したため再度吸引濾過を行い、氷冷したIPA、\ce{Et2O}の順に洗浄したところ、白色固体 (0.63 g) が得られた。この固体の融点の測定値は$90\D$であった。1次結晶、2次結晶を合わせた収量は 0.68 g (34\%) であった。
\item 1次結晶、2次結晶を合わせ、IRスペクトル (図\ref{IR_6-2-2}) および\ce{^1H} NMRスペクトル (図\ref{NMR_6-2-2}) を取得した。
\end{enumerate}

\paragraph{Appendix: \textbf{1}→\textbf{5}の再合成}
1度目の反応で、1段階目の反応の収率が著しく低かったため、化合物\textbf{1}の滴下速度を小さくして再合成を試みた。
\begin{enumerate}
\item ミツ口フラスコに\ce{NaOH} 1.15 g、水 100 mLを加え、撹拌して均一な溶液とした。
\item 油浴で$120\D$に加熱し、穏やかに還流させながら滴下管から化合物\textbf{1} 12.02 gを40 minかけて滴下した。 
\item 滴下終了後20 min撹拌したのち加熱を終了した。
\item 氷冷により室温まで冷却したのち、\ce{Et2O} 50 mLで2回抽出した。水層を\ce{NaCl}で飽和させたのち、\ce{Et2O} 50 mLで洗浄し、合わせた有機層をBrineで洗浄した。無水\ce{Na2SO4}で乾燥、濾別して得られたエーテル溶液を、エバポレーターによる減圧下 (300 mmHg) で溶媒を留去し減容した。
\item 得られた黒色の液体を減圧蒸留した。はじめ減圧すると液体が沸騰し、沸騰がおさまったのちに加熱すると、$79\D$/27 Torrで無色油状の留分 (3.42 g, 33.9\%) が得られた。蒸気の温度が$50\D$まで低下したところで蒸留操作を終了した。
\end{enumerate}

\subsection{Result and Discussion}
\subsubsection{反応機構}
図\ref{mechanism_6-2}に記した。分子内アルドール縮合により5員環を形成し、化合物\textbf{5}が生じる。分子間反応が競合して生じると考えられる。その後化合物\textbf{6}が1,4-付加して化合物\textbf{7}が生成する。この際、1,2-付加と1.4-付加の可能性があるが、求核剤はsoftであり、LUMOの係数がより大きいアルケンの末端に付加し、1.4-付加体が得られると考えられる。

\subsubsection{実際の反応について}
\paragraph{\textbf{1}→\textbf{5}}
一回めの反応では、収率が9.2\%と極めて低かった。TLCによる反応追跡では、UV照射下で出発物質($R_f = 0.29$)が反応進行に伴って消失し、新たに$R_f = 0.74, 0.19$、および0.2以下の多数のスポットが観測された。蒸留により精製後の化合物\textbf{5}では$R_f = 0.19$にスポットが見られ、蒸留後の残渣 (黒色、3 g程度) には$R_f = 0.74$および0.2以下の多数のスポットが観測された。また、抽出時の水層を同様に展開したところ、極薄い$R_f = 0.19$のスポットのみが見られた。このことから、若干の水層への化合物\textbf{5}の残存および、多数の副生物が収率低下の原因であることが考えられる。

副生物として、分子間アルドール反応による生成物が考えられる。分子内反応は化合物\textbf{1}の1次反応であるのに対して、分子間反応は2次反応であると考えられ、反応系中の化合物\textbf{1}濃度を低く保つことが分子間反応を抑制するのに重要であると考えられる。実際、滴下速度を小さくした再実験では、収率が34\%に向上した。

蒸留精製後のIRスペクトルを図\ref{IR_6-2-1}、帰属結果を表\ref{IR_6-2-1_attribute}に示す。共役したカルボニル\ce{C=O}伸縮に対応する吸収ピークが見られ、化合物\textbf{5}の結果を支持する。
水に由来すると考えられる\ce{O-H}伸縮ピークが見られ、脱水が不十分であったと考えられる。

\paragraph{\textbf{5} + \textbf{6}→\textbf{7}}
白色固体\textbf{6}は、撹拌すると水に溶解した。先に得られた化合物\textbf{5}が少量だったため、テキストの1/4量で実験を行った。
塩酸を加え酸性にすると、はじめ未反応の化合物\textbf{6}がプロトン化されて析出すると考えられる。その後に、撹拌を続けると1,4付加が起こり、化合物\textbf{6}に由来する白濁が解消したのちに結晶性の化合物\textbf{7}により反応溶液が白く濁ったと考えられる。酸性条件では、化合物\textbf{5}の酸素原子がプロトン化されて、反応が進行すると考えられる。

精製過程において、予測される不純物としては未反応の化合物\textbf{5},\textbf{6}である。これらは等量を用いているためどちらも残存の可能性がある。化合物\textbf{6}は水およびIPA、化合物\textbf{5}はエーテルへの溶解度が大きいと考えられ、これらの溶媒による洗浄で除くことができると考えられる。さらに、今回1次結晶が少なかった原因としては撹拌が不十分であり反応の進行が不十分であったか、結晶化が十分に起きていなかったことが考えられる。

なお、洗浄前の時点で結晶の量は洗浄後と大きくは変わらなかったため、洗浄過程における流出は主要因ではない。しかしながら、この過程でのロスを最小限にするため、2次結晶の洗浄では氷冷した溶媒を用いて洗浄した。

吸引濾過の濾液中に残った、結晶化していなかった化合物\textbf{7}が2次結晶として得られた要因として、しばらく真空引きを続けていたためにエーテルが揮発し、反応溶液が低温になっていたことが挙げられる。これにより、2次結晶が収率高く得られた可能性がある。2次結晶においては純度の低下が懸念されたが、融点測定の結果は、一定の文献値とほぼ一致する融点を示し、純度は十分高いと考えられる。

得られた白色結晶のIRスペクトルの帰属結果は表\ref{IR_6-2-2_attribute}に示した。\ce{C=O}伸縮およびスルホニル基に由来すると考えられる吸収ピークが見られ、化合物\textbf{7}の構造を支持する。

\ce{^1H} NMR結果の帰属結果を表\ref{NMR_6-2_attribute}に示した。水に由来する1.56 ppm程度のピーク以外は、全て化合物\textbf{7}に帰属され、純度よく合成できたと考えられる。

%\subsection{Conclusion}

%%参考文献
\begin{thebibliography}{99}
\bibitem{nmr_solvent}
Hugo E. Gottlieb, Vadim Kotlyar, and
Abraham Nudelman, 
NMR Chemical Shifts of Common
Laboratory Solvents as Trace Impurities
J. Org. Chem. 1997, 62, 7512-7515
\bibitem{solvent_bp}
\url{https://www.tcichemicals.com/assets/cms-pdfs/organic-solvents-j.pdf}
\bibitem{warren}
ウォーレン有機化学下巻 第2版 J.Clayden, N.Greeves, S.Warern 著、野依良治, 奥山格, 柴崎正勝, 檜山為二郎 監訳. pp.864-870.
\end{thebibliography}

\newpage
\section{Appendix}
\subsection{物性など}
\begin{table}[htp]
\caption{\textbf{1}→\textbf{4}の化合物の物性、等量一覧}
\begin{center}
\begin{tabular}{cccccc}
\toprule
compound & Mw & weight / g & mmol & m.p. / $\D$ & b.p. / $\D$\\
\midrule
\textbf{1} & 114.14 & 21.00 & 184.0 & - & 191\\
\textbf{2} & 94.13 & 17.67 (theoretical) & 184.0 & - & 93\\
  &  & 9.70 (used) & 100.9 & \\
無水マレイン酸 & 98.08 & 8.09  & 82.5 & \\
\textbf{3} & 194.19 & 17.11 (theoretical) & 82.5 & 68-71 & \\
  &  & 4.01 (used) & 20.6 & \\
\textbf{4} & 178.17 & 3.67 (theoretical) & 20.6 & 146-147 & \\
\bottomrule
\end{tabular}
\end{center}
\label{properties_6-1}
\end{table}%

\begin{table}[htp]
\caption{\textbf{1}→\textbf{7}の化合物の物性、等量一覧}
\begin{center}
\begin{tabular}{cccccc}
\toprule
compound & Mw & weight / g & mmol & m.p. / $\D$ & b.p. / $\D$\\
\midrule
\textbf{1} & 114.14 & 12.60 & 184.0 & - & 191\\
\textbf{5} & 96.13 & 10.11 (theoretical) & 105.1 & & 108 (75 mmHg) \\%沸点はGCの考察に必要
  &  & 0.80 (used) & 9.35 & \\
\textbf{6} & 266.18 (4水和物) & 2.5 & 9.40 & & \\
\textbf{7} & 238.3 & 2.23 (theoretical) & 9.35 & 86-87 & \\
\bottomrule
\end{tabular}
\end{center}
\label{properties_6-2}
\end{table}%
 
\newpage
\subsection{反応機構}
\begin{figure}[htbp]
\begin{center}
\includegraphics[width = 15 cm]{mechanism_6-1.jpg}
\caption{proposed mechanism for 6-1}
\label{mechanism_6-1}
\end{center}
\end{figure}

\begin{figure}[htbp]
\begin{center}
\includegraphics[width = 15 cm]{mechanism_6-2.jpg}
\caption{proposed mechanism for 6-2}
\label{mechanism_6-2}
\end{center}
\end{figure}

\newpage

\newpage
\subsection{IR}
\begin{table}[htp]
\caption{IR: \textbf{3} の帰属 (majorなもののみ)}
\begin{center}
\begin{tabular}{cc cc}
\toprule
No. & Wavenumber [cm-1] & Intensity & attribution \\
\midrule
1 & 2924 & strong & nujol\\
2 & 2853 & strong & nujol\\
3 & 1660 & mid & \ce{C=O} stretch\\
5 & 1463 & mid & nujol\\
6 & 1377 & mid  & nujol\\
\bottomrule
\end{tabular}
\end{center}
\label{IR_6-1-2_attribute}
\end{table}%

\begin{table}[htp]
\caption{IR: \textbf{4} の帰属 (majorなもののみ)}
\begin{center}
\begin{tabular}{cc cc}
\toprule
No. & Wavenumber [cm-1] & Intensity & attribution \\
\midrule
14 & 2954 & strong & nujol\\
15 & 2923 & strong & nujol\\
16 & 2853 & mid & nujol\\
19 & 1758 & strong  & \ce{C=O} stretch\\
22 & 1461 & mid & nujol\\
23 & 1401 & week & 芳香族\ce{C-H} stretch\\
24 & 1377 & mid & nujol\\
\bottomrule
\end{tabular}
\end{center}
\label{IR_6-1-3_attribute}
\end{table}%

\begin{table}[htp]
\caption{IR: \textbf{5} の帰属 (majorなもののみ)}
\begin{center}
\begin{tabular}{cc cc}
\toprule
No. & Wavenumber [cm-1] & Intensity & attribution \\
\midrule
1 & 3492 & strong (broad) & O-H stretch\\
3 & 2916 & mid & C-H stretch \\
4 & 2857 & mid & C-H stretch?\\
5 & 1698 & strong & \ce{C=O} stretch\\
6 & 1618 & mid & \ce{C=C} stretch?\\
\bottomrule
\end{tabular}
\end{center}
\label{IR_6-2-1_attribute}
\end{table}%

\begin{table}[htp]
\caption{IR: \textbf{7} の帰属 (majorなもののみ)}
\begin{center}
\begin{tabular}{cc cc}
\toprule
No. & Wavenumber [cm-1] & Intensity & attribution \\
\midrule
14 & 2954 & strong & nujol\\
15 & 2923 & strong & nujol\\
16 & 2853 & mid & nujol\\
19 & 1758 & strong  & \ce{C=O} stretch\\
22 & 1461 & mid & nujol\\
23 & 1401 & week & 芳香族\ce{C-H} stretch\\
24 & 1377 & mid & nujol\\
\bottomrule
\end{tabular}
\end{center}
\label{IR_6-2-2_attribute}
\end{table}%

\begin{figure}[htbp]
\begin{center}
\includegraphics[width = 15 cm]{IR_6-1-2.pdf}
\caption{IR: 化合物\textbf{3} (nujol)}
\label{IR_6-1-2}
\end{center}
\end{figure}

\begin{figure}[htbp]
\begin{center}
\includegraphics[width = 15 cm]{IR_6-1-3.pdf}
\caption{IR: 化合物\textbf{4} (nujol)}
\label{IR_6-1-3}
\end{center}
\end{figure}

\begin{figure}[htbp]
\begin{center}
\includegraphics[width = 15 cm]{IR_6-2-1.pdf}
\caption{IR: 化合物\textbf{5}}
\label{IR_6-2-1}
\end{center}
\end{figure}

\begin{figure}[htbp]
\begin{center}
\includegraphics[width = 15 cm]{IR_6-2-2.pdf}
\caption{IR: 化合物\textbf{7} (nujol)}
\label{IR_6-2-2}
\end{center}
\end{figure}

\newpage
\subsection{GC}
\begin{table}[htp]
\caption{GCの帰属}
\begin{center}
\begin{tabular}{ccccc}
\toprule
time [s] & height & Area & N(理論段数) & species\\
\midrule
70.77 & 812 & 20281.5 & 1143.8 & hexane\\
102.7 & 4.3 & 104.7 & 2594.2 & noise? \\
164.4 & 5.7 & 190.5 & 3401.4 & 化合物\textbf{1}\\
166.3 & 7.7 & 190.5 & 6388.5 & 化合物\textbf{1}\\
183.8 & 80.4 & 3029.1 & 3386.5 & 化合物\textbf{5} \\
\bottomrule
\end{tabular}
\end{center}
\label{GC_6-2_attribute}
\end{table}%
サンプルの注入途中にシリンジが止まったため、ピークが広がってしまっている。hexaneのピークも山が二つあるような形状をしており、化合物\textbf{1}に対応すると考えられるピークが2つ検出されたのはこれが要因と考えられる。\\
\ref{properties_6-2}に示した沸点の情報より、\textbf{5}の沸点は1気圧の下で$180\D$程度と換算され、化合物\textbf{1}の$191\D$と同程度である。よって同程度の流出時間でシグナルが得られることと矛盾しない。

\begin{figure}[htbp]
\begin{center}
\includegraphics[width = 15 cm]{GC_6-2-1.pdf}
\caption{GC (蒸留後の化合物\textbf{5}) 充填材: Thermon-3000, Length 4.1 m, Diameter 3.2 mm.}
\label{GC_6-2}
\end{center}
\end{figure}


\newpage
\subsection{NMR}
化合物\textbf{4}は2種類の水素のみを持ち、簡単に帰属される。溶媒分子のケミカルシフトは\cite{nmr_solvent}の記載を参考に帰属した。

化合物\textbf{7}について、まず積分比3程度のシグナルは、芳香環に結合したメチル基の\ce{H^a}がより低磁場シフトすると考えられるため下のように帰属される。芳香族領域の2つのシグナルは、電子吸引性のスルホニル基に隣接した\ce{H^c}が低磁場シフトすると考えられるため下のように帰属される。

そのほかの\ce{H^e}から\ce{H^j}に対しては、同じ炭素に結合したものもプロキラルであり等価でないためケミカルシフトが等しくなく、ジェミナルカップリングが観測される。\cite{warren}によると、カルボニル基などの電子吸引基に隣接する炭素に結合する水素間のジェミナルカップリングは特に大きい。2.18 ppm, 3.02 ppmのダブレットはJが等しく、18 Hzのジェミナルカップリングを有する\ce{H^e, H^f}に帰属した。

残りのピークは、複雑に分裂しており判別が困難である。理論上、ジェミナルカップリングおよびcisおよびtransで隣接する水素とそれぞれ異なるカップリング定数で結合し、ddd(8本)になると予想される。
ジェミナルカップリング定数は、\ce{H^i, H^j}の間でより大きいと予想されるが、NMRスペクトルからは判別ができなかった。ケミカルシフトもほぼ同程度であり、決定ができない。

\begin{figure}[htbp]
\begin{center}
\includegraphics[width = 10 cm]{NMR_6_attribute.png}
\caption{NMR}
\label{NMR_6_attribution}
\end{center}
\end{figure}

\begin{table}[htp]
\caption{化合物\textbf{4}のNMRの帰属}
\begin{center}
\begin{tabular}{cccc}
\toprule
chemical shift [ppm] & integration ratio & coupling & species\\
\midrule
1.56 & (6.23) & s & (\ce{H2O}) \\
2.68 & 6.22 & s & \ce{H^a}\\
7.26 & - & s & (\ce{CHCl3})\\
7.51 & 2.00 & s & \ce{H^b}\\
\bottomrule
\end{tabular}
\end{center}
\label{NMR_6-1_attribute}
\end{table}%

\begin{table}
\caption{化合物\textbf{7}のNMRの帰属}
\begin{center}
\begin{tabular}{cccc}
\toprule
chemical shift [ppm] & integration ratio & coupling & species\\
\midrule
1.42 & 3.00 & s & \ce{H^d}\\
1.56 & - & broad & (\ce{H2O}) \\
1.95 & 1.02 & m & \{\ce{H^g, H^h}\}または\{\ce{H^i, H^j}\}\\
2.18 & 1.02 & d (J = 18 Hz) & \{\ce{H^e, H^f}\}\\
2.34 & 1.02 & m & \{\ce{H^g, H^h}\}または\{\ce{H^i, H^j}\}\\
2.42 & 3.07 & s & \ce{H^a}\\
2.57 & 1.03 & m & \{\ce{H^g, H^h}\}または\{\ce{H^i, H^j}\}\\
2.78 & 1.00 & m & \{\ce{H^g, H^h}\}または\{\ce{H^i, H^j}\}\\
3.02 & 1.01 & d (J = 18 Hz) & \ce{H^e, H^f}\\
7.38 & 2.00 & d (J = 8 Hz) & \ce{H^b}\\
7.77 & 2.00 & d (J = 8 Hz) & \ce{H^c}\\
\bottomrule
\end{tabular}
\end{center}
\label{NMR_6-2_attribute}
\end{table}%


\begin{figure}[htbp]
\begin{center}
\includegraphics[width = 15 cm]{NMR_6-1-3.pdf}
\caption{NMR: \textbf{4} (溶媒: \ce{CDCl3})}
\label{NMR_6-1-3}
\end{center}
\end{figure}



\begin{figure}[htbp]
\begin{center}
\includegraphics[width = 15 cm]{NMR_6-2-2.pdf}
\caption{NMR: \textbf{7} (溶媒: \ce{CDCl3})}
\label{NMR_6-2-2}
\end{center}
\end{figure}



\end{document}
\documentclass{ltjsarticle}
%%%package読み込み
\usepackage{amsmath}
\usepackage{amssymb}
\usepackage{amsfonts}
\usepackage{mathtools}
\usepackage{bm}
% \usepackage{tikz} % ★消去: 代わりに graphicx 追加
% \usetikzlibrary{cd}
\usepackage{url}
\usepackage{graphicx} % ★追加: 図を挿入するため
\usepackage{float} % ★追加: 図の位置を制御するため
\usepackage{caption} % ★追加: 図のキャプションを柔軟に扱うため
%\usepackage{xcolor}
\usepackage{ascmac}
\usepackage{tcolorbox}
%\usepackage[dvipdfmx, setpagesize=false, bookmarks=true, bookmarksdepth=tocdepth, bookmarksnumbered=true, colorlinks=true, linkcolor=red]
\usepackage{hyperref}
\usepackage[version=4]{mhchem}
\usepackage{braket} % 追加した
\usepackage{booktabs}
\usepackage{bookmark}
%\usepackage[textwidth=45zw,lines=44]{geometry}
%\usepackage{pxjahyper}
%%%黒板太字
\newcommand{\N}{\mathbb{N}}
\newcommand{\Z}{\mathbb{Z}}
\newcommand{\Q}{\mathbb{Q}}
\newcommand{\R}{\mathbb{R}}
\newcommand{\C}{\mathbb{C}}
\newcommand{\F}{\mathbb{F}}
%%%約物
\newcommand{\abs}[1]{\left|#1\right|}
\newcommand{\lr}[1]{\left(#1\right)}
\newcommand{\st}{\; \mathrm{s.t.}\; }
\newcommand{\Ae}{\textrm{-a.e.}} 
%%%繰り返し
\newcommand{\pluss}[3]{#1_{#2}+\cdots+#1_{#3}}
\newcommand{\minuss}[3]{#1_{#2}-\cdots-#1_{#3}}
\newcommand{\timess}[3]{#1_{#2}\times\cdots\times #1_{#3}}
\newcommand{\leqs}[3]{#1_{#2}\leq\cdots\leq #1_{#3}}
\newcommand{\geqs}[3]{#1_{#2}\geq\cdots\geq #1_{#3}}
\newcommand{\opluss}[3]{#1_{#2}\oplus\cdots\oplus #1_{#3}}
\newcommand{\otimess}[3]{#1_{#2}\otimes\cdots\otimes #1_{#3}}
\newcommand{\commas}[3]{#1_{#2},\ldots,#1_{#3}}
%%%微分
\newcommand{\dx}[1]{\mathrm{d}#1}
\newcommand{\ddx}[1]{\frac{\mathrm{d}}{\mathrm{d}#1}}
\newcommand{\dydx}[2]{\frac{\mathrm{d}#1}{\mathrm{d}#2}}
\newcommand{\dydxn}[3]{\frac{\mathrm{d}^{#3}#1}{\mathrm{d}#2^{#3}}}
\newcommand{\del}[2]{\frac{\partial#1}{\partial#2}}
\newcommand{\dell}[2]{\frac{\partial^2#1}{{\partial#2}^2}}
\newcommand{\deln}[3]{\frac{\partial^{#3}#1}{{\partial#2}^{#3}}}
%%%
%%%演算子
%log type
\let\Re\relax
\DeclareMathOperator{\Re}{Re}
\let\Im\relax
\DeclareMathOperator{\Im}{Im}
\DeclareMathOperator{\sgn}{sgn}
\DeclareMathOperator{\sign}{sign}
\DeclareMathOperator{\Supp}{Supp}
\DeclareMathOperator{\tr}{tr}
\DeclareMathOperator{\Tr}{Tr}
\DeclareMathOperator{\Det}{Det}
\DeclareMathOperator{\Log}{Log}
\DeclareMathOperator{\rank}{rank}
\DeclareMathOperator{\diag}{diag}
\DeclareMathOperator{\corank}{corank}
\DeclareMathOperator{\Res}{Res}
\DeclareMathOperator{\Ker}{Ker}
\DeclareMathOperator{\coker}{coker}
\DeclareMathOperator{\Coker}{Coker}
\DeclareMathOperator{\Var}{Var}
\DeclareMathOperator{\Cov}{Cov}
\DeclareMathOperator{\sech}{sech}
\DeclareMathOperator{\csch}{csch}
\DeclareMathOperator{\arcsec}{arcsec}
\DeclareMathOperator{\arccot}{arccot}
\DeclareMathOperator{\arccsc}{arccsc}
\DeclareMathOperator{\arccosh}{arccosh}
\DeclareMathOperator{\arcsinh}{arcsinh}
\DeclareMathOperator{\arctanh}{arctanh}
\DeclareMathOperator{\arcsech}{arcsech}
\DeclareMathOperator{\arccsch}{arccsch}
\DeclareMathOperator{\arccoth}{arccoth}
\DeclareMathOperator{\grad}{grad}
\let\div\relax
\DeclareMathOperator{\div}{div}
\DeclareMathOperator{\rot}{rot}
%\DeclareMathOperator{\GL}{GL} % ★消去 : ここから↓
%\DeclareMathOperator{\SL}{SL}
%\DeclareMathOperator{\Sym}{Sym}
%\DeclareMathOperator{\Aut}{Aut}
%\DeclareMathOperator{\Inn}{Inn}
%\DeclareMathOperator{\Out}{Out}
%\DeclareMathOperator{\id}{id}
%\DeclareMathOperator{\pr}{pr}
%\DeclareMathOperator{\supp}{supp}
%\DeclareMathOperator{\diam}{diam}
%\DeclareMathOperator{\End}{End}
%\DeclareMathOperator{\Cl}{Cl}
%\DeclareMathOperator{\Hom}{Hom} % ★消去 : ここまで↑
%limit type
\DeclareMathOperator*{\argmin}{arg~min}
\DeclareMathOperator*{\argmax}{arg~max}
%%%
%%%定理
\usepackage{amsthm}
\theoremstyle{definition}
\newtheorem{lem}{補題}
\newtheorem*{lem*}{補題}
\newtheorem{prf}{証明}
\newtheorem*{prf*}{証明}
\newtheorem*{ex*}{Example}
\newtheorem*{rem*}{Remark}
\newenvironment{prb}[1]%
{\begin{itembox}[l]{\textbf{問題 #1}}}%
{\end{itembox}}
\newenvironment{sol}[2]%
{\setcounter{lem}{0}
\setcounter{prf}{0}
\par\noindent\textbf{解答 #1} (#2)\par}%
{\par\normalfont}

\renewcommand{\refname}{Reference}


%%%%%%%%%%%%%%%%%%%%%
\numberwithin{equation}{section}
%%%%%%%%%%%%%%%%%%%%%%

\newcounter{boxeddefcounter}
\newenvironment{problem}
{\refstepcounter{boxeddefcounter}\begin{itembox}[l]{問\theboxeddefcounter}}
{\end{itembox}}

%\usepackage[hang,small,bf]{caption}
%\usepackage[subrefformat=parens]{subcaption}
\captionsetup{compatibility=false}


\newcommand{\D}{^\circ\text{C}}
\newcommand{\ka}{\textasciitilde}


\pagestyle{myheadings}
\title{有機化学実験 文献調査}
\date{\today}
\author{Author: No.7 05253011 Fumiya Kashiwai / 柏井史哉}
\begin{document}
\maketitle
\markboth{Organic experiment Literature report No.7 05253011 Fumiya Kashiwai / 柏井史哉} {Organic experiment Literature report No.7 05253011 Fumiya Kashiwai / 柏井史哉}
%%ここまでタイトル

\newpage
\section{1,1-diphenylethyleneの合成}
\subsection{Purpose and Background}
グリニャール反応を経由して1,1-diphenylethyleneを合成する。
\begin{figure}[htbp]
\begin{center}
\includegraphics[width = 15 cm]{scheme_7-1.png}
\caption{7-1の反応スキーム}
\label{scheme_7-1}
\end{center}
\end{figure}

\subsection{Experimental}
\paragraph{Day1}
\begin{enumerate}
\item オーブンで乾燥させた三口フラスコに粉末状の\ce{Mg} 2.75 g, 無水\ce{Et2O} 8 mL, 化合物\textbf{3} 1.52 gを量りとった。外部への開放部は塩化カルシウム管に接続した。
\item 室温で激しく撹拌したところ、褐変して発熱した。
\item 化合物\textbf{3} 16.75 g, 無水\ce{Et2O} 41 mLを滴下ロートに加え、発熱が継続する程度に少量ずつ滴下した。
\item 滴下が完了したのち、発熱が終了するまで15 min程度撹拌した。
\item 滴下ロートに無水\ce{AcOEt} 4.40 g, 無水\ce{Et2O} 5 mLを加えた。
\item 氷浴により三口フラスコを冷却しながら、滴下ロートから溶液を10 minかけて滴下した。
\item 滴下完了後5 min撹拌し、飽和\ce{NH4Cl}\textit{aq.} 8 mLを滴下ロートから滴下し、白色~灰色の固体が析出した。デカントにより上澄を得たのち、残渣を\ce{Et2O} 30 mL $\times$ 5 で洗浄した。各洗液をTLCプレートにプロットし、UV照射下で呈色を確認し、呈色が薄くなったことを確認した。
\item 混入した固体をひだ付きろしにより除去したところ、濾紙を通過した。
\item 二重にした濾紙を用いて吸引濾過を行い、固体を完全に除去した。
\item 無水\ce{Na2SO4}を加え、乾燥、ひだ付きろしによる濾過により固体を除去した。
\item エバポレーション ($35\D$/300 mmHg) により溶媒を留去し、黄色溶液を得た。
\end{enumerate}

\paragraph{Day2}
\begin{enumerate}
\item Day1から放置した黄色溶液から、黄色固体 (6.60 g, 33.3\%) が析出していた。
\item 20\% \ce{H2SO4}\textit{aq.} 10 mL に黄色固体を溶解し、$130\D$の湯浴で還流した。TLC (hexane:EtOAc = 2:1) で反応を追跡し、1 h還流したのち、撹拌を停止して室温まで冷却した。
\item 反応溶液を水で希釈し、分液ロートに水で洗い込んだ。\ce{Et2O} 30 mL で抽出し、有機層をBrineで洗浄、無水\ce{MgSO4}で乾燥、濾過ののち、溶媒をエバポレーター ($35\D$/300 mmHg) で留去した。
\end{enumerate}
\paragraph{day3}
\begin{enumerate}
\item 得られた油状液体を減圧蒸留し、$110\D/22mmHg$で一つ目の留分(無色油状, 微量)、$125\D$/21 mmHgで二つ目の留分(無色油状, 0.95 g, 16.1\%) を得た。
\item 得られた無色油状の液体の\ce{^1H} NMRスペクトル(図\ref{NMR_7-1})、IRスペクトル(図\ref{IR_7-1})を取得した。
\end{enumerate}

\subsection{Result and Discussion}
Grignard試薬\textbf{4}は強力な塩基であり、水と反応する。具体的には、\textbf{4}は水からプロトンを引き抜いて
ベンゼンとマグネシウム塩が生成する副反応が生じる。そのため水を除いた反応系で反応を行うことが重要である(\textbf{課題1})。

また、このグリニャール反応の副反応として、酢酸エチルの一置換体(化合物\textbf{9})や、Grignard試薬が塩基として働き、酢酸エチルの$\alpha$水素を引き抜きエノラートを作ることによる縮合反応が生じる可能性がある。この経路についても反応機構(図\ref{mechanism_7-1})に示した。同時にWurtzカップリングを経てビフェニルが生じる可能性がある\cite{Wurtz}(\textbf{課題2})。

また、粒状の\ce{Mg}を用いているため、表面に酸化被膜が生じて、Grignard試薬を生じる反応が生じにくくなる。一度反応が生じると、発熱反応であるために自触媒的に反応が進行するが、反応を開始するためには反応系を温めるか、還元剤として\ce{I2}やジブロモエタンを加えることが効果的である(\textbf{課題3})。なお、今回の実験では室温での攪拌のみで反応が開始した。

反応溶液中、エーテルがGrignard試薬の\ce{Mg}に配位し、4配位状態を作っていると考えられる(図\ref{grignard})。この構造により、エーテルやTHFなどの配位性の溶媒がGrignard反応に用いられる。(\textbf{課題4})。

推定された反応機構は図\ref{mechanism_7-1}に示した。Grignard反応後の、\ce{NH4Cl}によるクエンチの後、TLC (hexane:EtOAc = 2:1) で$R_f = 0.81, 0.69$の2つのスポットがUV照射下でピンク色に呈色した。その後、硫酸酸性下での1 hの撹拌ののち、$R_f=0.81$のスポットのみがUV照射下で呈色した。このことから、$R_f=0.81$のスポットは化合物\textbf{1}、$0.69$のスポットが化合物\textbf{6}に対応すると考えられる。ある程度の脱水反応がすでに進行していたと考えられるが、grignard試薬が塩基として働いて脱水が生じた可能性がある。この場合、この反応もGrignard試薬を消費する副反応となり収率を低下させる原因となる。

化合物\textbf{1}の沸点は$277\D$であり、減圧蒸留中の21 mmHgでは$150\D$程度となると考えられる。今回の蒸留では$125\D$で流出したため、やや不純物が混じっていた可能性がある。再蒸留により純度を上げられる可能性があるが、今回得られた液体は0.95 gと少なかったため再蒸留は行わなかった。

IRスペクトルは多数のC-Hによると思われるピークが出現していた。
3000-3500 cm$^{-1}$にbroadに出現すると考えられる、\ce{OH}基に由来するピークは観測されなかったため、化合物\textbf{6}からの脱水が進行したことが確認される。

\ce{^1H} NMRスペクトルでは、図\ref{NMR_7-1_attribute}に示した通り、目的化合物\textbf{1}に対応するスペクトルが観測された。芳香環の10Hは、本来4:4:2に分かれるはずであるが、置換基の影響が小さいために無置換のbenzeneとほぼ同程度のケミカルシフトのピークとなり、重なり合っていた。主な不純物として芳香族領域に見られるピークはbiphenylのケミカルシフトの文献値\cite{biphenyl}%https://www.rsc.org/suppdata/cc/c3/c3cc45132a/c3cc45132a.pdf
とほぼ合致し、分裂様式も矛盾しない。なお最も高磁場側に生じると期待される1Hは、10.00Hに埋もれていると考えられる。biphenylの沸点は$255\D$、22mmHgで$135\D$程度であり、一つ目の留分($110\D$/22mmHg)がbiphenylであったと考えられる。2つ目の留分に対してもbipheylが相当量コンタミネーションしてしまっていたと考えられる。
\subsection{Conclusion}


\newpage
\section{4-Acetylcumeneの合成}
\subsection{Purpose and Background}
\begin{figure}[htbp]
\begin{center}
\includegraphics[width = 15 cm]{scheme_7-2.png}
\caption{7-2の反応スキーム}
\label{scheme_7-2}
\end{center}
\end{figure}

\subsection{Experimental}
\paragraph{Day2}
\begin{enumerate}
\item オーブンで乾燥させた三口フラスコに\ce{AlCl3} 7.92 g, 無水1,2-dichloroethane 30 mLを量りとり、塩化カルシウム管を経てアルカリトラップに接続した。
\item 氷浴による冷却下で撹拌し、黄色の懸濁液となるまで撹拌した。
\item 滴下ロートに化合物\textbf{8} 4.3 mLを量りとり、氷浴下で滴下した。
\item 化合物\textbf{7} 4.31 gを滴下ロートに量りとり、氷浴下で5 minかけて滴下した。滴下完了後、1,2-dichloroethane 5 mL程度で滴下ロートを洗い込んだ。
\item TLC (hexane:EtOAc = 2:1) で反応を追跡し、75 minの反応後に撹拌を停止した。
\item 氷50 gに反応溶液を注ぎ、濃\ce{HCl}\textit{aq.}5 mL$\times 2$で洗い込んだ。撹拌すると白色の沈澱が解消した。
\item 反応溶液を分液ロートに移し、\ce{CHCl3}10 mL$\times 2$で洗い込み、分液を行った。合わせた有機層を、水で洗浄、無水\ce{Na2SO4}で乾燥、濾過し、エバポレーション ($35\D$/100 mmHg) により溶媒を留去した。濾過の際、漏斗が転倒して一部の溶液をロスした。
\item 残渣を少量の\ce{CHCl3}により30 mL ナスフラスコに洗い込み、再度エバポレーション ($35\D$/100 mmHg) によって溶媒を留去した。
\end{enumerate}

\paragraph{day3}
\begin{enumerate}
\item 得られた残渣を減圧蒸留し、$105\D$/25 Torrで無色油状の留分 (3.02 g, 52.0\%) が得られた。
\item 得られた無色油状の液体の\ce{^1H} NMRスペクトル(図\ref{NMR_7-2})、IRスペクトル(図\ref{IR_7-2})を取得した。
\end{enumerate}

\subsection{Result and Discussion}


\subsection{Conclusion}


\section{Appendix}

\begin{table}[htp]
\caption{1,1-diphenylethylene合成における各化合物の物性、等量一覧}
\begin{center}
\begin{tabular}{cccccc}
\toprule
compound & Mw & weight / g & mmol & m.p. / $\D$ & b.p. / $\D$\\
\midrule
\textbf{3} & 157.01 & 18.27 & 116.4 & - & 156\\
Mg & 24.3 & 2.75 & 113.2 & & \\%沸点はGCの考察に必要
\textbf{4} & 181.3 & 20.3 (theoretical) & 113.2 & \\
\textbf{5} & 88.11 & 4.40 & 50.0 & & 77.1\\
\textbf{6} & 198.3 & 19.8 (theoretical) & 100 & &155 (12 mmHg) \\
 & & 6.60 (used) & 33.3 & \\
\textbf{1} & 180.25 & 6.0 (theoretical) & 33.3 & & 277\\ 
\bottomrule
\end{tabular}
\end{center}
\label{properties_6-2}
\end{table}%

\begin{table}[htp]
\caption{4-Acetylcumene合成における各化合物の物性、等量一覧}
\begin{center}
\begin{tabular}{cccccc}
\toprule
compound & Mw & weight / g & mmol & m.p. / $\D$ & b.p. / $\D$\\
\midrule
\textbf{7} & 120.19 & 4.3 & 116.4 & - & 152.4 \\
Mg & 24.3 & 2.75 & 113.2 & & 51\\%沸点はGCの考察に必要
\textbf{8} & 181.3 & 20.3 (theoretical) & 113.2 & \\
\textbf{2} & 88.11 & 4.40 & 50.0 & & 115-116 (12 mmHg)\\
\bottomrule
\end{tabular}
\end{center}
\label{properties_6-2}
\end{table}%



\begin{table}[htp]
\caption{IR: \textbf{1} の帰属 (majorなもののみ)}
\begin{center}
\begin{tabular}{cc cc}
\toprule
No. & Wavenumber [cm-1] & Intensity & attribution \\
\midrule
14 & 2954 & strong & nujol\\
15 & 2923 & strong & nujol\\
16 & 2853 & mid & nujol\\
19 & 1758 & strong  & \ce{C=O} stretch\\
22 & 1461 & mid & nujol\\
23 & 1401 & week & 芳香族\ce{C-H} stretch\\
24 & 1377 & mid & nujol\\
\bottomrule
\end{tabular}
\end{center}
\label{IR_7-1_attribute}
\end{table}%

\begin{table}[htp]
\caption{IR: \textbf{2} の帰属 (majorなもののみ)}
\begin{center}
\begin{tabular}{cc cc}
\toprule
No. & Wavenumber [cm-1] & Intensity & attribution \\
\midrule
14 & 2954 & strong & nujol\\
15 & 2923 & strong & nujol\\
16 & 2853 & mid & nujol\\
19 & 1758 & strong  & \ce{C=O} stretch\\
22 & 1461 & mid & nujol\\
23 & 1401 & week & 芳香族\ce{C-H} stretch\\
24 & 1377 & mid & nujol\\
\bottomrule
\end{tabular}
\end{center}
\label{IR_7-2_attribute}
\end{table}%


\begin{figure}[htbp]
\begin{center}
\includegraphics[width = 15 cm]{IR_7-1.pdf}
\caption{IR: 化合物\textbf{1}}
\label{IR_7-1}
\end{center}
\end{figure}

\begin{figure}[htbp]
\begin{center}
\includegraphics[width = 15 cm]{IR_7-2.pdf}
\caption{IR: 化合物\textbf{2}}
\label{IR_7-2}
\end{center}
\end{figure}

\begin{figure}[htbp]
\begin{center}
\includegraphics[width = 8 cm]{NMR_attribution_7.png}
\caption{NMRの帰属}
\label{NMR_7_attribute}
\end{center}
\end{figure}

\begin{table}[htp]
\caption{化合物\textbf{1}のNMRの帰属}
\begin{center}
\begin{tabular}{cccc}
\toprule
chemical shift [ppm] & integration ratio & coupling & species\\
\midrule
5.46 & 1.92 & s & \ce{H^a} \\
7.31-7.34 & 10.00 & m & \ce{H^b}, \ce{H^c}, \ce{H^d} + biphenyl\\
7.44 & 2.12 & t (J = 8 Hz) & diphenyl \\
7.59 & 2.08 & dd(J = 8, 1 Hz) & diphenyl\\
\bottomrule
\end{tabular}
\end{center}
\label{NMR_7-1_attribute}
\end{table}%

\begin{table}[htp]
\caption{化合物\textbf{2}のNMRの帰属}
\begin{center}
\begin{tabular}{cccc}
\toprule
chemical shift [ppm] & integration ratio & coupling & species\\
\midrule
1.27 & 6.30 & d & \ce{H^e} \\
2.58 & 2.89 & s & \ce{H^a}\\
2.97 & 1.00 & hep & \ce{H^d}\\
7.31 & 1.96 & s & \ce{H^c}\\
7.89 & 2.00 & s & \ce{H^b}\\
\bottomrule
\end{tabular}
\end{center}
\label{NMR_7-2_attribute}
\end{table}%


\begin{figure}[htbp]
\begin{center}
\includegraphics[width = 15 cm]{NMR_7-1.pdf}
\caption{\ce{^1H} NMR: 化合物\textbf{1} (溶媒: \ce{CDCl3})}
\label{NMR_7-1}
\end{center}
\end{figure}


\begin{figure}[htbp]
\begin{center}
\includegraphics[width = 15 cm]{NMR_7-2.pdf}
\caption{\ce{^1H} NMR: 化合物\textbf{2} (溶媒: \ce{CDCl3})}
\label{NMR_7-2}
\end{center}
\end{figure}


%%参考文献
\begin{thebibliography}{99}
\bibitem{nmr_solvent}
Hugo E. Gottlieb, Vadim Kotlyar, and
Abraham Nudelman, 
NMR Chemical Shifts of Common
Laboratory Solvents as Trace Impurities
J. Org. Chem. 1997, 62, 7512-7515
\end{thebibliography}

\end{document}